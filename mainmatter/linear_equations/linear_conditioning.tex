\section{Conditioning}%
\label{sec:conditioning}

The condition number for a given problem measures the sensitivity of the solution to the input data.
In order to define this concept precisely,
we consider a general problem of the form~$F(x, d) = 0$,
with unknown $x$ and data $d$.
The linear system~\eqref{eq:linear_system} can be recast in this form,
with the input data equal to $\vect b$ or $\mat A$ or both.
We denote the solution corresponding to perturbed input data $d + \Delta d$ by $x + \Delta x$.
The absolute and relative condition numbers are defined as follows.

\begin{definition}
    [Condition number for the problem $F(x, d) = 0$]
    The absolute and relative condition numbers with respect to perturbations of $d$ are defined as
    \[
        K_{\rm abs}(d) = \lim_{\varepsilon \to 0} \left( \sup_{\norm{\Delta d} \leq \varepsilon} \frac{\norm{\Delta x}}{\norm{\Delta d}} \right),
        \qquad
        K(d) = \lim_{\varepsilon \to 0} \left( \sup_{\norm{\Delta d} \leq \varepsilon} \frac{\norm{\Delta x} / \norm{x}}{\norm{\Delta d} / \norm{d}} \right).
    \]
    The short notation $K$ is reserved for the relative condition number,
    which is often more useful in applications.
\end{definition}

In the rest of this section,
we obtain an upper bound on the relative condition number for the linear system~\eqref{eq:linear_system} with respect to perturbations first of $\vect b$,
and then of $\mat A$.
We use the notation~$\norm{\placeholder}$ to denote both a vector norm on $\real^n$ and the induced operator norm on matrices.

\begin{proposition}
    [Perturbation of the right-hand side]
    \label{proposition:linear_perturbation_rhs}
    Let $\vect x + \Delta \vect x$ denote the solution to the perturbed equation $\mat A (\vect x + \Delta \vect x) = \vect b + \Delta \vect b$.
    Then it holds that
    \begin{equation}
        \label{eq:linear_perturbation_rhs}
        \frac{\norm{\Delta \vect x}}{\norm{\vect x}} \leq \norm{\mat A} \norm{\mat A^{-1}} \, \frac{\norm{\Delta \vect b}}{\norm{\vect b}},
    \end{equation}
\end{proposition}
\begin{proof}
    It holds by definition of $\Delta \vect x$ that $\mat A \Delta \vect x = \Delta \vect b$.
    Therefore, we have
    \begin{equation}
        \label{eq:linear_perturbation_rhs_to_rearrange}
        \norm{\Delta \vect x}
        = \norm{\mat A^{-1} \Delta \vect b}
        \leq \norm{\mat A^{-1}} \norm{\Delta \vect b}
        = \frac{\norm{\mat A \vect x}}{\norm{\vect b}} \norm{\mat A^{-1}} \norm{\Delta \vect b}
        \leq \frac{\norm{\mat A} \norm{\vect x}}{\norm{\vect b}} \norm{\mat A^{-1}} \norm{\Delta \vect b}.
    \end{equation}
    Here we employed~\eqref{eq:submultiplicative_mat_vec},
    proved in \cref{cha:vectors_and_matrices},
    in the first and last inequalities.
    Rearranging the inequality~\eqref{eq:linear_perturbation_rhs_to_rearrange},
    we obtain~\eqref{eq:linear_perturbation_rhs}.
\end{proof}
\Cref{proposition:linear_perturbation_rhs} implies that
the relative condition number of~\eqref{eq:linear_system} with respect to perturbations of the right-hand side is bounded from above by $\norm{\mat A} \norm{\mat A^{-1}}$.
\Cref{exercise:linear_sharp_inequality} shows that there are values of $\vect x$ and $\Delta \vect b$ for which the inequality~\eqref{eq:linear_perturbation_rhs} is an equality,
indicating that the inequality is sharp.

Studying the impact of perturbations of the matrix~$\mat A$ is slightly more difficult,
because this time the variation~$\Delta \vect x$ of the solution does not depend linearly on the perturbation of the data.
Before stating and proving the main result,
we show an ancillary lemma.
\begin{lemma}
    \label{lemma:linear_inverse_neumann}
    Let $\mat B \in \real^{n \times n}$ be such that $\norm{\mat B} < 1$ in some submultiplicative matrix norm~$\norm{\placeholder}$.
    Then $\mat I - \mat B$ is invertible and
    \begin{equation}
        \label{eq:linear_bound_inverse_perturbation_identity}
        \norm{(\mat I - \mat B)^{-1}}
        \leq \frac{1}{1 - \norm{\mat B}},
    \end{equation}
    where $\mat I \in \real^{n \times n}$ is the identity matrix.
\end{lemma}
\begin{proof}
    It holds for any matrix $\mat B \in \real^{n \times n}$ that
    \begin{equation}
        \label{eq:first_equation_neumann}
        \mat I - \mat B^{n+1} = (\mat I - \mat B)(\mat I + \mat B + \dotsb + \mat B^n).
    \end{equation}
    Since $\norm{\mat B} < 1$ in a submultiplicative matrix norm,
    both sides of the equation are convergent in the limit as $n \to \infty$.
    The left-hand side converges to the identity matrix $\mat I$,
    and the right-hand side converges as $n \to \infty$ because $\{\mat S_0, \mat S_1, \dotsc\}$ with
    \[
        \mat S_n := \mat I + \mat B + \dotsb + \mat B^n
    \]
    is a Cauchy sequence in the vector space of matrices endowed with the norm for which~$\norm{\mat B} < 1$.
    Indeed, by the triangle inequality and the submultiplicative property of the norm,
    it holds that
    \begin{align*}
        \norm{\mat S_{n+m} - \mat S_n}
        &= \norm{\mat B^{n+1} + \dotsb + \mat B^{n+m}} \\
        &\leq \norm{\mat B^{n+1}} + \dotsb + \norm{\mat B^{n+m}}
        \leq \norm{\mat B}^{n+1} + \dotsb + \norm{\mat B}^{n+m} \\
        &\leq  \frac{\norm{\mat B}^{n+1}}{1 - \norm{\mat B}} \xrightarrow[n \to \infty]{} \mat 0,
    \end{align*}
    where we employed the formula for a geometric series in the last inequality.
    Equating the limits of both sides of~\eqref{eq:first_equation_neumann},
    we obtain
    \[
        \mat I = (\mat I - \mat B) \sum_{i=0}^{\infty} \mat B^i.
    \]
    This implies that $(\mat I - \mat B)$ is invertible with inverse
    given by a so-called \emph{Neumann} series
    \begin{equation*}
        (\mat I - \mat B)^{-1} = \sum_{i=0}^{\infty} \mat B^i.
    \end{equation*}
    Applying the triangle inequality repeatedly,
    and then using the submultiplicative property of the norm,
    we obtain
    \[
        \forall n \in \nat,
        \qquad
        \norm*{\sum_{i=0}^{n} \mat B^i}
        \leq \sum_{i=0}^{n} \norm{\mat B^i}
        \leq \sum_{i=0}^{n} \norm{\mat B}^i
        = \frac{1}{1 - \norm{\mat B}}.
    \]
    where we used the summation formula for geometric series in the last equality.
    Letting $n \to \infty$ in this equation and
    using the continuity of the norm enables to conclude the proof.
\end{proof}

\begin{proposition}
    [Perturbation of the matrix]
    \label{proposition:linear_perturbation_matrix}
    Let $\vect x + \Delta \vect x$ denote the solution to the perturbed equation $(\mat A + \Delta \mat A) (\vect x + \Delta \vect x) = \vect b$.
    If $\mat A$ is invertible and $\norm{\Delta \mat A} < \norm{\mat A^{-1}}^{-1}$,
    then
    \begin{equation}
        \label{eq:linear_perturbation_matrix}
        \frac{\norm{\Delta \vect x}}{\norm{\vect x}}
        \leq \norm{\mat A} \norm{\mat A^{-1}} \frac{\norm{\Delta \mat A}}{\norm{\mat A}}
        \left(\frac{1}{1 - \norm{\mat A^{-1} \Delta \mat A}} \right).
    \end{equation}
\end{proposition}
\begin{proof}
    Left-multiplying both sides of the perturbed equation with $\mat A^{-1}$,
    we obtain
    \begin{equation}
        \label{eq:linear_perturbation_matrix_initial}
        \left(\mat I + \mat A^{-1} \Delta \mat A\right) (\vect x + \Delta \vect x) = \vect x
        \quad \Leftrightarrow \quad
        \left(\mat I + \mat A^{-1} \Delta \mat A\right) \Delta \vect x = - \mat A^{-1} \Delta \mat A \vect x.
    \end{equation}
    Since $\norm{\mat A^{-1} \Delta \mat A} \leq \norm{\mat A^{-1}} \norm{\Delta \mat A} < 1$ by assumption,
    we deduce from~\cref{lemma:linear_inverse_neumann} that the matrix on the left-hand side is invertible
    with a norm bounded as in~\eqref{eq:linear_bound_inverse_perturbation_identity}.
    Consequently,
    using in addition the assumed submultiplicative property of the norm,
    we obtain that
    \[
        \norm{\Delta \vect x}
        = \norm{(\mat I + \mat A^{-1} \Delta \mat A)^{-1} \mat A^{-1} \Delta \mat A \vect x}
        \leq \frac{\norm{\mat A^{-1} \Delta \mat A}}{1 - \norm{\mat A^{-1} \Delta \mat A}} \norm{\vect x}.
    \]
    which enables to conclude the proof.
\end{proof}
Using \cref{proposition:linear_perturbation_matrix},
we deduce that the relative condition number of~\eqref{eq:linear_system} with respect to perturbations of the matrix $\mat A$ is also bounded from above by $\norm{\mat A} \norm{\mat A^{-1}}$,
because the term between brackets on the right-hand side of~\eqref{eq:linear_perturbation_matrix} converges to 1 as $\norm{\Delta \mat A} \to 0$.


\Cref{proposition:linear_perturbation_rhs,proposition:linear_perturbation_matrix} show that
the condition number, with respect to perturbations of either~$\vect b$ or~$\mat A$,
depends only on $\mat A$.
This motivates the following definition.
\begin{definition}
    [Condition number of a matrix]
    The condition number of a matrix $\mat A$ associated with a vector norm $\norm{\placeholder}$ is defined as
    \[
        \kappa(\mat A) = \norm{\mat A} \norm{\mat A^{-1}}.
    \]
    The condition number for the $p$-norm,
    defined in \cref{definition:pnorm_vector},
    is denoted by $\kappa_p(\mat A)$.
\end{definition}
Note that the condition number $\kappa(\mat A)$ associated with an induced norm,
i.e.\ a matrix norm induced by a vector norm,
is at least one.
Indeed, since the identity matrix has induced norm 1,
it holds that
\[
    1 = \norm{\mat I} = \norm{\mat A \mat A^{-1}}\leq \norm{\mat A} \norm{\mat A^{-1}}.
\]

Since the 2-norm of an invertible matrix $\mat A \in \real^{n \times n}$ coincides with the spectral radius $\rho(\mat A^\t \mat A)$,
the condition number $\kappa_2$ corresponding to the $2$-norm is equal to
\[
    \kappa_2(\mat A) = \sqrt{\frac{\lambda_{\max}(\mat A^\t \mat A)}{\lambda_{\min}(\mat A^\t \mat A)}},
\]
where $\lambda_{\max}(\mat A^\t \mat A)$ and $\lambda_{\min}(\mat A^\t \mat A)$ are the maximal and minimal (both real and positive) eigenvalues of the matrix $\mat A^\t \mat A$.
\begin{example}
    [Perturbation of the matrix]
    Consider the following linear system
    with perturbed matrix
    \[
        (\mat A + \Delta \mat A)
        \begin{pmatrix}
            x_1 \\
            x_2
        \end{pmatrix}
        = \begin{pmatrix}
            0 \\
            0.01
        \end{pmatrix},
        \qquad
        \mat A
        = \begin{pmatrix}
            1 & 0 \\
            0 & 0.01
        \end{pmatrix},
        \qquad
        \Delta \mat A =
        \begin{pmatrix}
            0 & 0 \\
            0 & \varepsilon
        \end{pmatrix},
    \]
    where $0 < \varepsilon \ll 0.01$.
    Here the eigenvalues of $\mat A$ are given by $\lambda_1 = 1$ and $\lambda_2 = 0.01$.
    The solution when $\varepsilon = 0$ is given by $(0, 1)^\t$,
    and the solution to the perturbed equation is
    \[
        \begin{pmatrix}
        x_1 + \Delta x_1 \\
        x_2 + \Delta x_2
        \end{pmatrix}
        =
        \begin{pmatrix}
            0 \\
            \frac{1}{1 + 100 \varepsilon}
        \end{pmatrix}.
    \]
    Consequently, we deduce that, in the 2-norm,
    \[
        \frac{\norm{\Delta \vect x}}{\norm{\vect x}}
        = \abs*{\frac{100 \varepsilon}{1 + 100 \varepsilon}}
        \approx 100 \varepsilon
        = 100 \frac{\norm{\Delta \mat A}}{\norm{\mat A}}.
    \]
    In this case,
    the relative impact of perturbations of the matrix is close to $\kappa_2(\mat A) = 100$.
\end{example}
