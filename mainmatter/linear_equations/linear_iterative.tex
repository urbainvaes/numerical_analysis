\section{Iterative methods for linear systems}%
\label{sec:iterative_methods}
Iterative methods enjoy more flexibility than direct methods,
because they can be stopped at any point if the residual is deemed sufficiently small.
This generally enables to obtain a good solution at a computational cost that is significantly lower than that of direct methods.
In this section,
we present and study two classes of iterative methods:
basic iterative methods based on a splitting of the matrix $\mat A$,
and the so-called Krylov subspace methods.

\subsection{Basic iterative methods}%
\label{sub:basic_iterative_methods}
The basic iterative methods are particular cases of a general \emph{splitting method}.
Given a splitting of the matrix of the linear system as $\mat A = \mat M - \mat N$,
for a nonsingular matrix $\mat M \in \complex^{n \times n}$ and a matrix $\mat N \in  \complex^{n \times n}$,
together with an initial guess $\vect x^{(0)}$ of the solution,
one step of this general method reads
\begin{equation}
    \label{eq:linear_system_iterative_method}
    \mat M \vect x^{(k+1)} = \mat N \vect x^{(k)} + \vect b.
\end{equation}
For any choice of splitting,
the exact solution $\vect x_*$ to the linear system is a fixed point of this iteration,
in the sense that if $\vect x^{(0)} = \vect x_*$, then $\vect x^{(k)} = \vect x_*$ for all $k \geq 0$.
\Cref{eq:linear_system_iterative_method} is a linear system with matrix~$\mat M$,
unknown $\vect x^{(k+1)}$, and right-hand side $\mat N \vect x^{(k)} + \vect b$.
There is a compromise between the cost of a single step and the speed of convergence of the method.
In the extreme case where $\mat M = \mat A$ and $\mat N = 0$,
the method converges to the exact solution in one step,
but performing this step amounts to solving the initial problem.
In practice, in order for the method to be useful,
the linear system~\eqref{eq:linear_system_iterative_method} should be relatively simple to solve.
Concretely, this means that the matrix~$\mat M$ should be diagonal, triangular, block diagonal, or block triangular.
The error $\vect e^{(k)}$ and residual $\vect r^{(k)}$ at iteration $k$ are defined as follows:
\[
    \vect e^{(k)} = \vect x^{(k)} - \vect x_*,
    \qquad
    \vect r^{(k)} = \mat A \vect x^{(k)} - \vect b.
\]

\subsubsection{Convergence of the splitting method}%
\label{ssub:convergence_of_the_basic_splitting_method}
Before presenting concrete examples of splitting methods,
we obtain a necessary and sufficient condition for the convergence of~\eqref{eq:linear_system_iterative_method}
for any initial guess~$\vect x^{(0)}$.

\begin{proposition}
    [Convergence]
    \label{proposition:linear_convergence}
    The splitting method~\eqref{eq:linear_system_iterative_method} converges for any initial~$\vect x^{(0)}$
    if and only if $\rho(\mat M^{-1} \mat N) < 1$.
    In addition, for any $\varepsilon > 0$ there exists $K > 0$ such that
    \begin{equation}
        \label{eq:additional_claim}
        \forall k \geq K,
        \qquad
        \norm{\vect e^{(k)}} \leq \bigl(\rho(\mat M^{-1} \mat N) + \varepsilon\bigr)^k \norm{\vect e^{(0)}}.
    \end{equation}
\end{proposition}
\begin{proof}
    Let $\vect x_*$ denote the solution to the linear system.
    Since $\mat M \vect x_* - \mat N \vect x_* = \vect b$,
    we have
    \[
        \mat M (\vect x^{(k+1)} - \vect x_*) = \mat N (\vect x^{(k)} - \vect x_*).
    \]
    Using the assumption that $\mat M$ is nonsingular,
    we obtain that the error satisfies the equation
    \[
        \vect e^{(k+1)} = (\mat M^{-1} \mat N) \vect e^{(k)}.
    \]
    Applying this equality repeatedly,
    we deduce that
    \begin{equation}
        \label{eq:error_equation_basic_iterative_method}
        \vect e^{(k)} = (\mat M^{-1} \mat N) \vect e^{(k-1)} = \dots = (\mat M^{-1} \mat N)^{k} \vect e^{(0)}.
    \end{equation}

    \paragraph{Proof that $\rho(\mat M^{-1} \mat N) < 1$ is necessary for convergence.}
    We prove the equivalent claim that if $\rho(\mat M^{-1} \mat N) \geq 1$,
    then there exists $\vect x^{(0)}$ such that the method is not convergent.
    Indeed, assume that $\vect x^{(0)} = \vect x_* + \vect v^{(0)}$,
    where $\vect v^{(0)}$ is the eigenvector of $\mat M^{-1} \mat N$ associated with the eigenvalue of largest modulus.
    Then $\vect e^{(0)} = \vect v^{(0)}$ and the right-hand side of~\eqref{eq:error_equation_basic_iterative_method} does not converge to 0 in the limit as $k \to 0$,
    because
    \[
        \norm{(\mat M^{-1} \mat N)^{k} \vect e^{(0)}} = \rho(\mat M^{-1} \mat N)^k \norm{\vect v^{(0)}} \geq \norm{\vect v^{(0)}}.
    \]
    Thus, the condition $\rho(\mat M^{-1} \mat N) < 1 $ is necessary to ensure convergence for all initial guess $\vect x^{(0)}$.

    \paragraph{Proof that $\rho(\mat M^{-1} \mat N) < 1$ is sufficient for convergence.}
    In order to show that the condition is also sufficient,
    note that by~\eqref{eq:error_equation_basic_iterative_method}
    \[
        \forall k \geq 0, \qquad
        \norm{\vect e^{(k)}}
        \leq \norm[bigl]{(\mat M^{-1} \mat N)^{k}} \norm{\vect e^{(0)}}.
    \]
    By Gelfand's formula,
    proved in~\cref{proposition:matrices_gelfands} of \cref{cha:vectors_and_matrices},
    it holds that
    \begin{equation}
        \label{eq:gelfand}
        \lim_{k \to \infty} \norm[bigl]{(\mat M^{-1} \mat N)^{k}}^{\frac{1}{k}} = \rho(\mat M^{-1} \mat N).
    \end{equation}
    Therefore, we deduce that if $\rho(\mat M^{-1} \mat N) < 1$,
    then $\norm[bigl]{(\mat M^{-1} \mat N)^{k}} \to 0$ and so $\vect e^{(k)} \to \vect 0$.
    In addition, it follows from~\eqref{eq:gelfand} that for all $\varepsilon > 0$ there is $K \in \nat$ such that
    \[
        \forall k \geq K, \qquad
        \norm[bigl]{(\mat M^{-1} \mat N)^{k}}^{\frac{1}{k}} \leq \rho(\mat M^{-1} \mat N) + \varepsilon.
    \]
    Rearranging this inequality gives~\eqref{eq:additional_claim}.
\end{proof}

At this point,
it is natural to wonder whether there exist sufficient conditions on the matrix~$\mat A$ such that the inequality $\rho(\mat M^{-1} \mat N) < 1$ is satisfied,
which is best achieved on a case by case basis.
In the next sections,
we present four instances of splitting methods.
For each of them,
we obtain a sufficient condition for convergence.
We are particularly interested in the case where the matrix~$\mat A$ is Hermitian and positive definite,
which often arises in applications,
and in the case where $\mat A$ is strictly row or column diagonally dominant.
We recall that a matrix $\mat A$ is said to be row or column diagonally dominant
if, respectively,
\[
    \lvert a_{ii} \rvert \geq \sum_{j \neq i} \lvert a_{ij} \rvert \quad \forall i
    \qquad \text{ or } \qquad
    \lvert a_{jj} \rvert \geq \sum_{i \neq j} \lvert a_{ij} \rvert \quad \forall j.
\]

\subsubsection{Richardson's method}%
\label{ssub:richardson_s_method}

Arguably the simplest splitting of the matrix $\mat A$ is given by $\mat A = \frac{1}{\omega} \mat I - \bigl(\frac{1}{\omega} \mat I - \mat A\bigr)$,
for some parameter~$\omega \in \real$,
which leads to \emph{Richardson's method}:
\begin{equation}
    \label{eq:linear_richardson}
    \vect x^{(k+1)} = \vect x^{(k)} +  \omega (\vect b - \mat A \vect x^{(k)}).
\end{equation}
In this case the spectral radius which enters in the asymptotic rate of convergence is given by
\[
    \rho(\mat M^{-1} \mat N)
    = \rho\left(\omega \left(\frac{1}{\omega} \mat I - \mat A\right) \right)
    = \rho\bigl( \mat I - \omega \mat A \bigr)
\]
The eigenvalues of the matrix $\mat I - \omega \mat A$ are given by $1 - \omega \lambda_i$,
where $(\lambda_i)_{1 \leq i \leq L}$ are the eigenvalues of~$\mat A$.
Therefore, the spectral radius is given by
\[
    \rho(\mat M^{-1} \mat N)
    = \max_{1 \leq i \leq L} \abs{1  - \omega \lambda_i}.
\]
If the eigenvalues of the matrix~$\mat A$ do not either (i) all have a positive real part or (ii) all have a negative real part,
then
\[
    \forall \omega \in \real,
    \qquad \max_{1 \leq i \leq L} \abs{1  - \omega \lambda_i} \geq 1.
\]
In other words,
by~\cref{proposition:linear_convergence},
there is for any choice of~$\omega \in \real$ some $\vect x^{(0)}$ such that Richardson's method is non-convergent.
Therefore, in order for convergence to hold for all~$\vect x^{(0)}$,
it is necessary that the eigenvalues of $\mat A$ either all have a negative real part,
or all have a positive real part.
We focus in the next paragraph on the latter case and we also assume,
for simplicity, that $\mat A$ is Hermitian.

\paragraph{Case of symmetric positive definite $\mat A$.}%
\label{par:case_of_symmetric_positive_definite_mat_a_}
If the matrix $\mat A$ is Hermitian and positive definite,
the eigenvalues of $\mat A$ are all real and positive,
and it is possible to explicitly calculate the optimal value of $\omega$ for convergence.
In order for convergence to be as fast as possible,
the spectral radius of $\mat M^{-1} \mat N$ should be as small as possible,
in view of~\cref{proposition:linear_convergence}.
Denoting by $\lambda_{\min}$ and $\lambda_{\max}$ the minimum and maximum eigenvalues of $\mat A$,
it is not difficult to show that
\begin{equation}
    \label{eq:spectral_richardson}
    \rho(\mat M^{-1} \mat N)
    = \max_{1 \leq i \leq L} \abs{1  - \omega \lambda_i}
    = \max \bigl\{ \abs{1 - \omega \lambda_{\min}}, \abs{1 - \omega \lambda_{\max}}\bigr\}.
\end{equation}
The right-hand side is minimized $1 - \omega \lambda_{\min} = \omega \lambda_{\max} -1$,
in which case the two arguments of the maximum in~\eqref{eq:spectral_richardson} are equal.
From this we deduce the optimal value of $\omega$ and the associated spectral radius:
\[
    \omega_{\rm opt} = \frac{2}{\lambda_{\max} + \lambda_{\min}},
    \qquad
    \rho_{\rm opt}
    = 1 - \frac{2\lambda_{\min}}{\lambda_{\max} + \lambda_{\min}}
    = \frac{\lambda_{\max} - \lambda_{\min}}{\lambda_{\max} + \lambda_{\min}}
    =  \frac{\kappa_2(\mat A) - 1}{\kappa_2(\mat A) + 1}.
\]
We observe that the smaller the condition number of the matrix $\mat A$,
the better the asymptotic rate of convergence.

\begin{remark}
    [Link to optimization]
    \label{remark:linear_link_optimization}
    In the case where $\mat A \in \real^{n\times n}$ is symmetric and positive definite,
    the Richardson update~\eqref{eq:linear_richardson} may be viewed as a step of the steepest descent algorithm,
    which we study carefully in~\cref{cha:Optimization},
    for the function $f(\vect x) = \frac{1}{2} \vect x^\t \mat A \vect x - \vect b^\t \vect x$:
    \begin{equation}
        \label{eq:linear_richardson_gradient}
        \vect x^{(k+1)} = \vect x^{(k)} - \omega \nabla f(\vect x^{(k)}).
    \end{equation}
    The gradient of this function is $\nabla f(\vect x) = \mat A \vect x - \vect b$,
    and its Hessian matrix is $\mat A$.
    Since the Hessian matrix is positive definite, the function is convex
    and attains its global minimum when $\nabla f$ is zero,
    i.e.\ when $\mat A x = \vect b$.
\end{remark}

\subsubsection{Jacobi's method}%
\label{ssub:jacobi_s_method}
In Jacobi's method, the matrix $\mat M$ in the splitting is the diagonal matrix $\mat D$ with the same entries as those of $\mat A$ on the diagonal.
We denote by $\mat L$ and $\mat U$ the lower and upper triangular parts of~$\mat A$,
without the diagonal.
One step of the method reads
\begin{equation}
    \label{eq:linear_jacobi}
    \mat D \vect x^{(k+1)}
    = (\mat D - \mat A) \vect x^{(k)} + \vect b
    = - (\mat L + \mat U) \vect x^{(k)} + \vect b
\end{equation}
Since the matrix $\mat D$ on the left-hand side is diagonal,
this linear system with unknown $\vect x^{(k+1)}$ is very simple to solve.
The equation~\eqref{eq:linear_jacobi} can be rewritten as
\begin{equation*}
    \left\{
       \begin{aligned}
        & a_{11} x^{(\textcolor{darkblue}{k+1})}_1 + a_{12} x^{(k)}_2 + \dotsb + a_{1n} x^{(k)}_n = b_1 \\
        & a_{21} x^{(k)}_1 + a_{22} x^{(\textcolor{darkblue}{k+1})}_2 + \dotsb + a_{2n} x^{(k)}_n = b_2 \\
        & \vdots \\
        & a_{n1} x^{(k)}_1 + a_{n2} x^{(k)}_2 + \dotsb + a_{nn} x^{(\textcolor{darkblue}{k+1})}_n = b_n.
       \end{aligned}
   \right .
\end{equation*}
The updates for each of the entries of $\vect x^{(k+1)}$ are independent,
and so the Jacobi method lends itself well to parallel implementation.
The computational cost of one iteration,
measured in number of floating point operations required,
scales as $\bigo(n^2)$ if~$\mat A$ is a full matrix,
or $\bigo(nk)$ if~$\mat A$ is a sparse matrix with $k$ nonzero elements per row on average.
It is simple to prove the convergence of Jacobi's method is the case where $\mat A$ is diagonally dominant.
\begin{proposition}
    \label{proposition:linear_convergence_jacobi}
    Assume that $\mat A$ is strictly (row or column) diagonally dominant.
    Then it holds that $\rho(\mat M^{-1} \mat N) < 1$ for the Jacobi splitting.
\end{proposition}
\begin{proof}
    Assume that $\lambda$ is an eigenvalue of $\mat M^{-1} \mat N$
    and $\vect v$ is the associated unit eigenvector.
    Then
    \[
        \mat M^{-1} \mat N \vect v = \lambda \vect v
        \quad \Leftrightarrow \quad
        \mat N \vect v = \lambda \mat M \vect v
        \quad \Leftrightarrow \quad
        (\mat N - \lambda \mat M) \vect v = 0.
    \]
    In the case of Jacobi's splitting,
    this is equivalent to
    \[
        - (\mat L + \lambda \mat D + \mat U) \vect v = 0.
    \]
    If $\abs{\lambda} > 1$,
    then the matrix on the left-hand side of this equation is diagonally dominant and thus invertible
    (see \cref{exercise:invertibility_diagonal_dominant}).
    Therefore $\vect v = 0$, but this is a contradiction because $\vect v$ is vector of unit norm.
    Consequently, all the eigenvalues are bounded from above strictly by 1 in modulus.
\end{proof}


\subsubsection{Gauss--Seidel's method}%
\label{ssub:gauss_seidel_s_method}

In Gauss Seidel's method, the matrix $\mat M$ in the splitting is the lower triangular part of $\mat A$,
including the diagonal.
One step of the method then reads
\begin{equation}
    \label{eq:linear_gauss_seidel}
    (\mat L + \mat D) \vect x^{(k+1)} = - \mat U \vect x^{(k)} + \vect b
\end{equation}
The system is solved by forward substitution.
The equation~\eqref{eq:linear_gauss_seidel} can be rewritten equivalently as
\begin{equation*}
    \left\{
       \begin{aligned}
        & a_{11} x^{(\textcolor{darkblue}{k+1})}_1 + a_{12} x^{(k)}_2 + a_{13} x^{(k)}_3 + \dotsb + a_{1n} x^{(k)}_n = b_1 \\
        & a_{21} x^{(\textcolor{darkblue}{k+1})}_1 + a_{22} x^{(\textcolor{darkblue}{k+1})}_2 + a_{23} x^{(k)}_3 + \dotsb + a_{2n} x^{(k)}_n = b_2 \\
        & a_{32} x^{(\textcolor{darkblue}{k+1})}_1 + a_{32} x^{(\textcolor{darkblue}{k+1})}_2 + a_{33} x^{(\textcolor{darkblue}{k+1})}_3 + \dotsb + a_{3n} x^{(k)}_n = b_3 \\
        & \vdots \\
        & a_{n1} x^{(\textcolor{darkblue}{k+1})}_1 + a_{n2} x^{(\textcolor{darkblue}{k+1})}_2 + a_{n3} x^{(\textcolor{darkblue}{k+1})}_3 + \dotsb + a_{nn} x^{(\textcolor{darkblue}{k+1})}_n = b_n.
       \end{aligned}
   \right .
\end{equation*}
Given $\vect x^{(k)}$,
the first entry of $\vect x^{(k+1)}$ is obtained from the first equation.
Then the value of the second entry is obtained from the second equation, etc.
Unlike Jacobi's method,
the Gauss--Seidel method is sequential and the entries of $\vect x^{(k+1)}$ cannot be updated in parallel.

It is possible to prove the convergence of the Gauss--Seidel method in particular cases.
For example, the method converges if $\mat A$ is strictly diagonally dominant.
Proving this,
using an approach similar to that in the proof of~\cref{proposition:linear_convergence_jacobi},
is the goal of~\cref{exercise:linear_convergence_gauss_seidel}.
It is also possible to prove convergence when $\mat A$ is Hermitian and positive definite.
We show this in the next section for the relaxation method,
which generalizes the Gauss--Seidel method.

\subsubsection{Relaxation method}%

The relaxation method generalizes the Gauss--Seidel method.
It corresponds to the splitting
\begin{equation}
    \mat A = \left( \frac{\mat D}{\omega} + \mat L \right) - \left(\frac{1 - \omega}{\omega} \mat D - \mat U \right).
\end{equation}
When $\omega = 1$,
this is simply the Gauss--Seidel splitting.
The idea of the relaxation method is that,
by letting $\omega$ be a parameter that can differ from 1,
faster convergence can be achieved.
This intuition will be verified later.
The equation~\eqref{eq:linear_system_iterative_method} for this splitting can be rewritten equivalently as
\begin{equation*}
    \left\{
       \begin{aligned}
        & a_{11} \bigl(x^{(\textcolor{darkblue}{k+1})}_1 - x^{(k)}_1 \bigr) =  -\omega \left(a_{11} x^{(k)}_1 + a_{12} x^{(k)}_2 + \dotsb + a_{1n} x^{(k)}_n - b_1 \right) \\
        & a_{22} \bigl(x^{(\textcolor{darkblue}{k+1})}_2 - x^{(k)}_2 \bigr) =  -\omega \left(a_{21} x^{(\textcolor{darkblue}{k+1})}_1 + a_{22} x^{(k)}_2 + \dotsb + a_{2n} x^{(k)}_n - b_2 \right) \\
        & \vdots \\
        & a_{nn} \bigl(x^{(\textcolor{darkblue}{k+1})}_n - x^{(k)}_n \bigr) =  -\omega \left(a_{n1} x^{(\textcolor{darkblue}{k+1})}_1 + a_{n2} x^{(\textcolor{darkblue}{k+1})}_2 + \dotsb + a_{nn} x^{(k)}_n - b_n \right).
       \end{aligned}
   \right .
\end{equation*}
The coefficient on the right-hand side is larger than in the Gauss--Seidel method if $\omega > 1$,
and smaller if $\omega < 1$.
These regimes are called \emph{over-relaxation} and \emph{under-relaxation}, respectively.

To conclude this section,
we establish a sufficient condition for the convergence of the relaxation method,
and also of the Gauss--Seidel method as a particular case when $\omega = 1$,
when the matrix $\mat A$ is Hermitian and positive definite.
To this end, we begin by showing the following preparatory result,
which concerns a general splitting $\mat A = \mat M - \mat N$.
\begin{proposition}
    \label{proposition:criterion_convergence}
    Let $\mat A$ be Hermitian and positive definite.
    If the Hermitian matrix $\mat M^* + \mat N$ is positive definite,
    then $\rho(\mat M^{-1} \mat N) < 1$.
\end{proposition}
\begin{proof}
    First, notice that $\mat M^* + \mat N$ is indeed Hermitian because
    \[
        (\mat M^* + \mat N)^* = \mat M + \mat N^* = \mat A + \mat N + \mat N^*.
    \]
    We will show that $\norm{\mat M^{-1} \mat N}_{\mat A} < 1$,
    where $\norm{\placeholder}_{\mat A}$ is the matrix norm induced by the following norm on vectors:
    \[
        \norm{\vect x}_{\mat A} := \sqrt{\vect x^* \mat A \vect x}.
    \]
    Showing that this indeed defines a vector norm is the goal of~\cref{exercise:linear_norm_induced_A}.
    Since $\mat N = \mat M - \mat A$, it holds that
    \(
        \norm{\mat M^{-1} \mat N}_{\mat A}
        = \norm{\mat I - \mat M^{-1} \mat A}_{\mat A},
    \)
    and so
    \[
        \norm{\mat M^{-1} \mat N}_{\mat A}
        = \sup \bigl\{ \norm{\vect x - \mat M^{-1} \mat A \vect x}_{\mat A}: \norm{\vect x}_{\mat A} \leq 1 \bigr\}.
    \]
    Take $\vect x \in \complex^n$ with $\norm{\vect x}_{\mat A} \leq 1$
    and let $\vect y = \mat M^{-1} \mat A \vect x$.
    We calculate
    \begin{align*}
        \norm{\vect x - \mat M^{-1} \mat A \vect x}_{\mat A}^2
        &= \vect x^* \mat A \vect x - \vect y^* \mat A \vect x - \vect x^* \mat A \vect y + \vect y^* \mat A \vect y \\
        &= \vect x^* \mat A \vect x - \vect y^* \mat M \mat M^{-1} \mat A \vect x - (\mat M^{-1} \mat A \vect x)^* \mat M^* \vect y + \vect y^* \mat A \vect y \\
        &= \vect x^* \mat A \vect x - \vect y^* \mat M \vect y - \vect y^* \mat M^* \vect y + \vect y^* (\mat M - \mat N) \vect y \\
        &= \vect x^* \mat A \vect x - \vect y^* (\mat M^* + \mat N) \vect y \leq 1 - \vect y^* (\mat M^* + \mat N) \vect y < 1,
        % &= \vect x^\t \mat A \vect x - 2 \vect y^\t \mat M \vect \vect y + \vect y^\t \mat A \vect y \\
    \end{align*}
    where we used in the last inequality the assumption that $\mat M^* + \mat N$ is positive definite.
    This inequality holds true for all $\vect x \in \complex^n$ with $\norm{\vect x}_{\mat A} = 1$,
    and so we conclude that \( \norm{\mat M^{-1} \mat N}_{\mat  A} < 1\),
    which implies that $\rho(\mat M^{-1} \mat N) < 1$.
\end{proof}

As a corollary,
we obtain a sufficient condition for the convergence of the relaxation method.

\begin{corollary}
    \label{corollary:convergence_relaxation}
    Assume that $\mat A$ is Hermitian and positive definite.
    Then the relaxation method converges if $\omega \in (0, 2)$.
\end{corollary}
\begin{proof}
    For the relaxation method,
    we have
    \[
        \mat M + \mat N^*
        = \left( \frac{\mat D}{\omega} + \mat L \right) + \left(\frac{1 - \omega}{\omega} \mat D - \mat U \right)^*.
    \]
    Since $\mat A$ is Hermitian,
    it holds that $\mat D^* = \mat D$ and $\mat U^* = \mat L$.
    Therefore,
    \[
        \mat M + \mat N^*
        = \frac{2 - \omega}{\omega} \mat D.
    \]
    The diagonal elements of $\mat D$ are all positive,
    because $\mat A$ is positive definite.
    (Indeed, if there was an index $i$ such that $d_{ii} \leq 0$,
    then it would hold that $\vect e_i^\t \mat A \vect e_i = d_{ii} \leq 0$,
    contradicting the assumption that $\mat A$ is positive definite.)
    We deduce that $\mat M + \mat N^*$ is positive definite if and only if $\omega \in (0, 2)$.
    We can then conclude the proof by using \cref{proposition:criterion_convergence}.
\end{proof}
Note that \cref{corollary:convergence_relaxation} implies as a particular case
the convergence of the Gauss--Seidel method when $\mat A$ is Hermitian and positive definite.
The condition $\omega \in (0, 2)$ is in fact necessary for the convergence of the relaxation method,
not only in the case of a Hermitian positive definite matrix $\mat A$ but in general.

\begin{proposition}
    [Necessary condition for the convergence of the relaxation method]
    \label{proposition:necessary_sor}
    Let $\mat A \in \complex^{n \times n}$ be an invertible matrix,
    and let $\mat A = \mat M_{\omega} - \mat N_{\omega}$ denote the splitting of the relaxation method with parameter~$\omega$.
    It holds that
    \[
        \forall \omega \neq 0, \qquad
        \rho(\mat M_{\omega}^{-1} \mat N_{\omega})
        \geq \abs{\omega - 1}.
    \]
\end{proposition}
\begin{proof}
    We recall the following facts:
    \begin{itemize}
        \item the determinant of a product of matrices is equal to the product of the determinants.
        \item the determinant of a triangular matrix is equal to the product of its diagonal entries;
        \item the determinant of a matrix is equal to the product of its eigenvalues,
            to the power of their algebraic multiplicity.
            This can be shown from the previous two items,
            by passing to the Jordan normal form.
    \end{itemize}
    Therefore,
    we have that
    \[
        \det (\mat M_{\omega}^{-1} \mat N_{\omega}) = \det (\mat M_{\omega})^{-1}  \det (\mat N_{\omega})
        = \frac {\det \left(\frac{1 - \omega}{\omega} \mat D - \mat U \right)}{\det \left( \frac{\mat D}{\omega} + \mat L \right)}
        = (1 - \omega)^n.
    \]
    Since the determinant on the left-hand side is the product of the eigenvalues of $\mat M_{\omega}^{-1} \mat N_{\omega}$,
    it is bounded from above in modulus by $\rho(\mat M_{\omega}^{-1} \mat N_{\omega})^n$,
    and so we deduce $\rho(\mat M_{\omega}^{-1} \mat N_{\omega})^n \geq \abs{1 - \omega}^n$.
    The statement then follows by taking the $n$-th root.
\end{proof}

\subsubsection{Comparison between Jacobi and Gauss--Seidel for tridiagonal matrices~\moreinfo}%
\label{ssub:comparison_between_jacobi_and_gauss_seidel_for_tridiagonal_matrices}
For tridiagonal matrices,
the convergence rate of the Jacobi and Gauss--Seidel methods satisfy an explicit relation,
which we prove in this section.
We denote the Jacobi and Gauss--Seidel splittings by $\mat M_{\mathcal J} - \mat N_{\mathcal J}$ and  $\mat M_{\mathcal G} - \mat N_{\mathcal G}$,
respectively,
and use the following notation for the entries of the matrix $\mat A$:
\[
    \begin{pmatrix}
        a_1 & b_1 \\
        c_1 & \ddots & \ddots  \\
             & \ddots & \ddots & b_{n-1} \\
             & & c_{n-1} & a_n
    \end{pmatrix}.
\]
Before presenting and proving the result,
notice that for any $\mu \neq 0$ it holds that
\begin{equation}
    \label{eq:preliminary_equation}
    \begin{pmatrix}
        \mu \\
        & \mu^{2} &  \\
        & & \ddots & \\
        & & & \mu^{n}
    \end{pmatrix}
    \mat A
    \begin{pmatrix}
        \mu^{-1}  \\
        & \mu^{-2} &  \\
        & & \ddots & \\
        & & & \mu^{-n}
    \end{pmatrix}
    =
    \begin{pmatrix}
        a_1 & \mu^{-1} b_1 \\
        \mu c_1 & \ddots & \ddots  \\
             & \ddots & \ddots & \mu^{-1} b_{n-1} \\
             & & \mu c_{n-1} & a_n
    \end{pmatrix}.
\end{equation}
\begin{proposition}
    \label{proposition:tridiagonal}
    Assume that $\mat A$ is tridiagonal with nonzero diagonal elements,
    so that both $\mat M_{\mathcal J} = \mat D$ and $\mat \mat M_{\mathcal G} = \mat L + \mat D$ are invertible.
    Then
    \[
        \rho(\mat M_{\mathcal G}^{-1} \mat N_{\mathcal  G})
        = \rho(\mat M_{\mathcal J}^{-1} \mat N_{\mathcal  J})^2
    \]
\end{proposition}
\begin{proof}
    If $\lambda$ is an eigenvalue of $\mat M_{\mathcal G}^{-1} \mat N_{\mathcal G}$ with associated unit eigenvector $\vect v$,
    then
    \[
        \mat M_{\mathcal G}^{-1} \mat N_{\mathcal G} \vect v = \lambda \vect v
        \quad \Leftrightarrow \quad
        \mat N_{\mathcal G} \vect v = \lambda \mat M_{\mathcal G} \vect v
        \quad \Leftrightarrow \quad
        (\mat N_{\mathcal G} - \lambda \mat M_{\mathcal G}) \vect v = 0.
    \]
    For fixed $\lambda$,
    there exists a nontrivial solution $\vect v$ to the last equation if and only if
    \[
        p_{\mathcal G}(\lambda) := \det (\mat N_{\mathcal G} - \lambda \mat M_{\mathcal G})
        = - \det (\lambda \mat L + \lambda \mat D + \mat U) = 0.
    \]
    Likewise, $\lambda$ is an eigenvalue of $\mat M_{\mathcal J}^{-1} \mat N_{\mathcal J}$ if and only if
    \[
        p_{\mathcal J}(\lambda) := \det (\mat N_{\mathcal J} - \lambda \mat M_{\mathcal J})
        = - \det (\mat L + \lambda \mat D + \mat U) = 0.
    \]
    Now notice that
    \[
        p_{\mathcal G}(\lambda^2)
        = - \det \bigl(\lambda^2 \mat L + \lambda^2 \mat D + \mat U\bigr)
        = - \lambda^n \det \bigl(\lambda \mat L + (\lambda \mat D) + \lambda^{-1} \mat U\bigr).
    \]
    Applying~\eqref{eq:preliminary_equation} with $\mu = \lambda \neq 0$,
    we deduce
    \[
        p_{\mathcal G}(\lambda^2)
        = - \lambda^n \det \bigl(\mat L + \lambda \mat D + \mat U\bigr)
        = \lambda^n p_{\mathcal J}(\lambda)
    \]
    It is clear that this relation is true also if $\lambda = 0$.
    Consequently, it holds that if $\lambda$ is an eigenvalue of the matrix $\mat M_{\mathcal J}^{-1} \mat N_{\mathcal J}$ then
    $\lambda^2$ is an eigenvalue of $\mat M_{\mathcal G}^{-1} \mat N_{\mathcal G}$.
    Conversely, if $\lambda$ is an eigenvalue of~$\mat M_{\mathcal G}^{-1} \mat N_{\mathcal G}$,
    then the two square roots of $\lambda$ are eigenvalues of $\mat M_{\mathcal J}^{-1} \mat N_{\mathcal J}$.
\end{proof}

If a matrix $\mat A$ is tridiagonal and Toeplitz,
i.e.\ if it is of the form
\[
    \begin{pmatrix}
        a & b \\
        c & \ddots & \ddots  \\
             & \ddots & \ddots & b \\
             & & c & a
    \end{pmatrix},
\]
then it is possible to prove that the eigenvalues of~$\mat A$ are given by
\begin{equation}
    \label{eq:linear_eigenvalues_toeplitz}
    \lambda_k =
    a + 2\sqrt{bc} \cos \left(\frac{k \pi}{n+1} \right),
    \qquad k = 1, \dots, n.
\end{equation}
In this case, the spectral radius of $\mat M_{\mathcal J}^{-1} \mat N_{\mathcal J}$ can be determined explicitly.

\subsubsection{Monitoring the convergence}
\label{ssub:monitoring_the_convergence}
In practice,
we have access to the residual $\vect r^{(k)} = \mat A \vect x^{(k)} - \vect b$ at each iteration,
but not to the error $\vect e^{(k)} = \vect x^{(k)} - \vect x_*$,
as calculating the latter would require to know the exact solution of the problem.
Nevertheless,
the two are related by the equation
\[
    \vect r^{(k)} = \mat A \vect e^{(k)}
    \quad \Leftrightarrow \quad \vect e^{(k)} = \mat A^{-1} \vect r^{(k)}.
\]
Therefore, it holds that $\norm{\vect e^{(k)}} \leq \norm{\mat A^{-1}} \norm{\vect r^{(k)}}$.
Likewise, the relative error satisfies
\[
     \frac{\norm{\vect e^{(k)}}}{\norm{\vect x_*}}
     = \frac{\norm{\mat A^{-1} \vect r^{(k)}}}{\norm{\mat A^{-1} \vect b}},
\]
and since $\norm{\vect b} = \norm{\mat A \mat A^{-1} \vect b} \leq \norm{\mat A} \norm{\mat A^{-1} \vect b}$,
we deduce
\[
    \frac{\norm{\vect e^{(k)}}}{\norm{\vect x_*}}
     \leq \kappa(\mat A) \frac{\norm{\vect r^{(k)}}}{\norm{\vect b}}.
\]
The fraction on the right-hand side is the \emph{relative residual}.
If the system is well conditioned,
that is if $\kappa(A)$ is close to one,
then controlling the relative residual enables a good control of the relative error.

\subsubsection{Stopping criterion}%
\label{ssub:stopping_criterion}

In practice,
several criteria can be employed in order to decide when to stop iterating.
Given a small number $\varepsilon$ (unrelated to the machine epsilon in \cref{cha:rounding_errors}),
the following alternatives are available:
\begin{itemize}
    \item
        Stop when $\norm{\vect r^{(k)}} \leq \varepsilon$.
        The downside of this approach is that
        it is not \emph{scaling invariant}:
        when used for solving the following rescaled system
        \[
            k \mat A \vect x = k \vect b, \qquad k \neq 1,
        \]
        a splitting method with rescaled initial guess $k \vect x^{(0)}$
        will require a number of iterations that depends on $k$:
        fewer if $k \ll 1$ and more if $k \gg 1$.
        In practice, controlling the relative residual and the relative error is often preferable.

    \item
        Stop when $\norm{\vect r^{(k)}} / \norm{\vect r^{(0)}} \leq \varepsilon$.
        This criterion is scaling invariant,
        but the number of iterations is dependent on the quality of the initial guess $\vect x^{(0)}$.

    \item
        Stop when $\norm{\vect r^{(k)}} / \norm{\vect b}$.
        This criterion is generally the best,
        because it is both scaling invariant and the quality of the final iterate is independent of that of the initial guess.
\end{itemize}

\subsection{The conjugate gradient method}%
\label{sub:the_conjugate_gradient_method}

As already mentioned in~\cref{remark:linear_link_optimization},
when the matrix $\mat A \in \complex^{n \times n}$ in the linear system $\mat A \vect x = \vect b$ is symmetric and positive definite,
the system can be interpreted as a minimization problem for the function
\begin{equation}
    \label{eq:function_optimization}
    f(\vect x) = \frac{1}{2} \vect x^\t \mat A \vect x - \vect b^\t \vect x.
\end{equation}
The fact that
the exact solution $\vect x_*$ to the linear system is the unique minimizer of this function appears clearly
when rewriting $f$ as follows:
\begin{equation}
    \label{eq:rewritten_expression_of_f}
    f(\vect x)
    = \frac{1}{2} (\vect x - \vect x_*)^\t \mat A (\vect x - \vect x_*)  - \frac{1}{2} \vect x_*^\t \mat A \vect x_{*}.
\end{equation}
The second term is constant with $\vect x$,
and the first term is strictly positive if $\vect x - \vect x_* \neq \vect 0$,
because~$\mat A$ is positive definite.
We saw that Richardson's method can be interpreted as a steepest descent with fixed step size,
\[
    \vect x^{(k+1)} = \vect x^{(k)} - \omega \nabla f(\vect x^{(k)}).
\]
In this section,
we will present and study other methods
for solving the linear system~\eqref{eq:linear_system}
which can be viewed as optimization methods.
Since $\mat A$ is symmetric,
it is diagonalizable and the function $f$ can be rewritten as
\begin{align*}
    f(\vect x)
    &= \frac{1}{2} (\vect x - \vect x_*)^\t \mat Q \mat D \mat Q^\t (\vect x - \vect x_*)  - \frac{1}{2} \vect x_*^\t \mat A \vect x_{*} \\
    &= \frac{1}{2} (\mat Q^\t \vect e)^\t \mat D (\mat Q^\t \vect e)  - \frac{1}{2} \vect x_*^\t \mat A \vect x_{*}, \qquad  \vect e = \vect x - \vect x_*.
\end{align*}
Therefore, we have that
\begin{align*}
    f(\vect x)= \frac{1}{2} \sum_{i=1}^{n} \lambda_i \eta_i^2 - \frac{1}{2} \vect x_*^\t \mat A \vect x_{*}, \qquad  \vect \eta = \vect Q^\t (\vect x - \vect x_*),
\end{align*}
where $(\lambda_i)_{1 \leq i \leq n}$ are the diagonal entries of $\mat D$.
This shows that $f$ is a paraboloid
after a change of coordinates.

\subsubsection{Steepest descent method}%
\label{ssub:steepest_descent_method}

The steepest descent method is more general than Richardson's method in the sense that
the step size changes from iteration to iteration
and the method is not restricted to quadratic functions of the form~\eqref{eq:function_optimization}.
Each iteration is of the form
\[
    \vect x^{(k+1)} = \vect x^{(k)} - \omega_k \nabla f(\vect x^{(k)}).
\]
It is natural to wonder whether the step size $\omega_k$ can be fixed in such a way that
$f(\vect x^{(k+1)})$ is as small as possible.
For the case of the quadratic function~\eqref{eq:function_optimization},
this value of $\omega_k$ can be calculated explicitly for a general search direction $\vect d$,
and in particular also when $\vect d = \nabla f(\vect x^{(k)})$.
We calculate that
\begin{align}
    \notag
    f(\vect x^{(k+1)})
    &= f\bigl(\vect x^{(k)} - \omega_k \vect d\bigr) = \frac{1}{2} \left( \vect x^{(k)} - \omega_k \vect d\right)^\t \mat A \left(\vect x^{(k)} - \omega_k \vect d\right) - \left(\vect x^{(k)} - \omega_k \vect d\right)^\t \vect b \\
    \label{eq:linear_steepest}
    &= f\bigl( \vect x^{(k)} \bigr) + \frac{\omega_k^2}{2} \vect d^\t \mat A \vect d - \omega_k \vect d^\t \vect r^{(k)}.
\end{align}
When viewed as a function of the real parameter $\omega_k$,
the right-hand side is a convex quadratic function.
It is minimized when its derivative is equal to zero,
i.e.\ when
\begin{equation}
    \label{eq:linear_optimal_step_size_steepest_descent}
    \omega_k \vect d^\t \mat A \vect d - \vect d^\t (\mat A \vect x_k - \vect b) = 0
    \qquad \Rightarrow  \qquad \omega_k = \frac{\vect d^\t \vect r^{(k)}}{\vect d^\t \mat A \vect d}.
\end{equation}
The steepest descent algorithm with step size obtained from this equation is summarized in~\cref{algo:steepest_descent_method} below.
By construction,
the function value $f(\vect x^{(k)})$ is nonincreasing with $k$,
which is equivalent to saying that
the error $\vect x - \vect x_*$ is nonincreasing in the norm $\vect x \mapsto \sqrt{\vect x^\t \mat A \vect x}$.
In order to quantify more precisely the decrease of the error in this norm,
we introduce the notation
\[
    E_k = \norm{\vect x - \vect x_*}^2_{\mat A} := (\vect x^{(k)} - \vect x_*)^\t \mat A (\vect x^{(k)} - \vect x_*)
    = (\mat A \vect x^{(k)} - \vect b)^\t \mat A^{-1} (\mat A \vect x^{(k)} - \vect b).
\]
We begin by showing the following auxiliary lemma.
\begin{lemma}
    [Kantorovich inequality]
    Let $\mat A \in \real^{n \times n}$ be a symmetric and positive definite matrix,
    and let $\lambda_0 \leq \dots \leq \lambda_n$ denote its eigenvalues.
    Then for all nonzero $\vect z \in \real^n$ it holds that
    \[
        \frac{(\vect z^\t \vect z)^2}{(\vect z^\t \mat A \vect z) (\vect z^\t \mat A^{-1} \vect z)}
        \geq \frac{4\lambda_1 \lambda_n}{(\lambda_1 + \lambda_n)^2}.
    \]
\end{lemma}
\begin{proof}
    By the AM-GM (arithmetic mean-geometric mean) inequality,
    it holds for all $t > 0$ that
    \begin{align*}
        \sqrt{(\vect z^\t \mat A \vect z) (\vect z^\t \mat A^{-1} \vect z)}
        &= \sqrt{(t\vect z^\t \mat A \vect z) (t^{-1} \vect z^\t \mat A^{-1} \vect z)}
        \leq \frac{1}{2} \left( t \vect z^\t \mat A \mat z + \frac{1}{t} \vect z^\t \mat A^{-1} \mat z \right) \\
        &= \frac{1}{2} \vect z^\t \left( t \mat A + \frac{1}{t} \mat A^{-1} \right) \vect z.
    \end{align*}
    The matrix on the right-hand side is also symmetric and positive definite,
    with eigenvalues equal to $t \lambda_i + (t \lambda_i)^{-1}$.
    Therefore, we deduce
    \begin{equation}
        \label{eq:linear_kantorovich_bound}
        \forall t \geq 0, \qquad
        \sqrt{(\vect z^\t \mat A \vect z) (\vect z^\t \mat A^{-1} \vect z)}
        \leq \frac{1}{2} \left( \max_{i \in \{1, \dotsc, n\}} t \lambda_i + (t \lambda_i)^{-1} \right) \vect z^\t \vect z.
    \end{equation}
    The function $x \mapsto x + x^{-1}$ is convex,
    and so over any closed interval $[x_{\min}, x_{\max}]$
    it attains its maximum either at $x_{\min}$ or at $x_{\max}$.
    Consequently, it holds that
    \begin{equation*}
        % \label{eq:maximum_between_two_things}
        \left( \max_{i \in \{1, \dotsc, n\}} t \lambda_i + (t \lambda_i)^{-1} \right)
        = \max \left\{ t \lambda_{1} + \frac{1}{t \lambda_{1}}, t \lambda_{n} + \frac{1}{t \lambda_{n}} \right\}.
    \end{equation*}
    In order to obtain the best possible bound from~\eqref{eq:linear_kantorovich_bound},
    we should let $t$ be such that the maximum is minimized,
    which occurs when the two arguments of the maximum are equal:
    \[
            t \lambda_{1} + \frac{1}{t \lambda_{1}} = t \lambda_{n} + \frac{1}{t \lambda_{n}}
            \qquad \Rightarrow \qquad t = \frac{1}{\sqrt{\lambda_{1} \lambda_{n}}}.
    \]
    For this value of $t$,
    the maximum in~\eqref{eq:linear_kantorovich_bound} is equal to
    \[
        \sqrt{\frac{\lambda_1}{\lambda_{n}}} + \sqrt{\frac{\lambda_{n}}{\lambda_1}}.
    \]
    By substituting this expression in~\eqref{eq:linear_kantorovich_bound} and rearranging,
    we obtain the statement.
\end{proof}

We are now able to prove the convergence of the steepest descent method.
\begin{theorem}
    [Convergence of the steepest descent method]
    \label{theorem:linear_convergenec_steepest_descent}
    It holds that
    \[
        E_{k+1} \leq \left(\frac{\kappa_2(\mat A) - 1}{\kappa_2(\mat A) + 1}  \right)^2 E_k.
    \]
\end{theorem}
\begin{proof}
    Substituting $\vect x^{(k+1)} = \vect x^{(k)} - \omega_k \vect d$ in the expression for $E_{k+1}$,
    we obtain
    % we obtain
    % Substituting the formula for the step size~\eqref{eq:linear_optimal_step_size_steepest_descent},
    % we obtain
    \begin{align*}
        E_{k+1}
        &= (\vect x^{(k)} - \omega_k \vect d - \vect x_*)^\t \mat A (\vect x^{(k)} - \omega_k \vect d - \vect x_*) \\
        &= E_k - 2 \omega_k \vect d^\t \vect r^{(k)} + \omega_k^2 \vect d^\t \mat A \vect d \\
        &=  E_k - \frac{(\vect d^\t \vect d)^2}{\vect d^\t \mat A \vect d}
        =  \left( 1 - \frac{(\vect d^\t \vect d)^2}{(\vect d^\t \mat A \vect d) (\vect d^\t \mat A^{-1} \vect d)} \right) E_k,
    \end{align*}
    Using the Kantorovich inequality,
    we have
    \begin{align*}
        E_{k+1}
        \leq \left( 1  - \frac{4 \lambda_1 \lambda_n}{(\lambda_1 + \lambda_n)^2}  \right) E_k
        \leq \left(\frac{\lambda_1 - \lambda_n}{\lambda_1 + \lambda_n}  \right)^2 E_k
        = \left(\frac{\kappa_2(\mat A) - 1}{\kappa_2(\mat A) + 1}  \right)^2 E_k.
    \end{align*}
    We immediately deduce the statement from this inequality.
\end{proof}

\begin{algorithm}
\caption{Steepest descent method}%
\label{algo:steepest_descent_method}%
\begin{algorithmic}[1]
\State Pick $\varepsilon$ and initial $\vect x$%
\State $\vect r \gets \mat A \vect x - \vect b$%
\While{$\norm{\vect r} \geq \varepsilon \norm{\vect b}$}
    \State $\vect d \gets \vect r$
    \State $\omega \gets \vect d^\t \vect r/\vect d^\t \mat A \vect d$
    \State $\vect x \gets \vect x - \omega \vect d$
    \State $\vect r \gets \mat A \vect x - \vect b$
\EndWhile
\end{algorithmic}
\end{algorithm}

\subsubsection{Preconditioned steepest descent}%
\label{ssub:preconditioned_steepest_descent}
We observe from \cref{theorem:linear_convergenec_steepest_descent} that the convergence of the steepest descent method is faster when
the condition number of the matrix $\mat A$ is low.
This naturally leads to the following question:
can we reformulate the minimization of $f(\vect x)$ in~\eqref{eq:function_optimization}
as another optimization problem which is of the same form
but involves a matrix with a lower condition number,
thereby providing scope for faster convergence?
In order to answer this question,
we consider a linear change of coordinates~$\vect y = \mat T^{-1} \vect x$,
where $\mat T$ is an invertible matrix,
and we define
\begin{equation}
    \label{eq:linear_quadratic_function}
    \widetilde f(\vect y) = f(\mat T \vect y)
    = \frac{1}{2} \vect y^\t (\mat T^{\t} \mat A \mat T) \vect y - (\mat T^{\t} \vect b)^{\t} \vect y.
\end{equation}
This function is of the same form as $f$~in~\eqref{eq:function_optimization},
with the matrix $\widetilde {\mat A} := \mat T^\t \mat A \mat T$ instead of~$\mat A$ and the vector $\widetilde {\vect b} := \mat T^{\t} \vect b$ instead of $\vect b$.
Its minimizer is $\vect y_* = \mat T^{-1} \vect x_*$.
The steepest descent algorithm can be applied to~\eqref{eq:linear_quadratic_function} and,
from an approximation $\vect y^{(k)}$ of the minimizer~$\vect y_*$,
an approximation $\vect x^{(k)}$ of $\vect x_*$ is obtained by the change of variable $\vect x^{(k)} = \mat T \vect y^{(k)}$.
This approach is called \emph{preconditioning}.
By \cref{theorem:linear_convergenec_steepest_descent},
the steepest descent method
satisfies the following error estimate
when applied to the function~\eqref{eq:linear_quadratic_function}:
\begin{align*}
    E_{k+1}
    \leq
    \left(\frac{\kappa_2(\mat T^\t \mat A \mat T) - 1}{\kappa_2(\mat T^\t \mat A \mat T) + 1} \right)^2 E_k,
    \qquad E_{k} &= (\vect y^{(k)} - \vect y_*)^\t \widetilde {\mat A} (\vect y^{(k)} - \vect y_*), \\
    &= (\vect x^{(k)} - \vect x_*)^\t \mat A (\vect x^{(k)} - \vect x_*).
\end{align*}
Consequently, the convergence is faster than that of the usual steepest descent method if $\kappa_2(\mat T^\t \mat A \mat T) < \kappa_2(\mat A)$.
The optimal change of coordinates is given by $\mat T = \mat C^{-\t}$,
where $\mat C$ is the factor of the Cholesky factorization of $\mat A$ as $\mat C \mat C^\t$.
Indeed, in this case
\[
    \mat T^\t \mat A \mat T = \mat C^{-1} \mat C \mat C^\t \mat C^{-\t} = \mat I
    \qquad \Rightarrow \qquad
    \kappa_2(\mat T^\t \mat A \mat T) = 1,
\]
and the method converges in a single iteration!
However,
this iteration amounts to solving the linear system by direct Cholesky factorization of $\mat A$.
In practice, it is usual to define $\mat T$ from an approximation of the Cholesky factorization,
such as the \emph{incomplete Cholesky factorization}.

To conclude this section,
we demonstrate that the change of variable from $\vect x$ to $\vect y$ need not be performed explicitly in practice.
Indeed, one step of the steepest descent algorithm applied to function $\widetilde f$ reads
\[
    \vect y^{(k+1)} = \vect y^{(k)} - \widetilde \omega_k (\widetilde {\mat A} \vect y^{(k)} - \widetilde{\vect b}),
    \qquad \widetilde \omega_k = \frac{(\widetilde {\mat A} \vect y^{(k)} - \widetilde {\vect b})^\t(\widetilde {\mat A} \vect y^{(k)} - \widetilde {\vect b})}{(\widetilde {\mat A} \vect y^{(k)} - \widetilde {\vect b})^\t \widetilde{\mat A} (\widetilde {\mat A} \vect y^{(k)} - \widetilde {\vect b})}.
\]
Letting $\vect x^{(k)} = \mat T \vect y^{(k)}$,
this equation can be rewritten as the following iteration:
\[
    \vect x^{(k+1)} = \vect x^{(k)} - \widetilde \omega_k \vect d_k,
    \qquad \widetilde \omega_k = \frac{\vect d_k^\t \vect r^{(k)}}{\vect d_k^\t \mat A \vect d_k}, \qquad \vect d_k = \mat T^{\t} \mat T (\mat A \vect x^{(k)} - \vect b).
\]
A comparison with~\eqref{eq:linear_optimal_step_size_steepest_descent} shows that the step size $\widetilde \omega_k$ is such that $f(\vect x^{(k+1)})$ is minimized.
This reasoning shows that the preconditioned conjugate gradient method amounts to choosing the direction~$\vect d^{(k)} = \mat T \mat T^{\t} \vect r^{(k)}$ at each iteration,
instead of just $\vect r^{(k)}$,
as is apparent in \cref{algo:preconditioned_steepest_descent_method}.
It is simple to check that $- \vect d^{(k)}$ is a descent direction for $f$:
\[
    -\nabla f(\vect x)^\t \bigl(\mat T^\t \mat T (\mat A \vect x - \vect b)\bigr) =
    -\bigl(\mat T (\mat A \vect x - \vect b)\bigr)^\t \bigl(\mat T (\mat A \vect x - \vect b)\bigr) \leq 0.
\]
\begin{algorithm}[ht]
\caption{Preconditioned steepest descent method}%
\label{algo:preconditioned_steepest_descent_method}%
\begin{algorithmic}[1]
\State Pick $\varepsilon$, invertible $\mat T$ and initial $\vect x$%
\State $\vect r \gets \mat A \vect x - \vect b$%
\While{$\norm{\vect r} \geq \varepsilon \norm{\vect b}$}
    \State $\vect d \gets \mat T^\t \mat T \vect r$
    \State $\omega \gets \vect d^\t \vect r/\vect d^\t \mat A \vect d$
    \State $\vect x \gets \vect x - \omega \vect d$
    \State $\vect r \gets \mat A \vect x - \vect b$
\EndWhile
\end{algorithmic}
\end{algorithm}
\vspace{-.5cm}

\subsubsection{Conjugate directions method}%
\label{ssub:conjugate_directions}
\begin{definition}
    [Conjugate directions]
    Let $\mat A$ be a symmetric positive definite matrix.
    Two vectors $\vect d_1$ and $\vect d_2$ are called $\mat A$-orthogonal or conjugate with respect to $\mat A$
    if $\vect d_1^\t \mat A \vect d_2 = 0$,
    i.e.\ if they are orthogonal for the inner product $\ip{\vect x, \vect y}_{\mat A} = \vect x^\t \mat A \vect y$.
\end{definition}

Assume that $\vect d_0, \dotsc, \vect d_{n-1}$ are $n$ pairwise $\mat A$-orthogonal nonzero directions.
By \cref{exercise:linear_independence_conjugate_directions},
these vectors are linearly independent,
and so they form a basis of $\real^n$.
Consequently, for any initial guess $\vect x^{(0)}$,
the vector $\vect x^{(0)} - \vect x_*$,
where $\vect x_*$ is the solution to the linear system $\mat A \vect x = \vect b$,
can be decomposed as
\[
    \vect x^{(0)} - \vect x_* = \alpha_0 \vect d_0 + \dotsb + \alpha_{n-1} \vect d_{n-1}.
\]
Taking the $\ip{\placeholder, \placeholder}_{\mat A}$ inner product of both sides with $\vect d_k$,
with $k \in \{0, \dotsc, n-1\}$,
we obtain an expression for the scalar coefficient $\alpha_k$:
\[
    \alpha_k = \frac{\vect d_k^\t \mat A (\vect x^{(0)} - \vect x_*)}{\vect d_k^\t \mat A \vect d_k}
    = \frac{\vect d_k^\t (\mat A \vect x^{(0)} - \vect b)}{\vect d_k^\t \mat A \vect d_k}.
\]
Therefore, calculating the expression of the coefficient does not require to know the exact solution $\vect x_*$,
but only the residual $\vect r^{(0)}$!
Given conjugate directions, the exact solution can be obtained as
\begin{equation}
    \label{eq:linear_expansion_conjugate_direction}
    \vect x_* = \vect x^{(0)} - \sum_{k=0}^{n-1} \alpha_k \vect d_k, \qquad \alpha_k = \frac{\vect d_k^\t \vect r^{(0)}}{\vect d_k^\t \mat A \vect d_k}.
\end{equation}

If $\vect x^{(0)} = \vect 0$,
then $\vect r^{(0)} = -\vect b$ and this equations gives that
\[
    \vect x_* = \sum_{k=0}^{n-1} \frac{\vect d_k^\t \vect b}{\vect d_k^\t \mat A \vect d_k} \vect d_k
    = \left(\sum_{k=0}^{n-1} \frac{\vect d_k \vect d_k^\t}{\vect d_k^\t \mat A \vect d_k}\right) \vect b,
\]
which implies that that the inverse of~$\mat A$ is given by
\[
    \mat A^{-1} = \sum_{k=0}^{n-1} \frac{\vect d_k \vect d_k^\t}{\nu_k}, \qquad \nu_k = \vect d_k^\t \mat A \vect d_k.
\]
The conjugate directions method is illustrated in~\cref{algo:conjugate_directions}.
Its implementation is very similar to that of the steepest descent method,
the only difference being that the descent direction at iteration $k$ is given by $\vect d_k$ instead of~$\vect r^{(k)}$.
In particular, the step size at each iteration is such that $f(\vect x^{(k+1)})$ is minimized.
\begin{algorithm}
    \caption{Conjugate directions method}%
    \label{algo:conjugate_directions}%
    \begin{algorithmic}[1]
        \State Assuming $\vect d_0, \dotsc, \vect d_{n-1}$ are given.
        \State Pick initial $\vect x^{(0)}$
        \For{$k$ in $\{0, \dotsc, n-1\}$}
            \State $\vect r^{(k)} = \mat A \vect x^{(k)} - \vect b$
            \State $\omega_k = \vect d_k^\t \vect r^{(k)}/\vect d_k^\t \mat A \vect d_k$
            \State $\vect x^{(k+1)} = \vect x^{(k)} - \omega_k \vect d_k$
        \EndFor
    \end{algorithmic}
\end{algorithm}

Let us now establish the connection between the~\cref{algo:conjugate_directions} and~\eqref{eq:linear_expansion_conjugate_direction},
which may not be immediately apparent because~\eqref{eq:linear_expansion_conjugate_direction} involves only the initial residual~$\mat A \vect x^{(0)} - \vect b$,
while the residual at the current iteration~$\vect r^{(k)}$ is used in the algorithm.
\begin{proposition}
    [Convergence of the conjugate directions method]
    \label{proposition:conjugate_directions}
    The vector $\vect x^{(k)}$ obtained after $k$ iterations of the conjugate directions method is given by
    \begin{equation}
        \label{eq:series_expansion_of_y}
        \vect x^{(k)} = \vect x^{(0)}
        - \sum_{i=0}^{k-1} \alpha_i \vect d_i,
        \qquad \alpha_i = \frac{\vect d_i^\t \vect r^{(0)}}{\vect d_i^\t \mat A \vect d_i}.
    \end{equation}
    In particular,
    the method converges in at most $n$ iterations.
\end{proposition}
\begin{proof}
    Let us denote by $\vect y^{(k)}$ the solution obtained after $k$ steps of \cref{algo:conjugate_directions}.
    Our goal is to show that $\vect y^{(k)}$ coincides with $\vect x^{(k)}$ defined in~\eqref{eq:series_expansion_of_y}.
    The result is trivial for $k = 0$.
    Reasoning by induction,
    we assume that it is true up to~$k$.
    Then performing step $k+1$ of the algorithm gives
    \[
        \vect y^{(k+1)} = \vect y^{(k)} - \omega_k \vect d_k, \qquad \omega_k = \frac{\vect d_k^\t \vect r^{(k)}}{\vect d_k^\t \mat A \vect d_k}.
    \]
    On the other hand,
    it holds from~\eqref{eq:series_expansion_of_y} that
    \[
        \vect x^{(k+1)} = \vect x^{(k)} - \alpha_k \vect d_k, \qquad \alpha_k = \frac{\vect d_k^\t \vect r^{(0)}}{\vect d_k^\t \mat A \vect d_k}.
    \]
    By the induction hypothesis, it holds that $\vect y^{(k)} = \vect x^{(k)}$,
    so in order to prove that $\vect y^{(k+1)} = \vect x^{(k+1)}$,
    it is sufficient to show that $\omega_k = \alpha_k$, i.e. that
    \[
        \vect d_k^\t \vect r^{(k)} = \vect d_k^\t \vect r^{(0)}
        \quad \Leftrightarrow \quad
        \vect d_k^\t (\vect r^{(k)} - \vect r^{(0)}) = 0
        \quad \Leftrightarrow \quad
        \vect d_k^\t \mat A (\vect x^{(k)} - \vect x^{(0)}) = 0.
    \]
    The latter equality is obvious from the $\mat A$-orthonormality of the directions.
\end{proof}

Since $\omega_k$ in~\cref{algo:conjugate_directions} coincides with the expression in~\eqref{eq:linear_optimal_step_size_steepest_descent},
the conjugate directions algorithm satisfies the following ``local optimization'' property:
the iterate $\vect x^{(k+1)}$ minimizes $f$ on the straight line $\omega \mapsto \vect x^{(k)} - \omega \vect d_k$.
In contrast with the steepest descent method,
however,
the conjugate directions method also satisfies the following stronger property.
\begin{proposition}
    [Optimality of the conjugate directions method]
    \label{proposition:optimality_conjugate_directions}
    The iterate $\vect x^{(k)}$ is the minimizer of $f$ over the set $\vect x^{(0)} + \mathcal B_k$,
    where $\mathcal B_k = \Span \{\vect d_0, \dotsc, \vect d_{k-1} \}$.
\end{proposition}
\begin{proof}
    By~\eqref{eq:linear_expansion_conjugate_direction},
    it holds that
    \begin{equation*}
        \vect x_* = \vect x^{(0)} - \sum_{i=0}^{n-1} \alpha_i \vect d_i,
        \qquad
        \alpha_i = \frac{\vect d_i^\t \vect r^{(0)}}{\vect d_i^\t \mat A \vect d_i}
    \end{equation*}
    On the other hand,
    any vector $\vect y \in \vect x^{(0)} + \mathcal B_k$ can be expanded as
    \[
        \vect y = \vect x^{(0)} - \beta_0 \vect d_0 - \dotsb - \beta_{k-1} \vect d_{k-1}.
    \]
    Employing these two expressions, the formula for $f$ in~\eqref{eq:rewritten_expression_of_f},
    and the $\mat A$-orthogonality of the directions,
    we obtain
    \begin{align*}
        f(\vect y)
        &= \frac{1}{2} (\vect y - \vect x_*)^\t \mat A (\vect y - \vect x_*) - \frac{1}{2} \vect x_*^\t \mat A \vect x_* \\
        &= \frac{1}{2} \sum_{i=0}^{k-1} (\beta_i - \alpha_i)^2 \vect d_i^\t \mat A \vect d_i
        + \frac{1}{2} \sum_{i=k}^{n-1} \alpha_i^2 \vect d_i^\t \mat A \vect d_i - \frac{1}{2} \vect x_*^\t \mat A \vect x_*
    \end{align*}
    This is minimized when $\beta_i = \alpha_i$ for all $i \in \{0, \dotsc, k-1\}$,
    in which case $\vect y$ coincides with the $k$-th iterate $\vect x^{(k)}$ of the conjugate directions method in view of~\cref{proposition:conjugate_directions}.
\end{proof}
\begin{remark}
    Let $\norm{\placeholder}_{\mat A}$ denote the norm induced by the inner product $\ip{\placeholder,\placeholder}_{\mat A}$.
    Since
    \[
        \norm{\vect x^{(k)} - \vect x_*}_{\mat A} = \sqrt{2f(\vect x^{(k)}) + \vect x_*^\t \mat A \vect x_*},
    \]
    \cref{proposition:optimality_conjugate_directions} shows that $\vect x^{(k)}$ minimizes the norm~$\norm{\vect x^{(k)} - \vect x_*}_{\mat A}$
    over $\vect x^{(0)} + \mathcal  B_k$.
    This is not surprising since,
    by construction, the vector $\vect x^{(k)} - \vect x_*$ is the orthogonal projection
    of $\vect x^{(0)} - \vect x_*$ onto $\mathcal B_k^\perp$,
    for the inner product $\ip{\placeholder,\placeholder}_{\mat A}$.
    % Equation~\eqref{eq:series_expansion_of_y} shows that $\vect x^{(k)} - \vect x^{(0)}$ is the orthogonal projection,
    % with respect to the inner product $(\vect x, \vect y) \mapsto \vect x^\t \mat A \vect y$,
    % of the vector $\vect x_* - \vect x^{(0)}$ over $\mathcal B_k$.
    % In order to prove \cref{proposition:optimality_conjugate_directions},
    % we could have simply invoked the property that the orthogonal projections in Hilbert spaces,
    % it holds that $\vect x^{(k)} - \vect x^{(0)}$ minimizes
    % \[
    %     \norm[big]{(\vect x^{(k)} - \vect x^{(0)}) - (\vect x_* - \vect x^{(0)})}_{\mat A} = \norm{\vect x^{(k)} - \vect x_*}_{\mat A} = \sqrt{2f(\vect x^{(k)}) + \vect x_*^\t \mat A \vect x_*}.
    % \]
\end{remark}

A corollary of~\eqref{proposition:optimality_conjugate_directions} is that the gradient of~$f$ at $\vect x^{(k)}$,
i.e. the residual $\vect r^{(k)} = \mat A \vect x^{(k)} - \vect b$,
is orthogonal to any vector in $\{\vect d_0, \dotsc, \vect d_{k-1} \}$ for the usual Euclidean inner product.
This can also be checked directly from the formula
\[
    \vect x^{(k)} - \vect x_* = \sum_{i=k}^{n-1} \alpha_i \vect d_i,
    \qquad \alpha_i = \frac{\vect d_i^\t \vect r^{(0)}}{\vect d_i^\t \mat A \vect d_i},
\]
which follows directly from \cref{proposition:conjugate_directions}.
Indeed, it holds that
\begin{equation}
    \label{eq:orthogonality_residual}
    \forall j \in \{0, \dotsc, k-1\}, \qquad
    \vect d_j^\t \vect r^{(k)} = \vect d_j \mat A (\vect x^{(k)} - \vect x_*)
    = \sum_{i=k}^{n-1} \alpha_i \vect d_j^\t \vect d_i = 0.
\end{equation}

\subsubsection{The conjugate gradient method}%
\label{ssub:the_conjugate_gradient_method}
In the previous section,
we showed that, given $n$ conjugate directions,
the solution to the linear system $\mat A \vect x = \vect b$ can be obtained in a finite number of iterations using~\cref{algo:conjugate_directions}.
The conjugate gradient method can be viewed as a particular case of the conjugate directions method.
Instead of assuming that the conjugate directions are given,
they are constructed iteratively as part of the algorithm.
Given an initial guess $\vect x^{(0)}$,
the first direction is the residual~$\vect r^{(0)}$,
which coincides with the gradient of~$f$ at~$\vect x^{(0)}$.
The directions employed for the next iterations are obtained by applying the Gram-Schmidt process
to the residuals.
More precisely, given conjugate directions~$\vect d_0, \dotsc, \vect d_{k-1}$,
and letting $\vect x^{(k)}$ denote the $k$-th iterate of the conjugate directions method,
the direction $\vect d_k$ is obtained by
\begin{equation}
    \label{eq:conjugate_gradient_gram_schmidt}
    \vect d_k = \vect r^{(k)} - \sum_{i=0}^{k-1} \frac{\vect d_i^\t \mat A \vect r^{(k)}}{\vect d_i^\t \mat A \vect d_i} \vect d_i,
    \qquad \vect r^{(k)} = \mat A \vect x^{(k)} - \vect b.
\end{equation}
It is simple to check that $\vect d_k$  is indeed $\mat A$-orthogonal to $\vect d_i$ for $i \in \{0, \dotsc, k-1\}$,
and that $\vect d_k$ is nonzero if $\vect r^{(k)}$ is nonzero.
To prove the latter claim,
we can take the Euclidean inner product of both sides with $\vect r^{(k)}$ and use~\cref{proposition:optimality_conjugate_directions}
to deduce that
\begin{equation}
    \label{eq:conjugate_gradient_positive_omega}
    \vect d_k^\t \vect r^{(k)} = (\vect r^{(k)} )^\t \vect r^{(k)} > 0.
\end{equation}
Note also that since the directions are obtained by applying the Gram--Schmidt process to the residuals,
it holds that
\begin{equation}
    \label{eq:conjugate_gradient_equality_span}
    \forall k \in \{0, \dotsc, n-1\}, \qquad
    \mathcal B_{k+1} := \Span\left\{ \vect d_0, \dotsc, \vect d_{k} \right\} =
    \Span\left\{ \vect r^{(0)}, \dotsc, \vect r^{(k)} \right\}.
\end{equation}
The following result characterizes precisely the subspace $\mathcal B_{k+1}$.
% for the conjugate gradient descent.
\begin{proposition}
    \label{proposition:krylov_subspace}
    Assume that $\norm{\vect r^{(k)}} \neq 0$ for all $k < m \leq n$.
    Then it holds that
    \begin{equation}
        \label{eq:conjugate_gradient_krylov}
        \forall k \in \{0, \dotsc m\}, \quad
        \Span\left\{ \vect r^{(0)}, \vect r^{(1)}, \dotsc, \vect r^{(k)} \right\}
        = \Span\left\{ \vect r^{(0)}, \mat A \vect r^{(0)}, \dotsc, \mat A^k \vect r^{(0)} \right\}
    \end{equation}
    The subspace on the right-hand side is called a Krylov subspace.
\end{proposition}
\begin{proof}
    The result is clear for $k = 0$.
    Reasoning by induction,
    we prove that if the result is true up to $k < m$,
    then it is also true for $k+1$.
    A simple calculation gives that
    \begin{align}
        \nonumber
        \vect r^{(k+1)}
        &= \mat A \left( \vect  x^{(k)} - \omega_k \vect d_k \right)  - \vect b  \\
        \label{eq:residual_iteration}
        &= \vect r^{(k)} - \omega_k \mat A \vect d_k.
    \end{align}
    From~\eqref{eq:conjugate_gradient_gram_schmidt},
    we deduce that
    \[
        \vect r^{(k+1)}
        = \vect r^{(k)} - \omega_k \mat A \left( \vect r^{(k)} - \sum_{i=0}^{k-1} \frac{\vect d_i^\t \mat A \vect r^{(k)}}{\vect d_i^\t \mat A \vect d_i} \vect d_i \right).
    \]
     By~\eqref{eq:conjugate_gradient_equality_span} and the induction hypothesis,
    the bracketed expression on the right-hand side belongs to $\mathcal B_{k+1}$,
    so the inclusion~$\subset$ in~\eqref{eq:conjugate_gradient_krylov} is clear.
    The inclusion $\supset$ then follows from the fact the dimension of the subspace
    \[
        \Span\left\{ \vect d_0, \dotsc, \vect d_{k} \right\} =
        \Span\left\{ \vect r^{(0)}, \dotsc, \vect r^{(k)} \right\}
    \]
    is equal to $k+1$.
\end{proof}

It appears from~\eqref{eq:conjugate_gradient_gram_schmidt} that the cost of calculating a new direction grows linearly with the iteration index.
In fact, it turns out that only the last term in the sum is nonzero,
and so the cost of calculating a new direction does not grow with the iteration index~$k$.
Indeed, notice that if $i \leq k-2$, then
\[
    \vect d_i^\t \mat A \vect r^{(k)}
    = (\mat A \vect d_i)^\t \mat A \vect r^{(k)}
    = 0,
\]
because $\mat A \vect d_i \in \mathcal B_{i+2} \subset \mathcal B_{k}$ by~\cref{proposition:krylov_subspace},
and $\vect r^{(k)}$ orthogonal to~$\mathcal B_k$ for the Euclidean inner product by~\eqref{eq:orthogonality_residual}.
This observation leads to~\cref{algo:conjugate_gradient}.
\begin{algorithm}
    \caption{Conjugate gradient method}%
    \label{algo:conjugate_gradient}%
    \begin{algorithmic}[1]
        \State Pick initial $\vect x^{(0)}$
        \State $\vect d_0 = \vect r^{(0)} = \mat A \vect x^{(0)} - \vect b$
        \For{$k$ in $\{0, \dotsc, n-1\}$}
            \If{$\norm{\vect r^{(k)}} = 0$}
                \State{Stop}
            \EndIf
            \State $\omega_k = \vect d_k^\t \vect r^{(k)}/\vect d_k^\t \mat A \vect d_k$
            \State $\vect x^{(k+1)} = \vect x^{(k)} - \omega_k \vect d_k$
            \State $\vect r^{(k+1)} = \mat A \vect x^{(k+1)} - \vect b$
            \State $\beta_k  = \vect d_k^\t \mat A \vect r^{(k+1)}/\vect d_k^\t \mat A \vect d_k$.
            \State $\vect d_{k+1}  = \vect r^{(k+1)} - \beta_k \vect d_k$.
        \EndFor
    \end{algorithmic}
\end{algorithm}

Although the conjugate gradient method converges in a finite number of iterations,
performing $n$ iterations for very large systems  would require an excessive computational cost,
and so it is sometimes desirable to stop iterating when the residual is sufficiently small.
To conclude this section, we study the convergence of the method.
\begin{theorem}
    [Convergence of the conjugate gradient method]
    \label{theorem:convergence_conjugate_gradient}
    The error for the conjugate gradient method,
    measured as
    \[
        E_{k} := (\vect x^{(k)} - \vect x_*)^\t  \mat A (\vect x^{(k)} - \vect x_*),
    \]
    satisfies the following inequality:
    \begin{equation}
        \label{equation:convergence_conjugate_gradient}
        \forall q_k \in \poly(k), \qquad
        E_{k+1} \leq  \max_{1\leq i \leq n}\bigl(1 + \lambda_i q_k(\lambda_i) \bigr)^2   E_0.
    \end{equation}
    Here $\poly(k)$ is the vector space of polynomials of degree less than or equal to $k$.
\end{theorem}
\begin{proof}

In view of \cref{proposition:krylov_subspace},
the iterate $\vect x^{(k+1)}$ can be written as
\[
    \vect x^{(k+1)} = \vect x^{(0)} + \sum_{i=0}^{k} \alpha_i \mat A^i \vect r^{(0)}
    = \vect x^{(0)} + p_k(\mat A) \vect r^{(0)},
\]
where $p_k$ is a polynomial of degree $k$.
By~\cref{proposition:optimality_conjugate_directions},
$p_k$ is in fact the polynomial of degree~$k$ such that $f(\vect x^{(k+1)})$ is minimized,
and thus also $E_{k+1}$ by~\eqref{eq:rewritten_expression_of_f}.
Noting that
\begin{align*}
    \vect x^{(k+1)} - \vect x_*
    &= \vect x^{(0)}  - \vect x_* + p_k(\mat A) \vect r^{(0)}
    = \vect x^{(0)}  - \vect x_* + p_k(\mat A) \mat A (\vect x^{(0)} - \vect x_*) \\
    &= \bigl(\mat I + \mat A p_k(\mat A) \bigr) (\vect x^{(0)} - \vect x_*),
\end{align*}
we deduce that
\[
    \forall q_k \in \poly(k), \qquad
    E_{k+1} \leq (\vect x^{(0)} - \vect x_*)^\t \mat A \bigl(\mat I + \mat A q_k(\mat A) \bigr)^2  (\vect x^{(0)} - \vect x_*).
\]
In order to exploit this inequality,
it is useful to diagonalize $\mat A$ as $\mat A = \mat Q \mat D \mat Q^\t$,
for an orthogonal matrix $\mat Q$ and a diagonal matrix $\mat D$.
Since $q_k(\mat A) = \mat Q q_k(\mat D) \mat Q^\t$ for all $q_k \in \poly(k)$,
it holds that
\begin{align*}
    \forall q_k \in \poly(k), \qquad
    E_{k+1}
    &= \bigl(\mat Q^\t (\vect x^{(0)} - \vect x_*)\bigr)^\t \mat D \bigl(\mat I + \mat D q_k(\mat D) \bigr)^2  \bigl(\mat Q^\t (\vect x^{(0)} - \vect x_*)\bigr) \\
    &\leq  \max_{1\leq i \leq n}\bigl(1 + \lambda_i q_k(\lambda_i) \bigr)^2   \underbrace{\bigl(\mat Q^\t (\vect x^{(0)} - \vect x_*)\bigr)^\t \mat D   \bigl(\mat Q^\t (\vect x^{(0)} - \vect x_*)\bigr)}_{E_0},
\end{align*}
which completes the proof.
\end{proof}

A corollary of~\cref{theorem:convergence_conjugate_gradient} is that,
if $\mat A$ has $m \leq n$ distinct eigenvalues,
then the conjugate gradient method converges in at most $m$ iterations.
Indeed, in this case we can take
\[
    q_{m-1}(\lambda) = \frac{1}{\lambda} \left( \frac{(\lambda_1 - \lambda) \dotsc (\lambda_m - \lambda)}{\lambda_1 \dotsc \lambda_m} - 1 \right).
\]
It is simple to check that the right-hand side is indeed a polynomial,
and that $1 + \lambda_i  q_{m-1}(\lambda_i) = 0$ for all eigenvalues of $\mat A$.

In general, finding the polynomial that minimizes the right-hand side of~\eqref{equation:convergence_conjugate_gradient} is not possible,
because the eigenvalues of $\mat A$ are unknown.
However, it is possible to derive from this equation an error estimate with an explicit dependence on the condition number~$\kappa = \kappa_2(\mat A)$.
\begin{theorem}
    It holds that
    \[
        \forall k \geq 0, \qquad
        E_{k} \leq  4 \left( \frac{\sqrt{\kappa} - 1}{\sqrt{\kappa} + 1} \right)^{2k} E_0,
    \]
\end{theorem}
\begin{proof}
\Cref{theorem:convergence_conjugate_gradient} implies that
\begin{equation*}
    \forall q_k \in \poly(k), \qquad
    E_{k+1} \leq  \max_{\lambda \in [\lambda_1, \lambda_n]}\bigl(1 + \lambda q_k(\lambda) \bigr)^2   E_0,
\end{equation*}
where $\lambda_1$ and $\lambda_n$ are the minimum and maximum eigenvalues of $\mat A$.
Notice that
\[
    \Bigl\{ 1 + \lambda q_k: q_k \in \poly(k) \Bigr\}
    = \Bigl\{p_k : p_k \in \poly(k+1) \text{ and } p_k(0) = 1 \Bigr\}
\]
Therefore, it follows from~\cref{exercise:conjugate_gradient_chebyshev} that the right-hand side is minimized when
\begin{equation}
    \label{eq:conjugate_gradient_optimal_polynom}
    1 + \lambda q_k(\lambda) = \frac{T_{k+1}\left(\frac{\lambda_n + \lambda_1 - 2\lambda}{\lambda_n - \lambda_1}\right)}{T_{k+1}\left(\frac{\lambda_n + \lambda_1}{\lambda_n - \lambda_1}\right)},
\end{equation}
where $T_{k+1}$ is the \emph{Chebyshev} polynomial of degree $k+1$,
see~\eqref{eq:chebyshev}.
We recall that $|T_{k+1}(x)| \leq 1$ for all $x \in [-1, 1]$.
Consequently, by the expression of Chebyshev polynomials given in~\cref{exercise:yet_another_expression_cheby},
the following inequality holds true for all $\lambda \in [\lambda_1, \lambda_n]$:
\begin{align*}
    \bigl\lvert 1 + \lambda q_k(\lambda) \bigr\rvert
    \leq \frac{1}{T_{k+1}\left(\frac{\lambda_n + \lambda_1}{\lambda_n - \lambda_1}\right)}
    &= 2\left( \Big(r+\sqrt{r^2-1} \Big)^{k+1} + \Big(r-\sqrt{r^2-1}\Big)^{k+1} \right)^{-1},  \\
    &= 2 \left( \left( \frac{\sqrt{\kappa} + 1}{\sqrt{\kappa} - 1} \right)^{k+1} + \left( \frac{\sqrt{\kappa} - 1}{\sqrt{\kappa} + 1} \right)^{k+1} \right)^{-1}.
\end{align*}
where $r = \frac{\lambda_n + \lambda_1}{\lambda_n - \lambda_1}$.
Since the first term in the bracket converges to zero as $k \to \infty$,
it is natural to bound this expression by keeping only the second term,
which after simple algebraic manipulations leads to
\[
    \forall \lambda \in [\lambda_1, \lambda_n], \qquad
    \bigl\lvert 1 + \lambda q_k(\lambda) \bigr\rvert
    \leq 2 \left( \frac{\sqrt{\kappa} - 1}{\sqrt{\kappa} + 1} \right)^{k+1}.
\]
From this inequality,
the statement of the theorem follows immediately.
\end{proof}
