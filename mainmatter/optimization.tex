\chapter{Optimization}%
\label{cha:Optimization}

\minitoc

In this chapter, we focus on optimization problems of the following form:
\begin{equation}
    \label{eq:optimization}
    \text{Find} \quad \argmin_{\vect x \in \mathcal K} J(\vect x),
\end{equation}
where $\mathcal K$ is a given subset of $\real^n$ and $J\colon \mathcal K \to \real$ is a given \emph{objective function}.
% \text{ Find $u \in \mathcal K \subset \real^n$ such that $J(u) \leq J(v)$ for all $v \in \mathcal K$ },
We came across several examples of such problems earlier in these notes:
\begin{itemize}
    \item
        In \cref{cha:interpolation_and_approximation},
        in the context of least-squares approximation,
        we considered the problem of minimizing
        \[
            J(\vect \alpha) = \frac{1}{2} \norm{ \mat A \vect x - \vect b }^2.
        \]

    \item
        In \cref{cha:solution_of_linear_systems},
        we observed,
        if $\mat A$ is a symmetric and positive definite matrix,
        then solving the linear system~$\mat A \vect x = \vect b$
        amounts to finding the minimizer of the functional
        \[
            J(\vect x)  = \frac{1}{2} \vect x^\t \mat A \vect x - \vect b^\t \vect x.
        \]
\end{itemize}
When $\mathcal K = \real^n$,
equation~\eqref{eq:optimization} is an \emph{unconstrained} optimization problem,
and when~$\mathcal K \subsetneq \real^n$,
equation~\eqref{eq:optimization} is a \emph{constrained} optimization problem.
In practice,
the set~$\mathcal K$ is often an intersection of sets of the form
\[
    \bigl\{ \vect x \in \real^n : \phi(x) \leq 0 \bigr\},
    \qquad
    \text{ or }
    \qquad
    \bigl\{ \vect x \in \real^n : \phi(x) = 0 \bigr\},
\]
for appropriate $\phi\colon \real^n \to \real^m$.
Constraints of the former form are called \emph{inequality constraints},
while constraints of the latter form are called \emph{equality constraints}.
Our aim in this chapter is to give a brief introduction to numerical optimization.
We focus on the simplest method,
namely the \emph{steepest descent method} with fixed step.
The rest of this chapter is organized as follows:
\begin{itemize}
    \item
        We begin in~\cref{sec:convexity} by defining the notions of~\emph{convexity}, \emph{strict convexity} and \emph{strong convexity},
        which play an important role in optimization.

    \item
        Then, in~\cref{sec:unconstrained_optimization},
        we analyze the steepest descent method with fixed step in the setting of unconstrained optimization.
        To this end, we first establish conditions under which~\eqref{eq:optimization} is well posed.

    \item
        Finally, in~\cref{sec:constrained_optimization},
        we extend the steepest descent method to the case of optimization with constraints.
\end{itemize}

\begin{remark}
    For generality, we could consider the setting where the set~$\mathcal K$ in~\eqref{eq:optimization} is a subset of some finite dimensional or infinite dimensional vector space~$V$.
    An optimization problem over (a subset of) a finite dimensional vector space of dimension~$n$ can always be recast as an optimization problem over (a subset of) $\real^n$
    -- the type we study in this chapter -- by fixing a basis.
    The case of an infinite dimensional vector space, however,
    is more delicate, and we do not address it here.
\end{remark}

\section{Definition and characterization of convexity}
\label{sec:convexity}

\begin{definition}
    [Convexity]
    Assume that $J\colon \mathcal K \to \real$.
    \begin{itemize}
        \item
            The function~$J$ is said to be convex if
            \begin{equation}
                \label{eq:convexity}
                \forall (\vect x, \vect y) \in \mathcal K \times \mathcal K,
                \qquad \forall \theta \in [0, 1],
                \qquad
                J\bigl(\theta \vect x + (1 - \theta) \vect y\bigr)
                \leq \theta J(\vect x) + (1 - \theta) J(\vect y).
            \end{equation}

        \item
            The function~$J$ is called \emph{strictly convex} if~\eqref{eq:convexity} holds with strict inequality
            if $\vect x \neq \vect y$ and~$\theta \in (0, 1)$.

        \item
            The function~$J$ is called \emph{strongly convex} with parameter~$\alpha > 0$ if
            for all $(\vect x, \vect y) \in \mathcal K \times \mathcal K$ and for all~$\theta \in [0, 1]$,
            \begin{equation}
                \label{eq:strong_convexity}
                J\bigl(\theta \vect x + (1 - \theta) \vect y\bigr)
                \leq \theta J(\vect x) + (1 - \theta) J(\vect y)
                - \frac{\alpha}{2} \theta (1 - \theta) \norm{\vect x - \vect y}^2.
            \end{equation}
    \end{itemize}
\end{definition}

If the function~$J$ is differentiable,
then convexity, strict convexity and strong convexity can be characterized in terms of the gradient~$\nabla J$.
We illustrate this for strong convexity,
noting that a characterization of convexity is obtained by substituting~$\alpha = 0$ in the following result.

\begin{proposition}
    A function~$J\colon \real^n \to \real$ is strongly convex with parameter~$\alpha$ if and only if
    \begin{equation}
        \label{eq:strong_convex_eq1}
        \forall (\vect x, \vect y) \in \real^n \times \real^n,
        \qquad J(\vect x) \geq J(\vect y) + \ip[\big]{\nabla J(\vect y), \vect x - \vect y} + \frac{\alpha}{2} \norm{\vect x - \vect y}^2,
    \end{equation}
    or, equivalently,
    \begin{equation}
        \label{eq:strong_convex_eq2}
        \forall (\vect x, \vect y) \in \real^n \times \real^n,
        \qquad \ip[\big]{\nabla J(\vect x) - \nabla J(\vect y), \vect x - \vect y}
        \geq \alpha \norm{\vect x - \vect y}^2.
    \end{equation}
\end{proposition}
\begin{proof}
    For clarity, we divide the proof into items and prove one implication per item.
    \begin{itemize}
        \item
            \eqref{eq:strong_convexity} $\Rightarrow$~\eqref{eq:strong_convex_eq1}.
            Rearranging~\eqref{eq:strong_convexity}, we have
            \[
                \frac{J\bigl( \vect y + \theta (\vect x - \vect y) \bigr) - J(\vect y)}{\theta}
                \leq  J(\vect x)  - J(\vect y)  - \frac{\alpha}{2} (1 - \theta) \norm{\vect x - \vect y}^2.
            \]
            Taking the limit~$\theta \to 0$,
            we deduce that
            \[
                \ip[\big]{\nabla J(\vect y), \vect x - \vect y}
                \leq  J(\vect x)  - J(\vect y)  - \frac{\alpha}{2}  \norm{\vect x - \vect y}^2.
            \]
            This gives~\eqref{eq:strong_convex_eq1} after rearranging.

    \item
        \eqref{eq:strong_convex_eq1} $\Rightarrow$~\eqref{eq:strong_convexity}.
        To prove this implication, suppose that~\eqref{eq:strong_convex_eq1} holds,
        take $(\vect x, \vect y) \in \real^n \times \real^n$
        and let $\vect z = \theta \vect x + (1 - \theta) \vect y$.
        Using~\eqref{eq:strong_convex_eq1} successively with $(\vect x, \vect z)$ and~$(\vect y, \vect z)$,
        we deduce
        \begin{align*}
            J(\vect x) &\geq J(\vect z) + \ip[\big]{\nabla J(\vect z), \vect x - \vect z} + \frac{\alpha}{2} \norm{\vect x - \vect z}^2, \\
            J(\vect y) &\geq J(\vect z) + \ip[\big]{\nabla J(\vect z), \vect y - \vect z} + \frac{\alpha}{2} \norm{\vect y - \vect z}^2.
        \end{align*}
        Combining these inequalities,
        we deduce that
        \begin{align*}
            \theta J(\vect x) + (1 - \theta) J(\vect y)
            &\geq J(\vect z) + \ip[\Big]{\nabla J(\vect z), \theta \vect x + (1 - \theta) \vect y - \vect z} \\
            &\qquad + \frac{\alpha \theta}{2} \norm{\vect x - \vect z}^2 + \frac{\alpha(1 - \theta)}{2}  \norm{\vect y - \vect z}^2 \\
            &= J(\vect z) + 0 + \frac{\alpha}{2} \theta (1-\theta) \norm{\vect x - \vect y}^2.
        \end{align*}
        Rearranging gives~\eqref{eq:strong_convexity}.

    \item
        \eqref{eq:strong_convex_eq1} $\Rightarrow$~\eqref{eq:strong_convex_eq2}.
        Assuming that~\eqref{eq:strong_convex_eq1} holds and applying this inequality first to~$(\vect x, \vect y)$ and then to~$(\vect y, \vect x)$,
        we obtain
        \begin{align*}
            J(\vect x) &\geq J(\vect y) + \ip[\big]{\nabla J(\vect y), \vect x - \vect y} + \frac{\alpha}{2} \norm{\vect x - \vect y}^2 \\
            J(\vect y) &\geq J(\vect x) + \ip[\big]{\nabla J(\vect x), \vect y - \vect x} + \frac{\alpha}{2} \norm{\vect x - \vect y}^2.
        \end{align*}
        Adding these equations and rearranging,
        we deduce~\eqref{eq:strong_convex_eq2}.

    \item
        \eqref{eq:strong_convex_eq2} $\Rightarrow$~\eqref{eq:strong_convex_eq1}.
        Suppose that~\eqref{eq:strong_convex_eq2} holds and
        take $(\vect x, \vect y) \in \real^n \times \real^n$.
        Using the fundamental theorem of analysis and~\eqref{eq:strong_convex_eq2},
        we have
        \begin{align*}
            J(\vect x)
            &=  J(\vect y) + \int_{0}^{1} \ip[\Big]{\nabla J\bigl(\vect y + \theta (\vect x - \vect y) \bigr), \vect x - \vect y} \, \d \theta \\
            &\geq J(\vect y) + \int_{0}^{1} \ip[\Big]{\nabla J(\vect y), \vect x - \vect y} + \alpha \theta \norm{\vect x - \vect y}^2 \, \d \theta \\
            &= J(\vect y) + \ip[\Big]{\nabla J(\vect y), \vect x - \vect y} + \frac{\alpha}{2} \norm{\vect x - \vect y}^2,
        \end{align*}
        which gives~\eqref{eq:strong_convex_eq1}.
    \end{itemize}
    We have proved all the implications required to conclude the proof.
\end{proof}

\section{Unconstrained optimization}
\label{sec:unconstrained_optimization}
Throughout this section $\mathcal K = \real^n$.
We begin by establishing conditions under which the optimization problem~\eqref{eq:optimization} admits a unique solution in this setting.
We first prove existence of a global minimizer under appropriate conditions.
\begin{proposition}
    [Existence of a global minimizer]
    \label{proposition:existence_minimizer}
    Suppose that $J \colon \real^n \to \real$ is continuous and coercive,
    the latter meaning that $J(\vect x) \to \infty$ when $\norm{\vect x} \to \infty$.
    Then there exists a global minimizer of~$J$ in $\real^n$.
\end{proposition}
\begin{proof}
    Let $(\vect x_n)_{n \in \nat}$ be a minimizing sequence of~$J$,
    i.e.\ a sequence in $\real^n$ such that
    \[
        J(\vect x_n) \to \inf_{\vect x \in \real^n} J(\vect x) \quad \text{ as $n \to \infty$}.
    \]
    By coercivity, the sequence~$(\vect x_n)$ is bounded,
    because otherwise it would hold that $J(\vect x_n)~\to~\infty$.
    Therefore, since closed bounded sets in~$\real^n$ are compact,
    there is a subsequence $(\vect x_{n_k})_{k \in \nat}$ converging to some $\vect x_* \in \real^n$.
    Since $J$ is continuous, we have that
    \[
        J(\vect x_*) = \lim_{k \to \infty} J(\vect x_{n_k}) = \inf_{\vect x \in \real^n} J(\vect x).
    \]
    We conclude that $\vect x_*$ is a minimizer of $J$.
\end{proof}
\begin{remark}
    We relied crucially in the proof of~\cref{proposition:existence_minimizer} on the fact that bounded closed sets in $\real^n$ are compact.
    In the infinite-dimensional setting,
    coercivity and continuity alone are not sufficient to guarantee the existence of a minimizer.
\end{remark}

Uniqueness of the minimizer can be established under a strict convexity assumption.

\begin{proposition}
    [Uniqueness of the minimizer]
    \label{proposition:uniqueness_minimizer}
    If $J$ is strictly convex,
    then there exists at most one global minimizer.
\end{proposition}
\begin{proof}
    Suppose by contradiction that there were two minimizers~$\vect x_*$ and~$\vect y_*$.
    Then by strict convexity we have
    \[
        J \left( \frac{\vect x_* + \vect y_*}{2} \right)
        < \frac{1}{2} \bigl( J(\vect x_*) + J(\vect y_*) \bigr) = J(\vect x_*),
    \]
    which is a contradiction.
\end{proof}

Finally, before introducing the steepest descent algorithm,
we recall the following standard result from analysis,
the proof of which is left as an exercise.

\begin{theorem}
    [Euler condition]
    \label{theorem:euler}
    Suppose that $J\colon \real^n \to \real$ is differentiable.
    \begin{itemize}
        \item
            If $\vect x_*$ is a local minimizer of~$J$,
            then
            \(
                \label{eq:critical_point}
                \nabla J(\vect x_*) = 0.
            \)

        \item
            If $J$ is convex,
            then~$\nabla J(\vect x_*) = 0$ if and only if~$\vect x_*$ is a global minimizer.
    \end{itemize}
\end{theorem}

\paragraph{Steepest descent method.}
In this section, we study the more general version of the steepest descent with \emph{fixed step} given in~\cref{algo:steepest_descent_method_optimization}.
\begin{algorithm}
\caption{Steepest descent method}%
\label{algo:steepest_descent_method_optimization}%
\begin{algorithmic}[1]
    \State Pick $\lambda$, and initial $\vect x_0$.
    \For {$k \in \{0, 1, \dotsc\}$}
        \State $\vect d_k \gets \nabla J(\vect x_k)$
        \State $\vect x_{k+1} \gets \vect x_{k} - \lambda \vect d_k$
    \EndFor
\end{algorithmic}
\end{algorithm}

\begin{remark}
    We encountered the steepest descent with fixed step for a quadratic objective function when we
    analyzed Richardson's method for solving linear equations in~\cref{cha:solution_of_linear_systems}.
\end{remark}

In practice, \cref{algo:steepest_descent_method_optimization} must be supplemented with an appropriate stopping criterion.
This could be, for example, a criterion of the form $\norm{\vect x_{k+1} - \vect x_k} \leq \varepsilon$,
or $\bigl\lvert J(\vect x_{k+1}) - J(\vect x_k) \bigr\rvert \leq \varepsilon$.
It is sometimes also useful to use a normalized criterion of the form $\norm{\vect x_{k+1} - \vect x_k} \leq \varepsilon \norm{\vect x_0}.$
The steepest descent method may be viewed as a fixed point iteration for the function
\begin{equation}
    \label{eq:fixed_point_gradient_descent}
    \vect F_{\lambda}(\vect x) = \vect x - \lambda \nabla J(\vect x).
\end{equation}
A point $\vect x_* \in \real^n$ is a fixed point of this functional if and only if~$\vect x_*$ is a solution to the nonlinear equation $\nabla J(\vect x_*) = 0$.
We shall now prove the convergence of the steepest descent under appropriate assumptions on the function~$J$.
\begin{theorem}
    [Convergence of the steepest descent method]
    \label{theorem:convergence_steepest_descent_optimization}
    Suppose that $J$ is differentiable, strongly convex with parameter~$\alpha$,
    and that its gradient $\nabla J \colon \real^n \to \real^n$ is Lipschitz continuous with parameter~$L$:
    \begin{equation}
        \label{eq:lipschitz_gradient}
        \forall (\vect x, \vect y) \in \real^n \times \real^n, \qquad
        \norm{\nabla J(\vect x) - \nabla J(\vect y)} \leq L \norm{\vect x - \vect y}.
    \end{equation}
    Then provided that
    \begin{equation}
        \label{eq:condition_time_step}
        0 < \lambda < \frac{2 \alpha}{L},
    \end{equation}
    the steepest descent method with fixed step is convergent.
    More precisely,
    there exists $\rho \in (0, 1)$ such that for all~$k \geq 0$
    \begin{equation}
        \label{eq:exponential_convergence}
        \norm{\vect x_k - \vect x_*} \leq \rho^k \norm{\vect x_0 - \vect x_*}.
    \end{equation}
    % In other words, $\vect x_*$ is a globally exponentially stable fixed point for the fixed point iteration based oneq:fixed_point_gradient_descent
\end{theorem}
\begin{proof}
    Under the assumptions of the theorem,
    there exists a unique global minimizer of~$J$,
    which is the unique fixed point of~$\vect F_{\lambda}$.
    We begin by proving that $\vect F_{\lambda}$ defined in~\eqref{eq:fixed_point_gradient_descent} is globally Lipschitz continuous.
    We have
    \begin{align*}
        \norm{ \vect F_{\lambda}(\vect x) - \vect F_{\lambda}(\vect y) }^2
        &= \norm*{ \vect x - \vect y - \lambda \bigl(\nabla J(\vect x) - \nabla J(\vect y) \bigr) }^2 \\
        &= \norm{\vect x - \vect y}^2 - 2 \lambda \ip[\big]{\vect x - \vect y, \nabla J(\vect x) - \nabla J(\vect y)} + \lambda^2 \norm{\nabla J(\vect x) - \nabla J(\vect y)}^2 \\
        &\leq (1 - 2 \alpha \lambda + \lambda^2 L) \norm{\vect x - \vect y}^2,
    \end{align*}
    where we employed~\eqref{eq:strong_convex_eq2} for the second term and~\eqref{eq:lipschitz_gradient} for the third term.
    Thus, $\vect F_{\lambda}$ is globally Lipschitz continuous with constant $\rho = \sqrt{1 - 2 \alpha \lambda + \lambda^2 L}$,
    which is less than 1 if and only~\eqref{eq:condition_time_step} is satisfied.
    The bound~\eqref{eq:exponential_convergence} then follows by noting that
    \[
        \norm{\vect x_k - \vect x_*} = \norm{\vect F_{\lambda}(\vect x_{k-1}) - \vect F_{\lambda} (\vect x_*)}
        \leq \rho \norm{\vect x_{k-1} - \vect x_*} \leq \dots \leq \rho^k \norm{\vect x_0 - \vect x_*},
    \]
    which concludes the proof.
    (Note that~\eqref{eq:exponential_convergence} also follows from~\cref{theorem:exponential_convergence_fixed_point}.)
\end{proof}

\begin{remark}
    [Convergence speed]
    The choice of~$\lambda$ minimizing the Lipschitz constant~$\rho$ is given by $\lambda_* = \frac{\alpha}{L^2}$,
    which corresponds to $\rho_* = 1 - \left(\frac{\alpha}{L}\right)^2$.
    Often, in practice, it holds that $\alpha \ll L$,
    in which case the convergence of the steepest descent with fixed step is slow.
\end{remark}

\section{Constrained optimization}
\label{sec:constrained_optimization}

In this section,
we assume that $\mathcal K \subset \real^n$.
We begin by establishing well-posedness of the optimization problem~\eqref{eq:optimization} in this setting.

\begin{proposition}
    [Well posedness of~\eqref{eq:optimization} in the constrained setting]
    \label{proposition:well-posedness_constrained}
    The two items below concern existence and uniqueness, respectively.
    \begin{itemize}
        \item
            Suppose that~$\mathcal K \subset \real^n$ is \emph{closed} and that $J \colon \mathcal K \to \real$ is continuous and coercive.
            Then there exists a global minimizer of~$J$ in $\mathcal K$.

        \item
            Suppose that $\mathcal K \subset \real^n$ is \emph{convex} and that $J\colon \mathcal K \to \real$ is strictly convex.
            Then there exists at most one global minimizer.
    \end{itemize}
\end{proposition}
\begin{proof}
    The proof is very similar to those of~\cref{proposition:existence_minimizer} and~\cref{proposition:uniqueness_minimizer},
    and so we leave it to the reader.
    Note that the set~$\mathcal K$ must be closed to ensure existence,
    and convex to guarantee uniqueness.
    These assumptions are clearly satisfied when $\mathcal K = \real^n$,
    so~\cref{proposition:well-posedness_constrained} generalizes \cref{proposition:existence_minimizer,proposition:uniqueness_minimizer}.
\end{proof}

The following theorem,
which generalizes~\eqref{theorem:euler},
establishes a characterization of the minimizer when $J$ is differentiable.
\begin{theorem}
    [Euler--Lagrange conditions]
    \label{theorem:euler_lagrange}
    Suppose that $J\colon \mathcal K \to \real$ is differentiable and
    that $\mathcal K \subset \real^n$ is closed and convex.
    Then the following statements hold.
    \begin{itemize}
        \item
            If $\vect x_*$ is a local minimizer of~$J$,
            then
            \begin{equation}
                \label{eq:euler_lagrange}
                \forall \vect x \in \mathcal K,
                \qquad \ip{\nabla J(\vect x_*), \vect x - \vect x_*} \geq 0.
            \end{equation}

        \item
            Conversely, if~\eqref{eq:euler_lagrange} is satisfied and~$J$ is convex,
            then~$\vect x_*$ is a global minimizer of~$J$.
    \end{itemize}
\end{theorem}
\begin{proof}
    Suppose that $\vect x_*$ is a local minimizer of $J$.
    This means that there exists~$\delta > 0$ such that
    \[
        \forall \vect x \in B_{\delta}(\vect x_*) \cap \mathcal K, \qquad
        J(\vect x_*) \leq J(\vect x).
    \]
    Therefore $J(\vect x_*) \leq J\bigl((1-t)\vect x_* + t \vect x\bigr)$ for all $t \in [0, 1]$ sufficiently small.
    But then
    \[
        \ip{\nabla J(\vect x_*), \vect x - \vect x_*}
        = \lim_{t \to 0} \frac{ J\bigl((1-t)\vect x_* + t \vect x\bigr)  - J(\vect x_*)}{t} \geq 0.
        % 0 \leq \frac{\d}{\d t} \Bigl( J\bigl((1-t)\vect x_* + t \vect x\bigr) \Bigr) \Big\vert_{t=0} = \ip{\nabla J(\vect x_*), \vect y - \vect x_*}.
    \]
    Conversely, suppose that~\eqref{eq:euler_lagrange} is satisfied and that~$J$ is convex.
    Since~$J$ is convex, equation~\eqref{eq:strong_convex_eq1} holds with $\alpha = 0$,
    and applying this equation with~$\vect y = \vect x_*$,
    we deduce that $\vect x_*$ is a global minimizer.
\end{proof}

The steepest descent~\cref{algo:steepest_descent_method_optimization} can be extended to a class of optimization problems with constraints by introducing an additional projection step.
In order to precisely formulate the algorithm,
we begin by introducing the projection operator~$\Pi_{\mathcal K}$.

\begin{proposition}
    [Projection on a closed convex set]
    Suppose that $\mathcal K$ is a closed convex subset of~$\real^n$.
    Then for all $\vect x \in \real^n$ there a unique $\Pi_{\mathcal K} \vect x \in \mathcal K$,
    called the orthogonal projection of~$\vect x$ onto~$\mathcal K$,
    such that
    \[
        \norm{\Pi_{\mathcal K} \vect x - \vect x} = \inf_{\vect y \in \mathcal K} \norm{\vect y - \vect x}.
    \]
\end{proposition}
\begin{proof}
    The functional $J_{\vect x}(\vect y) = \norm{\vect y - \vect x}^2$ is strongly convex,
    and so~\cref{proposition:well-posedness_constrained} immediately implies the existence and uniqueness of~$\Pi_{\mathcal K} \vect  x$.
\end{proof}

\begin{remark}
    In view of~\cref{theorem:euler_lagrange},
    the projection~$\Pi_{\mathcal K} \vect x$ is the unique element of~$\mathcal K$ which satisfies
    \begin{equation}
        \label{eq:euler_lagrange_projection}
        \forall \vect y \in \mathcal K,
        \qquad \ip{\Pi_{\mathcal K} \vect x - \vect x, \vect y - \Pi_{\mathcal K}\vect x} \geq 0.
    \end{equation}
\end{remark}

We are now ready to present the steepest descent method with projection:
see~\cref{algo:steepest_descent_method_optimization_projection}.
Like~\cref{algo:steepest_descent_method_optimization},
the steepest descent with projection may be viewed as a fixed point iteration,
this time for the function
\begin{equation}
    \label{eq:constrained_fixed_point}
    \vect F_{\lambda}(\vect x) := \Pi_{\mathcal K}\bigl(\vect x - \lambda \nabla J(\vect x)\bigr)
\end{equation}
Under the assumption that~$J$ is convex,
a fixed point of this function is a solution of the constrained optimization problem~\eqref{eq:optimization}.
Indeed if~$\vect F_{\lambda}(\vect x_*) = \vect x_*$,
then $\vect x_* \in \mathcal K$ by definition of~$\Pi_{\mathcal K}$ and, by~\eqref{eq:euler_lagrange_projection},
\[
    \forall \vect y \in \mathcal K,
    \qquad \ip{\lambda \nabla J(\vect x_*), \vect y - \vect x_*} \geq 0.
\]
Therefore, using~\cref{theorem:euler_lagrange} and assuming that~$J$ is convex,
we obtain that $\vect x_*$ is a global minimizer of~$J$.
We now prove the convergence of the method.
\begin{algorithm}
\caption{Steepest descent with projection}%
\label{algo:steepest_descent_method_optimization_projection}%
\begin{algorithmic}[1]
    \State Pick $\lambda$, and initial $\vect x_0$.
    \For {$k \in \{0, 1, \dotsc\}$}
        \State $\vect d_k \gets \nabla J(\vect x_k)$
        \State $\vect x_{k+1} \gets \Pi_{\mathcal K}(\vect x_{k} - \lambda \vect d_k)$
    \EndFor
\end{algorithmic}
\end{algorithm}

\begin{theorem}
    [Convergence of steepest descent with projection]
    \label{theorem:convergence_steepest_descent_optimization_constraint}
    Suppose that $J$ is differentiable, strongly convex with parameter~$\alpha$,
    and that its gradient $\nabla J \colon \real^n \to \real^n$ is Lipschitz continuous with parameter~$L$.
    Assume also that $\mathcal K \subset \real^n$ is closed and convex.
    Then provided that
    \begin{equation}
        \label{eq:condition_time_step_constroined}
        0 < \lambda < \frac{2 \alpha}{L},
    \end{equation}
    the steepest descent method with fixed step is convergent.
    More precisely,
    there exists $\rho \in (0, 1)$ such that for all~$k \geq 0$
    \begin{equation*}
        \norm{\vect x_k - \vect x_*} \leq \rho^k \norm{\vect x_0 - \vect x_*}.
    \end{equation*}
\end{theorem}
\begin{proof}
    We already showed in the proof of~\cref{theorem:convergence_steepest_descent_optimization} that the mapping $\vect x \mapsto \vect x - \lambda \nabla J(\vect x)$ is a contraction if and only if~$\lambda$ satisfies~\eqref{eq:condition_time_step_constroined}.
    In order to prove that~$\vect F_{\lambda}$ given in~\eqref{eq:constrained_fixed_point} is a contraction,
    it is sufficient to prove that $\Pi_{\mathcal K} \colon \real^n \to \mathcal K$ satisfies the following estimate:
    \[
        \forall (\vect x, \vect y) \in \real^n \times \real^n, \qquad
        \norm{\Pi_{\mathcal K} \vect x - \Pi_{\mathcal K} \vect y} \leq \norm{\vect x - \vect y}.
    \]
    To this end, take $(\vect x, \vect y) \in \real^n \times \real^n$ and let $\vect \delta = \Pi_{\mathcal K} \vect x - \Pi_{\mathcal K} \vect y$.
    By~\eqref{eq:euler_lagrange_projection},
    it holds that
    \begin{align*}
        \norm{\vect \delta}^2
        &= \ip{\vect \delta, \Pi_{\mathcal K} \vect x - \vect x} + \ip{\vect \delta, \vect x - \vect y} + \ip{\vect \delta, \vect y - \Pi_{\mathcal K} \vect y} \\
        &\leq 0 + \ip{\vect \delta, \vect x - \vect y} + 0 \leq \norm{\vect \delta} \norm{\vect x - \vect y},
    \end{align*}
    which enables to conclude.
\end{proof}
