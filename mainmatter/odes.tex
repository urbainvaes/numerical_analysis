\chapter{Numerical ordinary differential equations}
\label{cha:odep}
\minitoc

\section*{Introduction}
This chapter concerns the numerical solution of ordinary differential equations (ODEs) of the following form:
\begin{equation}
    \label{eq:ode}
    \left \{
    \begin{aligned}
        & \vect x'(t) = \vect f\bigl(t, \vect x(t)\bigr), \\
        & \vect x(t_0) = \vect x_0.
    \end{aligned}
    \right.
\end{equation}
Here $\vect f\colon \real \times \real^n \to \real$ and $\vect x_0$ is the initial condition.
Equations of this type
are the building blocks of a plethora mathematical models in science and engineering.
They have applications in celestial dynamics, molecular simulation and fluid mechanics, to mention just a few.
Ordinary differential equations also arise after discretization of time-dependent partial differential equations,
which are also ubiquitous in science.
More often than not,
it is not possible to find an explicit solution of~\eqref{eq:ode},
and so one has to resort to numerical simulation.
The rest of chapter is organized as follows:
\begin{itemize}
    \item
        In \cref{sec:existence},
        we define precisely the notions of local and global solutions for the continuous-time problem~\eqref{eq:ode},
        and we recall fundamental results concerning the existence and uniqueness of a solution.

    \item
        In~\cref{sec:one-step_methods},
        we analyze the so-called one-step numerical methods to solve~\eqref{eq:ode}.
\end{itemize}

\section{Analysis of the continuous problem}
A differentiable function $\vect x\colon I \to \real^n$,
where $I$ denotes an interval of $\real$ containing $t_0$,
is a solution of~\eqref{eq:ode} if $\vect x(t_0) = \vect x_0$ and the equation~\eqref{eq:ode} is satisfied for all~$t \in I$.
The solution is called global if~$I = \real$,
and local otherwise.

\paragraph{Integral formulation.}
If $\vect x$ is a solution to~\eqref{eq:ode},
then it holds that
\begin{equation}
    \label{eq:ode_integral}
    \forall  t \in I, \qquad
    \vect x(t) = \vect x_0 + \int_{t_0}^{t} \vect f\bigl(s, \vect x(s)\bigr) \, \d s.
\end{equation}
The converse statement is not true in general,
because a solution to~\eqref{eq:ode_integral} need not necessarily be differentiable everywhere.
However, if the integral formulation~\eqref{eq:ode_integral} holds,
then necessarily $\vect x$ is absolutely continuous
and~\eqref{eq:ode} is satisfied for almost every~$t$.
Additionally, if~\eqref{eq:ode_integral} is satisfied and the function~$\vect f$ is continuous,
then the function~$s \mapsto \vect f\bigl(s, \vect x(s)\bigr)$ is continuous,
and so~\eqref{eq:ode} is satisfied for all $t \in I$ by the fundamental theorem of analysis.

\label{sec:existence}
\begin{theorem}
    [Existence of a solution]
    \label{theorem:existence}
    Let $\vect x_0 \in \real^n$ and let $\Omega_{\mathcal T, \mathcal R}$ denote the set
    \[
        \bigl\{(t, \vect x) \in \real \times \real^n: \abs{t-t_0} \leq \mathcal T \text{ and } \vecnorm{\vect x - \vect x_0} \leq \mathcal R \bigr\},
    \]
    Assume that the following conditions are satisfied:
    \begin{itemize}
        \item
            The function $\vect f$ is uniformly bounded on~$\Omega_{\mathcal T, \mathcal R}$:
            \begin{equation}
                \label{eq:boundedness_ode}
                \forall (t, \vect x) \in \Omega_{\mathcal T, \mathcal R}, \qquad
                \vecnorm{\vect f(t, \vect x)} \leq M.
            \end{equation}

        \item
            The function $\vect f$ satisfies the following Lipschitz condition:
            there is $L > 0$ such that
            \begin{equation}
                \label{eq:local_lipschitz_ode}
                \forall \bigl((t, \vect x_1), (t, \vect x_2)\bigr) \in \Omega_{\mathcal T, \mathcal R} \times \Omega_{\mathcal T, \mathcal R}, \qquad
                \vecnorm{ \vect f(t, \vect x_1) - \vect f(t, \vect x_2) } \leq L \vecnorm{\vect x_1 - \vect x_2}.
            \end{equation}
    \end{itemize}
    Then there exists $T \in (0, \mathcal T]$ depending on $\mathcal R$, $M$ and $L$ such that
    the differential equation~\eqref{eq:ode_integral} has a local solution $\vect x\colon [t_0 - T, t_0 + T] \to \real^n$.
\end{theorem}
\begin{proof}
    Fix $T \in (0, \mathcal T]$ and let $I = [t_0 - T, t_0 + T]$.
    Let also $\mathcal X$ denote the following subset of continuous functions defined from~$I$ to $\real^n$:
    \[
    \mathcal X := \left\{ \vect x\in C\left(I, \real^n\right) : \sup_{t \in I} \norm[\big]{\vect x(t) - \vect x_0} \leq \mathcal R \right\}
    \]
    The set~$\mathcal X$ endowed with supremum metric is a closed subset of~$C(I, \real^n)$.
    Since $\mathcal X$ is a closed subset of a complete metric space,
    it is itself complete.
    Let $\Phi \colon \mathcal X \to C(I, \real^n)$ denote the mapping
    \[
        \Phi(\vect x) \colon  t \mapsto  \vect x_0 + \int_{0}^{t} \vect f\bigl(s, \vect x(s)\bigr) \, \d s.
    \]
    The right-hand side, being the integral of a bounded function,
    is indeed a continuous function.
    We will show that, for~$T$ sufficiently small,
    \begin{itemize}
        \item the mapping $\Phi$ maps $\mathcal X$ into $\mathcal X$;
        \item the mapping $\Phi$ is a contraction.
    \end{itemize}
    From~\eqref{eq:boundedness_ode},
    it follows that
    \[
        \forall \vect x \in \mathcal X, \qquad
        \forall t \in I, \qquad
        \vecnorm*{\Phi(\vect x)(t) - \vect x_0} = \vecnorm*{ \int_{t_0}^{t} \vect f\bigl(s, \vect x(s)\bigr) \, \d s }
        \leq M T.
    \]
    On the other hand, from the Lipschitz condition~\eqref{eq:local_lipschitz_ode},
    it holds that
    \[
        \forall (\vect x, \vect y) \in \mathcal X \times \mathcal X, \qquad
        \norm{\Phi(\vect x) - \Phi(\vect y)} \leq 2 L T.
    \]
    Therefore, it suffices to take $T \leq \min \left\{ \mathcal T, \frac{\mathcal R}{M}, \frac{2}{L} \right\}$ to ensure that the above conditions are satisfied.
    For a value of~$T$ in this range,
    the Banach fixed point theorem, \cref{theorem:banach_fixed_point},
    gives the existence of a unique fixed point~$\vect x_* \in \mathcal X$ of~$\Phi$.
    Since a fixed point of~$\Phi$ is a solution to~\eqref{eq:ode_integral},
    the statement is proved.
\end{proof}
It may seem at first glance that uniqueness of the solution to~\eqref{eq:ode_integral} follows from the uniqueness of the fixed point guaranteed by~\cref{theorem:banach_fixed_point}.
However, this theorem implies uniqueness \emph{only in the set~$\mathcal X$},
a property known as~\emph{conditional uniqueness}.
In order to prove that the solution is unique over the full space $C([t_0-T, t_0+T], \real^n)$,
additional arguments are required.
A simple approach is to rely on Gr\"onwall's lemma.
\begin{lemma}
    [Gr\"onwall's lemma, simplified integral form]
    Suppose that $u \colon [t_0-T, t_0+T] \to \real_{\geq 0}$ is continuous, nonnegative, and satisfies
    \begin{equation}
        \label{eq:gronwall_condition}
        \forall t \in [t_0, t_0+T], \qquad
        u(t) \leq \alpha + \int_{t_0}^{t} \beta(s) \, u(s) \, \d s,
    \end{equation}
    where $\alpha \geq 0$ and $\beta\colon [t_0, t_0 + T] \to \real_{\geq 0}$ is continuous and nonnegative.
    Then
    \begin{equation}
        \label{eq:gronwall}
        \forall t \in [t_0, t_0+T], \qquad
        u(t) \leq \alpha \exp \left( \int_{t_0}^{t} \beta(s) \, \d s \right).
    \end{equation}
\end{lemma}
\begin{proof}
    Assume first that~$\alpha > 0$,
    so that the logarithm in~\eqref{eq:intermediate_gronwall} is well-defined.
    By the fundamental theorem of calculus and~\eqref{eq:gronwall_condition},
    it holds that
    \[
        \frac{\d}{\d t} \left( \alpha + \int_{t_0}^{t} \beta(s) \, u(s) \, \d s \right)
        \leq \beta(t) \left( \alpha + \int_{t_0}^{t} \beta(s) \, u(s) \, \d s \right)
    \]
    Therefore we have
    \begin{equation}
        \label{eq:intermediate_gronwall}
        \frac{\d}{\d t} \log \left( \alpha + \int_{t_0}^{t} \beta(s) \, u(s) \, \d s \right) \leq \beta(t),
    \end{equation}
    and after integrating and exponentiating,
    we obtain
    \[
        \alpha + \int_{t_0}^{t} \beta(s) \, u(s) \, \d s
        \leq \alpha \exp \left( \int_{t_0}^{t} \beta(s) \, \d s \right)
    \]
    The statement then follows by using~\eqref{eq:gronwall_condition} again.
    If~\eqref{eq:gronwall_condition} is satisfied for~$\alpha = 0$,
    then this condition is also satisfied for all $\alpha > 0$.
    Therefore~\eqref{eq:gronwall} also holds for all~$\alpha > 0$,
    and taking the limit $\alpha \to 0$ in this equation,
    we obtain the statement.
\end{proof}
Note that the estimate~\eqref{eq:gronwall} is sharp,
since the function $v \colon [t_0, t_0 + T]$ given by
\[
    v(t) = \alpha \exp \left( \int_{t_0}^{t} \beta(s) \, \d s \right)
\]
satisfies~\eqref{eq:gronwall_condition} with equality.
We are now ready to prove uniqueness.
\begin{theorem}
    [Uniqueness of the solution]
    \label{theorem:uniqueness}
    Let $\vect x_0 \in \real^n$ and let
    \[
        \Omega_{\mathcal T, \mathcal R}
        \bigl\{(t, \vect x) \in \real \times \real^n: \abs{t - t_0} \leq \mathcal T \text{ and } \vecnorm{\vect x - \vect x_0} \leq \mathcal R \bigr\},
    \]
    Assume that for all $\mathcal T \in \real_{>0}$ and $\mathcal R \in \real_{>0}$,
    there is $L_{\mathcal T,  \mathcal R}$ such that
    \begin{equation}
        \label{eq:local_lipschitz_uniqueness}
        \forall \bigl((t, \vect x_1), (t, \vect x_2)\bigr) \in \Omega_{\mathcal T, \mathcal R} \times \Omega_{\mathcal T, \mathcal R}, \qquad
        \vecnorm{ \vect f(t, \vect x_1) - \vect f(t, \vect x_2) } \leq L_{\mathcal T, \mathcal R} \vecnorm{\vect x_1 - \vect x_2}.
    \end{equation}
    Then if $\vect x_1$ and $\vect x_2$ in $C\bigl([t_0 - T, t_0 + T], \real^n\bigr)$ are local solutions to~\eqref{eq:ode_integral},
    it holds that $\vect x_1 = \vect x_2$.
\end{theorem}
\begin{proof}
    Let~$I = [t_0 - T, t_0 + T]$ and
    \[
        R := \max \left\{ \sup_{t \in I} \norm{\vect x_1(t) - \vect x_0},   \sup_{t \in I} \norm{\vect x_1(t) - \vect x_0} \right\} < \infty,
    \]
    If $\vect x_1$ and $\vect x_2$ are solutions,
    then
    \[
        \forall t \in [t_0 - T, t_0 + T], \qquad
        \vect x_1(t) - \vect x_2(t) = \int_{t_0}^{t} \Bigl( f\bigl(s, \vect x_1(s)\bigr) - f\bigl(s, \vect x_2(s)\bigr) \Bigr)  \, \d s.
    \]
    Taking the norm and using~\eqref{eq:local_lipschitz_uniqueness},
    we obtain
    \[
        \forall t \in [t_0, t_0 + T], \qquad
        \vecnorm{\vect x_1(t) - \vect x_2(t)} \leq L_{T,R} \int_{t_0}^{t} \vecnorm{\vect x_1(s) - \vect x_2(s)} \, \d s
    \]
    Using Gr\"onwall's lemma,
    we deduce that~$\vect x_1(t) = \vect x_2(t)$ for all $t \in [t_0, t_0 + T]$.
    A similar argument can be employed to show that $\vect x_1 = \vect x_2$ on $[t_0 - T, t_0]$.
\end{proof}

\begin{corollary}
    [Maximal solutions]
    \label{corollary:maximal_solutions}
    Assume that $\vect f$ is continuous in~$t$ and satisfies the local Lipschitz condition~\eqref{eq:local_lipschitz_uniqueness}.
    Then there exists $0 \leq T_- < T_+ \leq \infty$ such that $t_0 \in (T_-, T_+)$ and
    the following properties are satisfied.
    \begin{itemize}
        \item
            there exists a solution $\vect x_* \colon (T_-, T_+) \to \real^n$ to~\eqref{eq:ode_integral};

        \item
            if $\vect x\colon I \to \real^n$ is a local solution of~\eqref{eq:ode_integral},
            then $I \subset (T_-, T_+)$ and $\vect x(t) = \vect x_*(t)$ for all $t \in I$.

        \item
            If $T_+$ is finite, then $\lim_{t \to T_+} \vecnorm[\big]{\vect x(t)} = \infty$,
            and if $T_-$ is finite,
            then $\lim_{t \to T_-} \vecnorm[\big]{\vect x(t)} = \infty$.
    \end{itemize}
    The solution $\vect x_*$ is called the maximal solution of~\eqref{eq:ode_integral}.
\end{corollary}
\begin{proof}
    Let $\mathcal I$ denote the union of all the open intervals~$I$ such that there exists a solution in~$C(I, \real^n)$ to~\eqref{eq:ode_integral}.
    The open set $\mathcal I$ is connected and, by~\cref{theorem:existence},
    it contains a neighborhood of~$t_0$.
    Therefore $\mathcal I$ is of the form $(T_-, T_+)$,
    where $0 \leq T_- <  t_0 < T_+ \leq \infty$.
    In view of~\cref{theorem:uniqueness},
    all the local solutions coincide where they are defined,
    and so they can be patched together in order to obtain a solution~$\vect x_*\colon (T_-, T_+) \to \real$.
    It remains to prove the third item.
    To this end, suppose for contradiction that $T_+$ was finite and that
    there was $(t_n)_{n \in \nat}$ such that $t_n \to T_+$ and~
    \[
        K := \sup_{n \in \nat} \vecnorm{\vect x_*(t_n)} < \infty.
    \]
    Since $\vect f$ is continuous,
    there is $M$ such that $\lvert f(t, \vect x) \rvert$ is uniformly bounded from above by~$M$ for all $(t, \vect x) \in [T_- - 1, T_+ + 1] \times B_{K+1}(\vect 0)$.
    Furthermore, by the assumption~\eqref{eq:local_lipschitz_uniqueness},
    there is $L$ such that for all $t \in [T_- - 1, T_+ + 1]$,
    the following Lipschitz condition holds:
    \[
        \forall (\vect x_1, \vect x_2) \in  \times B_{K+1}(\vect 0) \times B_{K+1}(\vect 0), \qquad
        \vecnorm{ \vect f(t, \vect x_1) - \vect f(t, \vect x_2) } \leq L \vecnorm{\vect x_1 - \vect x_2}.
    \]
    Consequently, \cref{theorem:existence} with $\mathcal T = \mathcal R = 1$ implies for all~$n$ the existence of a solution to
    \[
        \left \{
        \begin{aligned}
            & \vect x'(t) = \vect f\bigl(t, \vect x(t)\bigr), \\
            & \vect x(t_n) = \vect x_*(t_n).
        \end{aligned}
        \right.
    \]
    over the time interval $[t_n - T, t_n + T]$, where $T > 0$ depends only on $M$ and $L$, and not on~$n$.
    But then, for~$n$ sufficiently large,
    this solution extends beyond $T_+$,
    which contradicts the maximality of~$\mathcal I$.
    An analogous reasoning can be employed for~$T_-$.
\end{proof}
\begin{example}
    Consider the ODE
    \[
        \left \{
        \begin{aligned}
            & x'(t) = x(t)^2, \\
            & x(0) = 1.
        \end{aligned}
        \right.
    \]
    The maximal solution is $x_* \colon (-\infty, 1) \to \real$ given by
    \[
        \vect x_*(t) = \sqrt{\frac{1}{1-t}}.
    \]
\end{example}

In certain settings,
it is possible to prove the maximal solution to~\eqref{eq:ode_integral} is globally defined for any initial condition.
We discuss a few important examples.
\begin{itemize}
    \item
        The first case is when $\vect f\colon \real \times \real^n$ is globally Lipschitz in its second argument,
        with a Lipschitz constant that depends continuously on the first argument.

    \item
        The second case,
        generalizing the first,
        is when the growth of~$\vect f(t, \placeholder)$ is at most affine:
        \[
            \forall (t, \vect x) \in \real \times \real^n, \qquad
            \vecnorm{ \vect f(t, \vect x) } \leq C(t) + L(t) \vecnorm{ \vect x },
        \]
        with continuous constants~$C(t)$ and $L(t)$.

    \item
        The third case is when $\vect f$ is independent of~$t$ and there is a function~$W \in C^1(\real^n)$ such that~$W(\vect x) \to \infty$ in the limit as $\vecnorm{\vect x} \to \infty$
        and
        \[
            \forall \vect x \in \real^n, \qquad
            \nabla W(\vect x) \cdot \vect f(\vect x) \leq c < \infty
        \]
        Such a function is called a \emph{Lyapunov function}.
\end{itemize}
The strategy of proof for global existence usually relies on an argument by contradiction.
Consider for example the third setting.
Since the assumptions of~\cref{corollary:maximal_solutions} are satisfied,
there exists a maximal solution $\vect x_*\colon (T_-, T_+) \to \infty$.
Assume for contradiction that~$T_+$ is finite.
Then the third item in~\cref{corollary:maximal_solutions} implies that $\lim_{t \to T_+} \vecnorm{\vect x_*(t)} \to \infty$,
and so~$W\bigl(\vect x_*(t)\bigr)$ blows up as~$t$ approaches $T^+$.
On the other hand, we have
\[
    \frac{\d}{\d t} W\bigl(\vect x_*(t)\bigr) = \nabla W\bigl(\vect x_*(t)\bigr) \cdot \vect f\bigl(\vect x_*(t)\bigr) \leq c.
\]
Therefore $\lim_{t \to T_+} W\bigl(\vect x_*(t)\bigr) \leq W\bigl(\vect x_*(t_0)\bigr) + \lvert c \rvert (T_+ - t_0)$,
which is a contradiction.

% \paragraph{Lyapunov stability.}
% Before we move to the study of numerical schemes for SDEs,
% we introduce one last important concept: that of \emph{Lyapunov stability}.
% To precisely define this concept,
% Suppose that the assumptions of~\cref{corollary:maximal_solutions} are satisfied and
% consider the differential equation~\eqref{eq:ode} over an interval~$[-T, T]$,
% with $T_- < -T < T < T_+$, and the following perturbation thereof:
% \begin{equation}
%     \label{eq:ode_perturbed}
%     \left \{
%     \begin{aligned}
%         & \dot {\vect x_{\delta}}(t) = \vect f\bigl(t, \vect x(t)\bigr) + \vect \delta\bigl(t, \vect x(t)\bigr), \\
%         & \vect x_{\delta}(t_0) = \vect x_0 + \vect \delta_0.
%     \end{aligned}
%     \right.
% \end{equation}
% Roughly speaking, Lyapunov stability is a property of the differential equation which
% guarantees that the solution to the perturbed equation~\eqref{eq:ode_perturbed} remains close to that of the unperturbed equation~\eqref{eq:ode}
% when the perturbation is small.
% \begin{definition}
%     [Stability in the sense of Lyapunov]
%     Equation~\eqref{eq:ode} is stable in the sense of Lyapunov if there is $C = C(T)$ such that
%     for all~$\varepsilon > 0$ and all $\vect \delta \colon \real \times \real^n \to \real$ continuous and bounded and all $\vect \delta_0$ with
%     \[
%         \sup_{(t, \vect x) \in [-T, T] \times \real^d} \vecnorm{ \vect \delta(t, \vect x)} \leq \varepsilon
%         \quad \text{ and } \quad
%         \vecnorm{\vect \delta_0} \leq \varepsilon
%     \]
% \end{definition}

\section{One-step methods}
\label{sec:one-step_methods}
From now on, we assume for simplicity that $t_0 = 0$ and that
the initial value problem~\eqref{eq:ode} admits a unique solution over the interval $[0, T]$.
Most numerical methods for ODEs construct an approximation of the solution at discrete points:
\[
    \vect x_n \approx \vect x(t_n), \qquad n = 0, 1, 2, \dotsc.
\]
The discretization points $(t_n)_{n \in \nat}$ are commonly equidistant,
i.e.~$t_n = n \Delta$ where $\Delta$ is the \emph{discretization step}.
Sometimes, it is useful to employ a variable time step,
but we assume throughout this section that the time step is fixed, for simplicity.
We begin in~\cref{sub:explicit_euler} and~\cref{sub:implicit_euler} by studying the simplest one-step methods,
namely the forward and backward Euler methods.
Then, in~\cref{sub:one_step_general},
we present a general approach to the analysis of one-step methods.
Finally, in~\cref{sub:other_one_step},
we present other widely used one-step methods in applications.

\subsection{Forward Euler method}
\label{sub:explicit_euler}
Assume that~\eqref{eq:ode} has a unique solution~$\vect x(t)$ over the interval $[0, T]$.
If $\vect x(t)$ is twice continuously differentiable,
then by Taylor's formula,
we have
\begin{equation}
    \label{eq:taylor_forward}
    \vect x(t + \Delta) = \vect x(t) + \Delta \vect f(t, \vect x) + \frac{\Delta^2}{2} \vect x''(\tau),
    \qquad \tau \in (t, t+\Delta).
\end{equation}
This motivates a method known as the \emph{forward} or \emph{explicit} Euler method:
\[
    \vect x_{n+1} = \vect x_{n} + \Delta \vect f(t_n, \vect x_n),
\]
with the same initial condition as for the continuous equation~\eqref{eq:ode}.
The convergence of this method can be proved under a global Lipschitz assumption on the function~$\vect f$.

\begin{theorem}
    [Convergence of the forward Euler method]
    \label{theorem:forward_euler}
    Assume that there is $L \in \real_{>0}$ such that
    \begin{equation}
        \label{eq:global_lipschitz}
        \forall (t, \vect x, \vect y) \in \real \times \real^n \times \real^n, \qquad
        \vecnorm{\vect f(t, \vect x) - \vect f(t, \vect y)}
        \leq L \vecnorm{\vect x - \vect y}.
    \end{equation}
    Suppose in addition that there exists a unique, twice continuously differentiable of~\eqref{eq:ode} over the interval~$[0, T]$,
    and let
    \[
        M = \sup_{t \in [0,T]} \vecnorm{\vect x''(t)}
    \]
    Then the following error estimate holds:
    \begin{equation}
        \label{eq:error_bound_forward_euler}
        \forall n \in \left\{0, 1, \dotsc, \floor*{\frac{T}{\Delta}} \right\},
        \qquad
        \vecnorm{\vect x(t_n) - \vect x_n}
        \leq
        \frac{\Delta M}{2} \left( \frac{\e^{L t_n} - 1}{L} \right).
    \end{equation}
\end{theorem}
\begin{proof}
    Notice that
    \begin{align*}
        \vect x(t_{n}) - \vect x_{n}
        &= \left( \vect x(t_{n-1}) + \Delta \, \vect f\bigl(t_{n-1}, \vect x(t_{n-1})\bigr) + \frac{\Delta^2}{2} \vect x''(\tau_n) \right)
        - \Bigl( \vect x_n + \Delta \, f\bigl(t_{n-1}, \vect x_{n-1}\bigr) \Bigr) \\
        &= \bigl(\vect x(t_{n-1}) - \vect x_{n-1}\bigr) + \Delta \Bigl( f\bigl(t_{n-1}, \vect x(t_{n-1})\bigr) - f\bigl(t_{n-1}, \vect x_{n-1}\bigr) \Bigr) + \frac{\Delta^2}{2} \vect x''(\tau_n),
    \end{align*}
    for some $\tau_n \in (t_{n-1}, t_n)$.
    Let $\vect e_n = \vect x(t_n) - \vect x_n$ and $\vect \varepsilon_n = \frac{\Delta^2}{2} \vect x''(\tau_n)$.
    The first term is the error at iteration~$n$,
    and the second may be bounded from~\eqref{eq:global_lipschitz},
    which gives
    \begin{align*}
        \vecnorm{\vect e_{n}}
        &\leq (1 + \Delta L) \vecnorm{\vect e_{n-1}}
        + \vecnorm{\vect \varepsilon_n}.
    \end{align*}
    The structure of this equation is important,
    as it appears in the analysis of all one-step methods for ODEs.
    The first term is an amplification of the error at the previous iteration,
    and the second term is an upper bound on the additional error introduced at step~$n$.
    Applying this inequality to the previous time steps, we obtain
    \begin{align}
        \nonumber
        \vecnorm{\vect e_{n}}
        &\leq (1 + \Delta L) \Bigl( (1 + \Delta L) \vecnorm{\vect e_{n-2}} + \vecnorm{\vect \varepsilon_{n-1}} \Bigr) + \vecnorm{\vect \varepsilon_{n}} \\
        \label{eq:sum_local_and_accumulation}
        &\leq \dotsc
        \leq (1 + \Delta L)^{n} \vecnorm{\vect e_{0}} + \sum_{i=1}^{n} (1 + \Delta L)^{n-i} \vecnorm{\vect \varepsilon_{i}}.
    \end{align}
    Since $\vecnorm{\vect \varepsilon_i} \leq \Delta^2M/2$,
    we have by using the formula for geometric series that
    \[
        \vecnorm{\vect e_{n}}
        \leq (1 + \Delta L)^{n} \vecnorm{\vect e_{0}} + \frac{(1+\Delta L)^{n} - 1}{\Delta L} \left( \frac{\Delta^2 M}{2} \right).
    \]
    The first term is zero because $\vecnorm{\vect e_{0}} = 0$.
    Using the bound $(1+\Delta L)^n \leq \bigl(\exp(\Delta L)\bigr)^n = \e^{L t_n}$ in the second term and rearranging,
    we finally obtain the statement~\eqref{eq:error_bound_forward_euler}.
\end{proof}

\subsection{Backward Euler method}
\label{sub:implicit_euler}
If we apply the Taylor expansion~\eqref{eq:taylor_forward} backward around~$t + \Delta$,
instead of forward around~$t$,
then we obtain
\[
    \label{eq:taylor_backward}
    \vect x(t) = \vect x(t + \Delta) - \Delta \vect f(t + \Delta, \vect x) + \frac{\Delta^2}{2} \vect x''(\tau),
    \qquad \tau \in (t, t+\Delta).
\]
This motivates the so-called \emph{backward} or \emph{implicit} Euler method:
\begin{equation}
    \label{eq:backward_euler}
    \vect x_{n+1} = \vect x_{n} + \Delta \vect f(t_{n+1}, \vect x_{n+1}).
\end{equation}
Observe that the right-hand side depends on $\vect x_{n+1}$.
Therefore, given $t_n$ and $\vect x_n$,
this is a nonlinear equation for the unknown $\vect x_{n+1}$,
which can be solved using any of the methods studied in~\cref{cha:solution_of_nonlinear_systems}.
Finding a solution to~\eqref{eq:backward_euler} amounts to finding a fixed point of the function
\[
    \vect y \mapsto \vect F(\vect y) := \vect x_{n} + \Delta \vect f(t_n, \vect y)
\]
A priori, the existence and uniqueness of such a fixed point is not guaranteed.
We proved in~\cref{theorem:exponential_convergence_fixed_point} that a sufficient condition for these properties to hold is
that~$\vect F$ is globally Lipschitz with a constant strictly less than 1,
which holds if and only if the function~$\vect y \mapsto \vect f(t_n, \vect y)$ is globally Lipschitz with a constant strictly less than $1/\Delta$.
If the condition~\eqref{eq:global_lipschitz} holds, for example,
then the backward Euler method~\eqref{eq:backward_euler} is guaranteed to be well defined for $\Delta < \frac{1}{L}$.
\Cref{theorem:exponential_convergence_fixed_point} also ensures that, if~$\vect F$ is globally Lipschitz with a constant less than~$1$,
then the fixed point can be approximated by using the iteration
\begin{equation}
    \label{eq:fixed_point_ode}
    \vect y_{k+1} = \vect F(\vect y_k).
\end{equation}
and there is exponential convergence~$\vect y_k \to \vect x_{n+1}$ in the limit as $k \to \infty$.
A natural starting point for~\eqref{eq:fixed_point_ode} is~$\vect y_0 = \vect x_n$.
An alternative approach to~\eqref{eq:fixed_point_ode} is to use Newton--Raphson's method,
which is faster in principle but must be initialized sufficiently close to the fixed point.

Using a reasoning similar to that employed for proving~\cref{theorem:forward_euler},
we can prove the following result.
\begin{theorem}
    [Convergence of the backward Euler method]
    If the assumptions of~\cref{theorem:forward_euler} hold and $\Delta < \frac{1}{L}$,
    then the following error estimate holds:
    \begin{equation}
        \label{eq:estimate_backward_euler}
        \forall n \in \left\{0, 1, \dotsc, \floor*{\frac{T}{\Delta}} \right\},
        \qquad
        \vecnorm{\vect x(t_n) - \vect x_n}
        \leq
        \frac{\Delta M}{2} \left( \frac{\left( \frac{1}{1 - \Delta L} \right)^n - 1}{L} \right).
    \end{equation}
\end{theorem}
\begin{proof}
    The proof is left as an exercise.
\end{proof}
\begin{remark}
    Note that, if $\Delta < \frac{1}{2L}$,
    then
    \begin{align*}
         \frac{1}{1 - \Delta L}
         &= 1 + \Delta L + (\Delta L)^2 + (\Delta L)^3 + (\Delta L)^4 \dotsc \\
         &\leq 1 + \Delta L + (\Delta L)^2 + \frac{1}{2} (\Delta L)^2 + \frac{1}{4} (\Delta L)^2 + \dotsc \\
         &\leq 1 + \Delta L + 2 (\Delta L)^2 \leq \exp\bigl(\Delta L + (\Delta L)^2\bigr),
     \end{align*}
     and so the error estimate~\eqref{eq:estimate_backward_euler} gives
     \[
        \vecnorm{\vect x(t_n) - \vect x_n}
        \leq
        \frac{\Delta M}{2} \left( \frac{\exp(L t_n + \Delta L^2 t_n) - 1}{L} \right),
     \]
     which makes it clear that the right-hand side of~\eqref{eq:estimate_backward_euler} is close,
     in absolute and relative terms, to that of~\eqref{eq:error_bound_forward_euler} when $\Delta \ll 1$.
\end{remark}

\subsection{Analysis of general one-step methods}
\label{sub:one_step_general}
In general, one-step methods to solve differential equations are of the abstract form
\begin{equation}
    \label{eq:general_one_step}
    \vect x_{n+1} = \vect x_n + \Delta \vect \Phi_{\Delta}(t_n, \vect x_n).
\end{equation}
where $\vect \Phi_{\Delta}\colon \real \times \real^n \to \real^n$ is a function such that
\begin{equation}
    \label{eq:one_step_integral}
    \vect \Phi_{\Delta}(t, \vect x)
    \approx \frac{1}{\Delta} \int_{t}^{t+\Delta} \vect f\bigl(s, \vect x^{t, \vect x}(s)\bigr) \, \d s
    = \frac{\vect x^{t, \vect x}(t + \Delta) - \vect x}{\Delta}.
\end{equation}
Here $\vect x^{t, \vect x}$ denotes the solution to the differential equation~\eqref{eq:ode} with initial condition $\vect x(s) = \vect x$.
The main goal of this section is to establish general conditions,
known as \emph{consistency} and \emph{stability},
under which the numerical scheme~\eqref{eq:general_one_step} is convergent.
As we observed in the proof of~\cref{theorem:forward_euler}
-- specifically in equation~\eqref{eq:sum_local_and_accumulation} --
the error at the final iteration for the forward Euler method is a sum of local errors,
each amplified by a factor depending to the number of iterations left to reach the final time.
\emph{Consistency} of a numerical method enables to control the size of local errors when they arise,
while~\emph{stability} enables to control their growth.

Note that both the forward and the backward Euler methods may be recast in the form~\eqref{eq:general_one_step}.
For the forward Euler method $\Phi_{\Delta}(t, \vect x) = \vect f(t, \vect x)$,
while for the backward Euler method, the function $\vect \Phi_{\Delta}$ is defined implicitly as the function which to~$(t, \vect x)$ associates the solution~$\vect \phi \in \real^n$ to the equation
\[
    \vect \phi = \vect f(t + \Delta, \vect x + \Delta \vect \phi).
\]
\subsubsection*{Local truncation error and consistency}%
The local truncation error is the residual error obtained when substituting the exact solution of the differential equation in~\eqref{eq:general_one_step}:
\[
    \vect \eta_{n+1} := \frac{\vect x(t_{n+1}) - \vect x(t_n)}{\Delta} + \vect \Phi_{\Delta}\bigl(t_n, \vect x(t_n)\bigr).
\]
Since there is a division by~$\Delta$, the local truncation error has the same physical dimension as that of $\vect x'$,
and so it should be viewed as an error \emph{per time unit}.

\begin{definition}
    [Consistency]
    A numerical method is consistent if
    \[
        \lim_{\Delta \to 0} \left( \max_{1 \leq n \leq N} \vecnorm {\vect \eta_n} \right) = 0, \qquad N = \floor*{\frac{T}{\Delta}}.
    \]
    It is consistent with order~$p$ if there exists~$C$ such that
    \[
        \forall \Delta > 0, \qquad
         \max_{1 \leq n \leq N} \vecnorm {\vect \eta_n} \leq C \Delta^p.
    \]
\end{definition}
Proving the consistency of a numerical method is usually achieved on a case-by-case basis by application of Taylor's formula.

\subsubsection*{Stability}%
\label{ssub:Stability}
The stability of a numerical method qualifies its sensitivity to perturbations.
Roughly speaking, it expresses that small perturbation of the right-hand side of~\eqref{eq:general_one_step} lead to small perturbations of the numerical solution.
\begin{definition}
    [Stability]
    A numerical method of the form~\eqref{eq:general_one_step} is stable if there exists a constant~$S(T) > 0$ independent of $\Delta$ such that for all sequence~$(\vect y_n)_{1\leq n \leq N}$ satisfying
    \begin{equation}
        \label{eq:general_one_step_perturbed}
        \vect y_{n+1} = \vect y_n + \Delta \vect \Phi_{\Delta}(t_n, \vect y_n) + \Delta \vect \delta_n, \qquad \vect y_0 = \vect x_0,
    \end{equation}
    it holds that
    \begin{equation}
        \label{eq:stability_numerical_method}
        \max_{1\leq n \leq N} \vecnorm{ \vect x_n - \vect y_n } \leq S(T) \Delta \sum_{n=1}^{N} \vecnorm{ \vect \delta_n }.
    \end{equation}
\end{definition}
It is convenient to introduce the following norms for sequences of vectors $(\vect u_{n})_{1 \leq n \leq N}$:
\[
    \norm{\vect u_{\placeholder}}_{\ell^1_T} = \sum_{n=1}^{N} \vecnorm{\vect u_{n}},
    \qquad
    \norm{\vect u_{\placeholder}}_{\ell^{\infty}_T} = \max_{1\leq n \leq N} \vecnorm{\vect u_{n}},
\]
With these notations, equation~\eqref{eq:stability_numerical_method} may be rewritten compactly as follows:
\[
    \norm{ \vect x_{\placeholder} - \vect y_{\placeholder} }_{\ell^{\infty}_T} \leq S(T) \Delta \norm{ \vect \delta_{\placeholder} }_{\ell^{1}_T}
\]
One could argue that this equation is neater than~\eqref{eq:stability_numerical_method};
it bounds a norm of just one mathematical object, namely the sequence $(\vect x_n - \vect y_n)_{1\leq n \leq N}$,
by a norm of another object, namely the sequence $(\vect \delta_n)_{1\leq n \leq N}$.
Arguments for proving that a numerical scheme is stable often rely on some form of Lipschitz continuity.
If the function $\Phi_{\Delta}(t, \vect y)$ is globally Lipschitz continuous with respect to~$\vect y$,
then stability is particularly simple to prove,
as we now demonstrate.

\begin{proposition}
    Assume that there is $L_{\Phi} > 0$ such that for all $t \in [0, T]$ and $\Delta > 0$,
    the function $\Phi_{\Delta}(t, \placeholder)$ is globally Lipschitz continuous with constant~$L_{\Phi}$.
\end{proposition}
\begin{proof}
    By~\eqref{eq:general_one_step} and~\eqref{eq:general_one_step_perturbed},
    it holds that
    \[
        \vect x_{n} - \vect y_{n}
        = \vect x_{n-1} - \vect y_{n-1}
        + \Delta \Bigl( \vect \Phi_{\Delta}(t_{n-1}, \vect x_{n-1}) - \vect \Phi_{\Delta}(t_{n-1}, \vect y_{n-1}) \Bigr)
        - \Delta \vect \delta_{n}.
    \]
    Taking the Euclidean norm and employing the Lipschitz continuity assumption,
    we obtain
    \[
        \vecnorm{ \vect x_{n} - \vect y_{n} }
        \leq (1 + \Delta L_{\Phi}) \vecnorm{ \vect x_{n-1} - \vect y_{n-1} }
        + \Delta \vecnorm{ \vect \delta_{n} }.
    \]
    By a reasoning similar to that in the proof of~\cref{theorem:forward_euler},
    we then obtain
    \[
        \vecnorm{ \vect x_{n} - \vect y_{n} }
        \leq (1 + \Delta L_{\Phi})^{n} \vecnorm{ \vect x_{0} - \vect y_{0} } + \sum_{i=1}^{n} (1 + \Delta L_{\Phi})^{n-i} \Delta \vecnorm{\vect \delta_{i}}
        \leq 0 + \e^{L_{\Phi} t_n} \Delta \sum_{i=1}^{n}\vecnorm{\vect \delta_{i}}.
    \]
    We conclude that~\eqref{eq:stability_numerical_method} is satisfied with~$S(T) = \e^{L_{\Phi} T}$.
\end{proof}

\subsubsection*{Convergence}%
We are now ready to prove that consistency and stability of the numerical~\eqref{eq:general_one_step} together imply convergence,
in the sense that
\[
    \lim_{\Delta \to 0} \left( \max_{1 \leq n \leq N}  \vecnorm{ \vect x(t_n) - \vect x_n } \right) = 0, \qquad  N = \floor*{\frac{T}{\Delta}}.
\]
This result is an instance of the \emph{Lax equivalence theorem},
a pillar of numerical analysis with far-reaching applications.

\begin{theorem}
    [Consistence and stability imply convergence]
    \label{theorem:lax_convergence}
    Assume that the one-step numerical method~\eqref{eq:general_one_step} is consistent and stable.
    Then the method is also convergent.
\end{theorem}
\begin{proof}
By definition of the local truncation error, it holds that
\[
    \vect x(t_{n+1}) = \vect x(t_n) + \Delta \vect \Phi_{\Delta}\bigl(t_n, \vect x(t_n)\bigr) + \Delta \vect \eta_{n+1}.
\]
Therefore, the sequence $\bigl(\vect x(t_n)\bigr)_{1 \leq n \leq N}$ satisfies~\eqref{eq:general_one_step_perturbed} with $\vect \delta_n = \vect \eta_n$,
and so the stability estimate~\eqref{eq:stability_numerical_method} implies that
\[
    \max_{1\leq n \leq N} \vecnorm{ \vect x(t_n) - \vect x_n } \leq S(T) \Delta \sum_{n=1}^{N} \vecnorm{ \vect \eta_n }.
\]
By consistency, the right-hand side converges to zero in the limit as~$\Delta \to 0$,
which concludes the proof.
\end{proof}

\begin{remark}
    If we assume in \cref{theorem:lax_convergence} that the method is consistent with order~$p$,
    then by adapting the proof,
    we find that the error satisfies
    \[
        \max_{1\leq n \leq N} \vecnorm{ \vect x(t_n) - \vect x_n } \leq C S(T) \Delta^p.
    \]
    In this setting,
    the numerical scheme is said to be \emph{convergent with order~$p$}.
\end{remark}

\subsection{Widely used one-step methods}
\label{sub:other_one_step}
In this section, we motivate and describe some of the other widely-used one-step methods,
namely methods of Taylor and Runge--Kutta type.
We assume in this section that the equation~\eqref{eq:ode} admits a unique smooth solution over the interval~$[0, T]$.

\subsubsection*{Taylor methods}%
In order to construct a method with a smaller local truncation error than that of the forward Euler method,
a Taylor expansion of higher order than~\eqref{eq:taylor_forward} can be employed:
\begin{equation}
    \label{eq:taylor_high_order}
    \vect x(t + \Delta) = \vect x(t)
    + \Delta \vect x'(t)
    + \dotsb
    + \frac{\Delta^p}{p!} \vect x^{(p)}(t)
    + \frac{\Delta^{p+1}}{(p+1)!} \vect x^{(p+1)}(\tau),
    \qquad \tau \in (t, t + \Delta).
\end{equation}
Since $\vect x\colon [0, T] \to \real$ is a smooth solution to~\eqref{eq:ode} by assumption,
the time derivatives of~$\vect x$ can be obtained by differentiation of~\eqref{eq:ode}:
\[
    \vect x'(t) = \vect f\bigl(t, \vect x(t)\bigr),
    \qquad
    \vect x''(t) = \partial_t \vect f\bigl(t, \vect x(t)\bigr)
    + \nabla_{\vect x} \vect f\bigl( t, \vect x(t) \bigr) \cdot \vect f\left(t, \vect x(t)\right),
    \qquad
    \dotsc
\]
In general, it is immediate to show inductively that $\vect x^{(p)}(t) = \vect f^{(p-1)}\bigl(t, \vect x(t)\bigr)$,
where the functions~$\vect f^{(p)}\colon \real \times \real^n \to \real$ are defined recursively from the following equation:
\[
    \vect f^{(p+1)} = \partial_t \vect f^{(p)}\bigl(t, \vect x(t)\bigr) + \nabla_{\vect x} \vect f^{(p)}\bigl( t, \vect x(t) \bigr) \cdot \vect f\left(t, \vect x(t)\right).
\]
The Taylor expansion~\eqref{eq:taylor_high_order} motivates the so-called Taylor methods for integrating~\eqref{eq:ode} numerically,
which are based on the following iteration:
\begin{equation}
    \label{eq:taylor_scheme}
    \vect x_{n+1} = \vect x_n + \Delta \vect T^{p}_{\Delta}\bigl(t_n, \vect x_n\bigr),
\end{equation}
where
\[
    \vect T^p_{\Delta}(t, \vect x)
    := \vect f(t, \vect x) + \frac{\Delta}{2!} \vect f^{(1)}(t, \vect x)
    + \dotsb + \frac{\Delta^{p-1}}{p!} \vect f^{(p-1)}(t, \vect x).
\]
Note that, for any~$p$,
the Taylor scheme~\eqref{eq:taylor_scheme} may be rewritten as
\[
    % \label{eq:taylor_scheme}
    \vect x_{n+1} = \vect x_n
    + \Delta \frac{\d \vect x^{t_n, \vect x_n}}{\d t} (t_n)
    + \dotsb
    + \frac{\Delta^p}{p!} \frac{\d^p \vect x^{t_n, \vect x_n}}{\d t^{p}} \vect (t_n).
\]
For~$p=1$, this scheme coincides with the forward Euler scheme.


\subsubsection*{Runge--Kutta methods}%
% Like Taylor methods, Runge--Kutta methods enjoy a local truncation error of high order,
Runge--Kutta methods resemble Taylor methods,
but they do not require to calculate the derivatives of the function~$\vect f$.
This is achieved by approximating the derivatives in Taylor methods by difference quotients.
Consider for example the Taylor method of order $p = 2$:
\begin{equation}
    \label{eq:taylor_order_2}
    \vect x_{n+1} = \vect x_n
    + \Delta \frac{\d \vect x^{t_n, \vect x_n}}{\d t} (t_n)
    + \frac{\Delta^2}{2!} \frac{\d^2 \vect x^{t_n, \vect x_n}}{\d t^{2}} \vect (t_n).
\end{equation}
Substituting the approximation
\begin{align}
    \nonumber
    \frac{\d^2 \vect x^{t_n, \vect x_n}}{\d t^{2}} \vect (t_n)
    &\approx \frac{1}{\Delta} \left( \frac{\d \vect x^{t_n, \vect x_n}}{\d t} \vect (t_n + \Delta) - \frac{\d \vect x^{t_n, \vect x_n}}{\d t}(t_n) \right) \\
    \nonumber
    &= \frac{1}{\Delta} \Bigl( \vect f \bigl( t_n + \Delta, \vect x^{t_n, \vect x_n}(t_n + \Delta) \bigr) - \vect f \bigl( t_{n}, \vect x_n \bigr) \Bigr) \\
    \label{eq:approx_heun}
    &\approx \frac{1}{\Delta} \Bigl( \vect f \bigl( t_{n} + \Delta, \vect x_n + \Delta \vect f(t_n, \vect x_n) \bigr) - \vect f \bigl( t_{n}, \vect x_n \bigr) \Bigr)
\end{align}
in~\eqref{eq:taylor_order_2},
we obtain an explicit method known as \emph{Heun's method}:
\[
    \vect x_{n+1} = \vect x_n
    + \frac{\Delta}{2}\vect f(t_n, \vect x_n)
    + \frac{\Delta}{2} \vect f \bigl( t_n + \Delta, \vect x_n + \Delta \vect f(t_n, \vect x_n) \bigr).
    % =: \vect x_n + \vect K^2_{\Delta}(t_n, \vect x_n)
\]
It is possible to show that
% $\vect K^2_{\Delta}(t_n, \vect x_n) = \vect T^2_{\Delta}(t_n, \vect x_n) + \bigo(\Delta^2)$, and so
the local truncation error for this method also scales as~$\Delta^2$.
Heun's method is a particular instance of a Runge--Kuta method.
In general, an explicit Runge--Kutta method with~$s$ stages is of the form
\begin{align*}
    \vect x_{n+1} &= \vect x_n + \Delta \sum_{i=1}^s b_i \vect k_i \\
    \vect k_1 &= f(t_n, \vect x_n),  \\
    \vect k_2 &= f\bigl(t_n + c_2 \Delta, \vect x_n+\Delta(a_{21} \vect k_1)\bigr), \\
    \vect k_3 &= f\bigl(t_n + c_3 \Delta, \vect x_n+\Delta(a_{31} \vect k_1 + a_{32} \vect k_2)\bigr), \\
              &\;\;\vdots \\
    \vect k_i &= f\left(t_n + c_i \Delta, \vect x_n + \Delta \sum_{j = 1}^{i-1} a_{ij} k_j\right),
\end{align*}
with appropriate coefficients $c_i$ and $a_{ij}$.
Heun's iteration can be recast in this form as follows:
\begin{align*}
    \vect x_{n+1} &= \vect x_n + \frac{\Delta}{2}(\vect k_1 + \vect k_2) \\
    \vect k_1 &= \vect f(t_n, \vect x_n)  \\
    \vect k_2 &= \vect f \bigl( t_{n} + \Delta, \vect x_n + \Delta \vect k_1 \bigr).
\end{align*}
The approach we employed to construct Heun's method may be generalized to higher orders.
For example, the most widely known Runge--Kutta method
approximates the Taylor method of order $p=4$ with the following iteration:
\begin{align*}
    \vect x_{n+1} &= \vect x_n + \frac{\Delta}{6}(\vect k_1 + 2 \vect k_2 + 2 \vect k_3 + \vect k_4), \\
     \vect k_1 &= \vect f(t_n, \vect x_n), \qquad
               && \vect k_2 = \vect f\left(t_n + \frac{\Delta}{2}, \vect x_n + \Delta\frac{\vect k_1}{2}\right), \\
     \vect k_3 &= \vect f\left(t_n + \frac{\Delta}{2}, \vect x_n + \Delta\frac{\vect k_2}{2}\right), \qquad
               && \vect k_4 = \vect f\left(t_n + \Delta, \vect x_n + \Delta \vect k_3\right).
\end{align*}
The local truncation error for this method scales as~$\Delta^4$ and,
when~$\vect f(t, \vect x) = \vect f(t)$,
this method coincides with Simpson's formula~\eqref{eq:simpsons} for the approximation of the integral in~\eqref{eq:one_step_integral}.
The systematic derivation of Runge--Kutta methods is cumbersome,
and so we do not address this issue in this course.

\begin{remark}
    Explicit Runge--Kutta methods of a given order are not uniquely defined.
    For example, if we employ instead of~\eqref{eq:approx_heun} the approximation
    \[
        \frac{\d^2 \vect x^{t_n, \vect x_n}}{\d t^{2}} \vect (t_n)
        \approx \frac{2}{\Delta} \left( \vect f \left(t_n + \frac{\Delta}{2}, \vect x_n + \frac{\Delta}{2} \vect f(t_n, \vect x_n) \right) - \vect f \bigl( t_{n}, \vect x_n \bigr) \right),
    \]
    then we obtain by substitution in~\eqref{eq:taylor_order_2} the so-called \emph{explicit midpoint method},
    which is also a Runge--Kutta method of order 2:
    \begin{align*}
        \vect x_{n+1} &= \vect x_n + \Delta \vect f \left( t_{n} + \frac{1}{2}\Delta, \vect x_n + \frac{\Delta}{2} \vect f(t_n, \vect x_n) \right).
    \end{align*}
\end{remark}

\subsubsection*{Implicit methods}%
To conclude this section,
we mention two common implicit methods with a better order of convergence than that of the backward Euler method.
\begin{itemize}
    \item
        The Crank--Nicolson method:
        \begin{equation}
            \label{eq:crank_nicolson}
            \vect x_{n+1} = \vect x_n + \frac{\Delta}{2} \bigl( f(t_n, \vect x_n) + f(t_n + \Delta, \vect x_{n+1}) \bigr).
        \end{equation}
        When $\vect f$ is independent of~$\vect x$ and depends only on~$t$,
        this method coincides with the trapezoidal rule for numerical integration.

    \item
        The implicit midpoint method:
        \[
            \vect x_{n+1} = \vect x_n + \Delta  f\left(t_n + \frac{\Delta}{2}, \frac{\vect x_n + \vect x_{n+1}}{2} \right).
        \]
\end{itemize}
Similarly to the backward Euler method,
each iteration of these methods requires the resolution of a nonlinear equation.
Implicit methods often enjoy better stability than their explicit counterparts.
This subject is further discussed in~\cref{sec:absolute_stability}.

\section{Multistep methods}
The idea of multistep methods is to use,
in the construction of a new iterate,
information from previous iterations.
This degree of freedom enables to construct more economical numerical methods than one-step methods for the same order of convergence,
at the cost of a more difficult initialization.
In this section we focus on~\emph{linear} multistep methods of the form
 \begin{align}
     \nonumber
     \vect x_{n+1}  &= a_0 \vect x_n + a_1 \vect x_{n-1} + \dotsb + a_{k} \vect x_{n-k} \\
     \label{eq:linear_mulitstep}
                    &\qquad + \Delta \Bigl( b_{-1} \vect f(t_{n+1}, \vect x_{n+1}) + b_0 \vect f(t_{n}, \vect x_{n}) + \dotsb + b_k \vect f(t_{n-k}, \vect x_{n-k}) \Bigr).
\end{align}
This equation defines an explicit method if $b_{-1} = 0$,
and an implicit method if $b_{-1} \neq 0$.
Note that explicit methods of the form~\eqref{eq:linear_mulitstep} require only one additional evaluation~$f(t_{n}, \vect x_{n})$ per iteration.
When~$b_0 \neq 0$, the iteration~\eqref{eq:linear_mulitstep} is a nonlinear equation for the unknown~$\vect x_{n+1}$,
which must itself be solved by resorting to a numerical method.

\paragraph{Initialization.}
In order to initiate the numerical method~\eqref{eq:linear_mulitstep},
the values $\vect x_0, \dotsc, \vect x_{k-1}$ are required.
These can be calculated by using a one-step method with an order of convergence matching that of the multistep method.

\paragraph{Local truncation error.}
Consistently with the setting of one-step methods,
the local truncation error for~\eqref{eq:linear_mulitstep} is defined as the residual error left when the exact solution is substituted in the numerical scheme:
\begin{align}
    \nonumber
    \Delta \vect \eta_{n+1}
        &:= \vect x(t_n + \Delta) - a_0 \vect x(t_n) - a_1 \vect x(t_n - \Delta) - \dotsb - a_k \vect x(t_n - k \Delta) \\
        \label{eq:truncation_multistep}
        & \qquad- \Delta \Bigl( b_{-1} \vect x'(t_n + \Delta) + b_0 \vect x'(t_n) + \dotsb + b_k \vect x'(t_n - k \Delta) \Bigr).
\end{align}
The multistep method~\eqref{eq:linear_mulitstep} is of order~$p$ if the maximum local truncation error
over all the discretization points, in norm, scales as~$\bigo(\Delta^p)$.
The following result is useful for estimating the order of consistency of a linear multistep method.
\begin{proposition}
    The linear multistep method~\eqref{eq:linear_mulitstep} is consistent with order~$p$ for any smooth $\vect x\colon [0, T] \to \real^n$ if and only if
    the local truncation error~\eqref{eq:truncation_multistep} is everywhere zero when~$\vect x(t)$ is of the form
    \begin{equation}
        \label{eq:polynomial_form}
        \vect x(t) = \vect e t^{q}, \qquad q \in \{0, \dotsc, p\},  \qquad \vect e \in \real^n.
    \end{equation}
\end{proposition}
% Before presenting the proof,
% we note that the right-hand side of~\eqref{eq:truncation_multistep} is linear in~$\vect x$.
% Therefore, if the local truncation error is zero for any monomial~$\vect x(t)$,
% then it is also zero for any polynomial
\begin{proof}
    Assume that the method is consistent with order~$p$, fix $q \in \{1, \dotsc, p\}$ and $\vect e \in \real^n$,
    and let $\vect x(t) = \vect e t^{q}$.
    Fix also $t \in [0, T]$ and consider the function $\vect \xi\colon \{\Delta : t/\Delta \in \nat_{> 0}\} \to \real^n$ given by
    \begin{align}
        \nonumber
        \Delta \vect \xi(\Delta) = \Delta \vect \eta_{(t/\Delta) +1}
            &= \vect x(t + \Delta) - a_1 \vect x(t) - a_2 \vect x(t - \Delta) - \dotsb - a_k \vect x\bigl(t - (k-1) \Delta\bigr) \\
            \label{eq:truncation_multistep}
            & \qquad- \Delta \Bigl( b_0 \vect x'(t + \Delta) + b_1 \vect x'(t) + \dotsb + b_k \vect x'\bigl(t - (k-1) \Delta\bigr) \Bigr).
    \end{align}
    The quantity $\vect \xi(\Delta)$ should be understood as the local truncation error at~$t$ for time step $\Delta$.
    Each entry of $\vect \xi$, viewed as a function of~$\Delta$,
    is a polynomial of degree at most~$p$ scaling as~$\bigo(\Delta^{p+1})$.
    Therefore, it holds necessarily that~$\vect \xi(\Delta) = 0$.

    Conversely, assume that the right-hand side of~\eqref{eq:truncation_multistep} is equal to zero for any function of the form~\eqref{eq:polynomial_form}.
    If~$\vect x(t)$ denotes a smooth solution of~\eqref{eq:ode},
    then by Taylor's theorem there is $C > 0$ independent of $t_n$ such that
    \[
        \forall t \in [0, T], \qquad
        \left\{
        \begin{aligned}
            \vecnorm{\vect x(t) - \vect y(t)} &\leq C|t-t_n|^{p+1} \\
            \vecnorm{\vect x'(t) - \vect y'(t)} &\leq C|t-t_n|^{p}
        \end{aligned}
        \right. ,
        \qquad
        \vect y(t) := \vect x(t_n) + \sum_{i=1}^{p} \vect e_i (t-t_n)^i,
    \]
    for appropriate vectors $\vect e_i \in \real^n$ depending on~$t_n$.
    Substituting $\vect x(t) = \vect y(t) + \bigl(\vect x(t) - \vect y(t)\bigr)$ in the right-hand side of~\eqref{eq:truncation_multistep},
    we obtain
    \[
        \Delta \vecnorm{\vect \eta_{n+1}} = \bigo(\Delta^{p+1}) + \Delta \bigo(\Delta^p) = \bigo(\Delta^{p+1}),
    \]
    with the constant implicit in the big~$\bigo$ notation independent of~$n$.
    This concludes the proof.
\end{proof}
\begin{example}
    In the one-dimensional setting,
    we wish to find parameters $a_0$, $a_1$ and~$b_1$ such that the order of consistency
    of the following multistep scheme is as high as possible:
     \[
         x_{n+1}  = a_0 x_n + a_2 x_{n-1} + b_0 \Delta f(t_{n}, x_{n}).
    \]
    Substituting $x(t) = 1$ in the formula~\eqref{eq:truncation_multistep} for the local truncation error,
    we obtain
    \[
        \eta_{n+1} = x(t_n + \Delta) - a_0 x(t_n) - a_1 x(t_n - \Delta) - b_0 \Delta x'(t_n)
        = 1 - a_0 - a_1.
    \]
    Therefore $a_1 = (1 - a_0)$.
    Next, substituting $x(t) = t - t_n$ in~\eqref{eq:truncation_multistep},
    we obtain
    \[
        \eta_{n+1} = \Delta (2 - a_0 - b_0),
    \]
    which gives $b_0 = 2 - a_0$.
    Finally, substituting $x(t) = (t-t_n)^2$,
    we obtain
    \[
        \eta_{n+1} = \Delta^2 a_0,
    \]
    and so $a_0 = 1$.
    We conclude that the best parameters,
    leading to a local truncation error scaling as~$\bigo(\Delta^2)$,
    are given by~$a_0 = 0$, $a_1 = 1$ and $b_0 = 2$.
    The resulting method reads
    \[
         x_{n+1} = x_{n-1} + 2 \Delta f(t_{n}, x_{n}),
    \]
    and is known as the \emph{multistep midpoint method}.
\end{example}
We now present two widely used systematic approaches for constructing multistep methods,
known as the Adams--Bashforth and Adams--Moulton approaches.

\subsection{Adams--Bashforth methods}
Let $\vect x\colon [0, T] \to \real^n$ denote the exact solution to the differential equation~\eqref{eq:ode}.
Integrating this equation between $t_n$ and $t_{n+1}$,
we obtain
\begin{equation}
    \label{eq:adams_bashforth_intro}
    \vect x(t_{n+1}) = \vect x(t_n) + \int_{t_n}^{t_{n+1}} \vect f\bigl(t, \vect x(t)\bigr) \, \d t.
\end{equation}
The key idea of the Adams--Bashforth method is to approximate the function $t \mapsto \vect f\bigl(t, \vect x(t)\bigr)$ by the interpolating polynomial~$\widehat {\vect f}$ of degree $k$ at the nodes
$t_{n-k}, \dotsc, t_n$:
\begin{equation}
    \label{eq:interpolation}
    \widehat {\vect f}(t) = \sum_{i=0}^{k} \vect f\bigl(t_{n-i}, \vect x(t_{n-i})\bigr) L_i(t),
    \qquad L_i(t) := \prod_{\substack{j=0 \\ j\neq i}}^{k} \frac{t - t_{n-j}}{t_{n-i} - t_{n-j}}.
\end{equation}
Substituting this approximation in~\eqref{eq:adams_bashforth_intro},
we obtain
\[
    \vect x(t_{n+1}) \approx \vect x(t_n) + \sum_{i=0}^{k} \vect f\bigl(t_{n-i}, \vect x(t_{n-i})\bigr) \int_{t_n}^{t_{n+1}} L_i(t)\, \d t.
\]
This motivates the following \emph{Adams--Bashforth} numerical scheme:
\begin{equation}
    \label{eq:adams_bashforth}
    \vect x_{n+1} = \vect x_n + \Delta \sum_{i=0}^{k} b_i \vect f\bigl(t_{n-i}, \vect x_{n-i}\bigr),
    \qquad b_i := \int_{0}^{1} \prod_{\substack{j=0 \\ j\neq i}}^{k} \frac{s + j}{-i + j} \, \d s.
\end{equation}
Since the Lagrange polynomials $(L_i)_{0\leq i\leq k}$ depend on~$k$,
so do the coefficients $b_i$.
However, these are independent of~$\Delta$,
and so they can be tabulated.
The value of these coefficients for the first few Adams--Bashforth methods are collated in~\cref{table:adams_bashforth}.

\begin{table}[ht]
    \centering
    \begin{tabular}{|c|c|c|c|c|}
         \hline \phantom{$\Big($}
             $i$ & $0$ & $1$ & $2$ & $3$
         \\ \hline \phantom{$\Big($}
             $k=0$ & $1$ & & &
         \\ \hline \phantom{$\Big($}
             $k=1$ & $\frac{3}{4}$ & $-\frac{1}{4}$ & &
         \\ \hline \phantom{$\Big($}
             $k=2$ & $\frac{23}{12}$ & $-\frac{16}{12}$ & $\frac{5}{12}$ &
         \\ \hline \phantom{$\Big($}
             $k=3$ & $\frac{55}{24}$ & $-\frac{59}{24}$ & $\frac{37}{24}$ & $-\frac{9}{24}$
         \\ \hline
    \end{tabular}
    \caption{%
        Coefficients $(b_i)_{0 \leq i \leq k}$ of the Adams--Bashforth methods.
    }
    \label{table:adams_bashforth}
\end{table}

\paragraph{Local truncation error.}
Applying \cref{theorem:interpolation_error} for the interpolation error component-wise,
we obtain
\[
    \forall t \in [0, T], \qquad \vecnorm*{\vect x'(t) - \widehat {\vect f}(t)}_{\infty}
    \leq \frac{\lvert t- t_{n-k} \rvert \dotsb \lvert t - t_n \rvert}{(k+1)!} \sup_{t \in [0, T]} \vecnorm*{\vect x^{(k+2)}(t)}_{\infty},
\]
where $\widehat {\vect f}$ is the function defined in~\eqref{eq:interpolation}.
Since
\[
    \Delta \vect \eta_{n+1}
    = \vect x(t_{n+1}) - \vect x(t_n) - \Delta \sum_{i=0}^{k} b_i \vect f\bigl(t_{n-i}, \vect x(t_{n-i})\bigr)
    = \int_{t_n}^{t_{n+1}} \Bigl(\vect x'(t) - \widehat{\vect f}(t)\Bigr) \, \d t,
\]
we deduce that
\begin{equation}
    \label{eq:truncation_adams_bashforth}
    \vecnorm{\vect \eta_{n+1}} \leq C_k M_{k+2} \Delta^{k+1},
    \qquad M_{k+2} := \sup_{t \in [0, T]} \vecnorm*{\vect x^{(k+2)}(t)}_{\infty},
\end{equation}
for an appropriate numerical constant~$C_k$ independent of~$n$ and the problem data.
Therefore the Adams--Bashforth method~\eqref{eq:adams_bashforth} is consistent with order~$k+1$.

\paragraph{Convergence.}
Using a reasoning similar to that in the proof of~\cref{theorem:forward_euler},
we prove the convergence of the Adams--Bashforth method under a Lipschitz condition for the function~$\vect f$.
\begin{theorem}
    \label{theorem:adams_bashforth}
    Assume that the solution $\vect x\colon [0, T] \to \real^n$ to~\eqref{eq:ode} is $k+2$ times continuously differentiable
    and that the global Lipschitz condition~\eqref{eq:global_lipschitz} is satisfied.
    Suppose also that
    \[
        \forall i \in \{0, \dotsc, k\}, \qquad
        \vecnorm{\vect x(t_i) - \vect x_i}
        \leq \delta.
    \]
    Then the following error estimate holds for the Adams--Bashforth method~\eqref{eq:adams_bashforth}:
    \begin{equation}
        \label{eq:error_bound_forward_euler}
        \forall n \in \left\{0, 1, \dotsc, \floor*{\frac{T}{\Delta}} \right\},
        \qquad
        \vecnorm{\vect x(t_n) - \vect x_n}
        \leq
        \delta \e^{L B} + C_k M_{k+2} \Delta^{k+1} \left( \frac{\e^{LB t_n} - 1}{LB} \right),
    \end{equation}
    where $C_k$ and $M_{k+2}$ are the constants from~\eqref{eq:truncation_adams_bashforth},
    and with $B := \abs{b_0} + \dotsb + \abs{b_k}$.
\end{theorem}
\begin{proof}
    Let $\vect e_n := \vect x(t_{n+1}) - \vect x_{n+1}$.
    From the equation
    \[
        \vect x(t_{n+1}) - \vect x_{n+1} =
        \vect x(t_{n}) - \vect x_n + \Delta \sum_{i=0}^{k} b_i \Bigl( \vect f\bigl(t_{n-i}, \vect x(t_{n-i})\bigr) -  \vect f\bigl(t_{n-i}, \vect x_{n-i}\bigr) \Bigr)
        + \Delta \vect \eta_{n+1},
    \]
    which is valid for $n \geq k$,
    we deduce that
    \[
        \max\Bigl\{\vecnorm{\vect e_{0}},  \dotsc  , \vecnorm{\vect e_{n+1}} \Bigr\} \leq \left(1 + \Delta L B\right) \max\Bigl\{\vecnorm{\vect e_{0}},  \dotsc  , \vecnorm{\vect e_{n}} \Bigr\} + C_k M_{k+2} \Delta^{k+2}.
    \]
    Since $\max\Bigl\{\vecnorm{\vect e_{0}},  \dotsc  , \vecnorm{\vect e_{k}} \Bigr\} \leq \delta$ by assumption,
    the statement easily follows.
\end{proof}

\subsection{Adams--Moulton methods}
The Adams--Moulton methods are very similar to their Adams--Bashforth cousins.
The only difference is that the former are obtained by interpolating the function $t \mapsto \vect f\bigl(t, \vect x(t)\bigr)$ in~\eqref{eq:adams_bashforth_intro} at the nodes
$t_{n-k+1}, \dotsc, t_{n+1}$,
which leads to the method
\begin{equation}
    \label{eq:adams_moulton}
    \vect x_{n+1} = \vect x_n + \Delta \sum_{i=-1}^{k-1} b_i \vect f\bigl(t_{n-i}, \vect x_{n-i}\bigr),
    \qquad b_i := \int_{0}^{1} \prod_{\substack{j=-1 \\ j\neq i}}^{k-1} \frac{s + j}{-i + j} \, \d s.
\end{equation}
Unlike the Adams--Bashforth methods, which are \emph{explicit},
the Adams--Moulton methods are \emph{implicit}.
The value of the coefficients for the first few Adams--Moulton methods are collated in~\cref{table:adams_moulton}.
Notice that, for~$k = 0$, the Adams--Moulton method coincides with the backward Euler method,
and for $k =1$ it coincides with the Crank--Nicolson method.
\begin{table}[ht]
    \centering
    \begin{tabular}{|c|c|c|c|c|}
         \hline \phantom{$\Big($}
             $i$ & $-1$ & $0$ & $1$ & $2$
         \\ \hline \phantom{$\Big($}
             $k=0$ & $1$ & & &
         \\ \hline \phantom{$\Big($}
             $k=1$ & $\frac{1}{2}$ & $\frac{1}{2}$ & &
         \\ \hline \phantom{$\Big($}
             $k=2$ & $\frac{5}{12}$ & $\frac{8}{12}$ & $-\frac{1}{12}$ &
         \\ \hline \phantom{$\Big($}
             $k=3$ & $\frac{9}{24}$ & $\frac{19}{24}$ & $-\frac{5}{24}$ & $\frac{1}{24}$
         \\ \hline
    \end{tabular}
    \caption{%
        Coefficients $(b_i)_{0 \leq i \leq k}$ of the Adams--Moulton methods.
    }
    \label{table:adams_moulton}
\end{table}

\section{Absolute stability}
\label{sec:absolute_stability}


