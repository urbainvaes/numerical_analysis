\appendix
\chapter{Brief introduction to Julia}%
\label{cha:a_very_short_introduction_to_julia}

In this chapter, 
we very briefly present  some of the basic features and commands of Julia.
Most of the information contained in this chapter can be found in the online manual: 
\begin{itemize}
    \item \url{https://docs.julialang.org/en/v1/stdlib/Pkg/} (packages);
    \item \url{https://docs.julialang.org/en/v1/manual/control-flow/} (if, while, for);
    \item \url{https://docs.julialang.org/en/v1/manual/functions/} (functions);
    \item \url{https://docs.julialang.org/en/v1/manual/arrays/} (multi-dimensional arrays);
    \item \url{https://docs.juliaplots.org/latest/tutorial/} (plots).
\end{itemize}

\subsection*{Obtaining documentation}%
\label{sub:getting_documentation}

To get help on a function from the Julia shell, 
type ``?'' and then the name of the function.
Tab completion is helpful to see available names.

\begin{exercise}
    Read the help pages for \julia{if}, \julia{while} and \julia{for}.
\end{exercise}

\subsection*{Installing and using a package}%
\label{sub:installing_and_using_a_package}
To install a package for the Julia REPL (Read Evaluate Print Loop, also more simply called the Julia shell),
first type ``]'' to enter the package REPL,
and then type \julia{add} followed the name of the package to install.
For example, to use the \julia{Plots} package, write
\begin{minted}{julia}
import Plots
\end{minted}
in your the Julia shell or in a script.
If a function, say \julia{plot}, is defined by the package,
it can be accessed as \julia{Plots.plot}.

Alternatively, you may import the package with the \julia{using} keyword,
and then functions can be accessed without specifying the package name.
While convenient, this approach is less explicit, 
and does not enable to quickly know what package a function comes from.

\begin{exercise}
    Install the \texttt{Plots} package,
    read the documentation of the \texttt{Plots.plot} function,
    and use it to plot the function $f(x) = \sin(x)$.
    The tutorial on plotting, linked at the beginning of the chapter,
    is useful for this exercise.
\end{exercise}


\subsection*{Printing output}%
The functions \julia{println} or \julia{print} enable to display output.
The former adds a new line at the end and the latter does not.

\subsection*{Conditional construct}%
The basic structure of an ``if'' block is the following.
The \julia{elseif} and \julia{else} clauses are not necessary.
\begin{minted}{julia}
a = 1
if a == 0
    x = 0
elseif a == 1
    x = 1
else
    x = 2
end
\end{minted}
If there is no \julia{elseif} clause,
it is sometimes convenient to use the shorthand notation
\begin{minted}{julia}
x = (a == 0) ? 0 : 2
\end{minted}

\subsection*{While and For loops}%

Examples of while and for are presented below loops.
\begin{minted}{julia}
julia> i, j = 1, 3;
julia> while i <= j
           println(i)
           global i += 1
       end

1
2
3
\end{minted}


For more information,
see \url{https://docs.julialang.org/en/v1/manual/control-flow/}.

\subsection*{}%


