\chapter{Solution of linear systems of equation}
\label{cha:solution_of_linear_systems}

Systems of linear equations appear in a variety of applications.
They naturally arise after discretization of linear partial differential equation,
which govern a wide range of physical phenomena including heat and wave propagation, 
gravity, fluid flows and electromagnetism.
In the context of partial differential equations,
the matrices appearing in linear systems are often very sparse: many of the entries are equal to 0.
Additionally, they are often symmetric and positive definite,
provided these properties are satisfied by the underlying partial differential operator.

There are two main approaches for solving linear systems: 
\begin{itemize}
    \item 
        Direct methods enable to calculate the exact solution to the system of equations, 
        up to round-off errors.
        Although this is an attractive property,
        they are usually too computationally costly for large systems.

    \item 
        Iterative methods
\end{itemize}

% Whether or not these inevitable errors are sufficiently small to be neglected in a numerical method
% depends in general on the \emph{numerical stability} of the method.
% A numerical method is said to be \emph{numerically unstable} is small errors are magnified,
% and \emph{numerically stable} otherwise.
% these errors generally need not be a cause worry when they appear in methods that are numerically stables.

\begin{itemize}
    \item 
        In \cref{sec:conditioning},
        we introduce the concept of \emph{conditioning}.
        The condition number associated with a problem provides information on the sensitivity of the solution to errors in the data,
        so it is useful for determining the potential impact of round-off errors.
\end{itemize}

\section{Conditioning}%
\label{sec:conditioning}
