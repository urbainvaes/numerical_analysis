\documentclass[10pt]{article}
\usepackage{setspace}
\onehalfspacing
\usepackage{fontawesome}
\usepackage{svg}
\usepackage[newfloat]{minted}
\usepackage{algpseudocode}
\usepackage{algorithm}
\usepackage[margin=1.2in]{geometry}
\usepackage{amsmath,amsthm,amssymb}
\usepackage{mathtools}
\usepackage[colorlinks=true]{hyperref}
\theoremstyle{definition}
\newtheorem{question}{{\normalfont \faGears}~Question}
\newtheorem{compexercise}{{\normalfont \faLaptop}~Implementation exercise}
\DeclarePairedDelimiter\abs{\lvert}{\rvert}
\DeclarePairedDelimiter\norm{\lVert}{\rVert}
\DeclarePairedDelimiter\ip{\langle}{\rangle}

\DeclareMathOperator{\diag}{diag}
\DeclareMathOperator{\cond}{cond}
\DeclareMathOperator{\Span}{span}
\DeclareMathOperator{\sign}{sign}
\DeclareMathOperator*{\trace}{tr}
\DeclareMathOperator*{\argmin}{arg\,min}
\DeclareMathOperator*{\argmax}{arg\,max}

\renewcommand{\d}{\mathrm d}
\renewcommand{\t}{T}
\newcommand{\e}{\mathrm e}
\newcommand{\real}{\mathbf R}
\newcommand{\complex}{\mathbf C}
\newcommand{\nat}{\mathbf N}
\newcommand{\integer}{\mathbf Z}
\newcommand{\floating}{\mathbf F}
\newcommand{\madd}{\mathbin{\widehat +}}
\newcommand{\mtimes}{\mathbin{\widehat *}}
\newcommand{\msub}{\mathbin{\widehat -}}
\newcommand{\vect}{\mathbf}
\newcommand{\mat}{\mathsf}
\newcommand{\placeholder}{\mathord{\color{black!33}\bullet}}%
\newcommand{\julia}[1]{\mintinline{julia}{#1}}

\renewcommand{\mymarks}[1]{\textbf{#1 marks}}
\usepackage{fancyhdr}
\pagestyle{fancy}
\fancyhf{}
\lhead{Name: }
\chead{Midterm exam}
\rhead{Page \thepage}

\begin{document}

\title{Numerical Analysis: Midterm {\small (\mymarks{50})}}
\author{Urbain Vaes}
\date{Fall 11111100101}
\maketitle

\noindent \emph{You are not required to complete every question.Although the total marks on the exam sum to 55, your midterm grade will be calculated out of 50.}
\vspace{.5cm}

\begin{question}
    [Floating point arithmetic, \mymarks{10}]
    True or false?
    \begin{enumerate}
        \item Let $(\placeholder)_2$ denote binary representation.
            Then
            \(
                (0.1101)_2 + (0.0011)_2 = (1.0)_2.
            \)

        \item
            Let $\varepsilon_{64}$ denote the machine epsilon for the \julia{Float64} format,
            i.e.\ \julia{eps(Float64)}.
            Then the number~2 is representable exactly in this format,
            and the next representable number is $2 + \varepsilon_{64}$.

        \item It holds that
            \(
                (10000)_2 \times (0.1010)_2 = (1010)_2.
            \)
        \item 
            In Julia, \julia{Float64(.6) == Float32(.6)} evaluates to \julia{true}.

        \item 
            The spacing (in absolute value) between successive single-precision (\julia{Float32}) floating point numbers is constant.

        \item
            Infinitely many distinct real numbers can be represented exactly in the~\julia{Float64} format,
            but only finitely many can be represented exactly in the~\julia{Float32} format.

        \item 
            It holds that $(0.\overline{101})_2 = \frac{5}{7}$.

        \item 
            Machine addition~$\madd$ is an operation that is \emph{associative} but not \emph{commutative}.

        \item
            The machine epsilon is the smallest number of the form~$2^{-n}$ with $n \in \nat$ that can be represented exactly in a floating point format.

        \item
            In Julia, the expression \julia{1 + eps()/3 == 1 + eps()} evaluates to \julia{true}.

        \item
            \textbf{Bonus.}
            In Julia, the expression \julia{exp(eps()/2) == 1 + eps()} evaluates to \julia{true}.

            \textit{Explain briefly:}
            \vspace{1cm}

        \item
            \textbf{Bonus.}
            In Julia, the expression \julia{cos(eps()) == 1} evaluates to \julia{true}.

            \textit{Explain briefly:}
            \vspace{1cm}
    \end{enumerate}
\end{question}

\newpage
\begin{question}
    [Interpolation and approximation, \mymarks{10}]
    Are the following statements true or false? Prove or disprove.
    Recall that~$\mathcal P_d$ denotes the set of polynomials of degree at most~$d$.

    \begin{enumerate}
        \item
            Assume that $x_0 < x_1 < x_2 < x_3$ and $y_0, y_1, y_2, y_3$ are given real numbers. 
            Then there exists a polynomial~$p \in \mathcal P_3$ such that
            \(
                p(x_i) = y_i 
            \)
            for all $i \in \{0, 1, 2, 3\}$.

            \textit{Justification:}
            \vspace{3cm}

        \item
            Assume that $x_0 < x_1 < x_2$ and $y_0, y_1, y_2$ are given real numbers. 
            Then, there can exist \emph{at most one} polynomial~$p \in \mathcal P_2$ such that
            \(
                p(x_i) = y_i 
            \)
            for all $i \in \{0, 1, 2\}$.

            \textit{Justification:}
            \vspace{3cm}

        \item
            Let $p \in \mathcal P_d$ be a polynomial of degree $d > 0$,
            and let $q\colon \real \to \real$ be given by $q(x) = p(x+1) - p(x)$.
            Then it holds that~$q \in \mathcal P_{d-1}$.

            \textit{Justification:}
            \vspace{3cm}

        \item
            For $n \in \nat$, let $x^n_i = i/n$ for $i = 0, 1, \dotsc, n$.
            Assume that $u \colon \real \to \real$ is the smooth function given by~$u(x) = \sin(3x) + x^3$,
            and let $p_n \in \mathcal P_n$ denote the polynomial interpolation of~$u$ at the points $x^n_0, x^n_1, \dotsc, x^n_n$.
            Then it holds that
            \[
                \lim_{n \to \infty} \left( \max_{x \in [0, 1]} \bigl\lvert u(x) - p_n(x) \bigr\rvert \right) = 0 \, .
            \]

            \textit{Justification:}
            \vspace{3cm}
    \end{enumerate}
\end{question}

\newpage
\begin{question}
    [Interpolation, open-ended question, \mymarks{5}]
    In polynomial interpolation, the error depends both on the function being interpolated and the choice of interpolation nodes. Consider two families of nodes on the interval $[-1,1]$:
    \begin{itemize}
        \item Equally spaced nodes,
        \item Chebyshev nodes.
    \end{itemize}
    Discuss qualitatively (and illustrate with examples or plots if you wish) how the interpolation error behaves as the degree 
    $n$ increases in each case. 
    Why does one choice of nodes perform better for large~$n$,
    and what is the mathematical motivation for using Chebyshev nodes?
    \emph{You may refer to Runge’s phenomenon, but go beyond merely stating it.}
\end{question}

\newpage
\begin{compexercise}
    [Interpolation, \mymarks{5}]
    Write Julia code that computes and plots the interpolating polynomial~$p \in \mathcal P_3$ through the following points:
    $(0, 0)$, $(1, 4)$, $(2, 15)$, $(3, 40)$.
    The plot should display both the interpolation points and the graph of the interpolating polynomial over an appropriate range.
    Do not use any other library than the ones already imported.

        \begin{minted}{julia}
    using LinearAlgebra
    using Plots
    # Write your code here
        \end{minted}

    \vspace{9cm}
\end{compexercise}


\newpage
\begin{question}
    [Numerical integration, \mymarks{5}]
    Are the following statements true or false?
    Justify briefly.
    \begin{enumerate}
        \item
            The degree of precision of the following quadrature rule is 2:
            \[
                \int_{-1}^{1} u(x) \, \d x \approx 2u(0) \, .
            \]
            \textit{Justification:}
            \vspace{.7cm}

        \item
            The degree of precision of the following rule is equal to 3:
            \[
                \int_{-1}^{1} u(x) \, \d x \approx u \left(-\frac{1}{3}\right) + u \left(\frac{1}{3}\right).
            \]
            \textit{Justification:}
            \vspace{.7cm}

        \item
            For any natural number $N > 0$,
            there exists a quadrature rule with a degree of precision equal to $2N - 1$ of the form
            \[
                \int_{-1}^{1} u(x) \, \d x \approx
                \sum_{n=1}^{N} w_n u(x_n).
            \]
            \textit{Justification:}
            \vspace{.7cm}

        \item
            Let $x^N_i = i/N$ and consider the following approximation of $\int_{0}^{1} u(x) \, \d x$:
            \begin{equation}
                \label{eq:trapezium}
                \widehat I_N = \frac{1}{2N} \Bigl( u\left(x^N_0\right) + 2 u\left(x^N_1\right) + 2 u\left(x^N_2\right) + \dotsc + 2 u\left(x^N_{N-2}\right) + 2 u\left(x^N_{N-1}\right) + u\left(x^N_N\right) \Bigr).
            \end{equation}
            Suppose first that $u$ is the Runge function, given by $u(x) = (1 + 25 x^2)^{-1}$.
            Then $\widehat I_N$ diverges in the limit $N \to \infty$.

            \textit{Justification:}
            \vspace{.7cm}

        \item
            Let $u(x) = \cos(3x)$ and let $\widehat I_N$ be as in~\eqref{eq:trapezium}.
            Then it holds that
            \[
                \lim_{N \to +\infty}
                \left( \left\lvert \widehat I_N - \int_{0}^1 u(x) \, \d x \right\rvert \right) = 0.
            \]
            \textit{Justification:}
            \vspace{.7cm}

        \item
            (\textbf{Bonus.})
            Fix $u(x) = 2x - 1$ and let $\widehat I_N$ be as in~\eqref{eq:trapezium}.
            Then $\widehat I_N = 0$ for all $N \geq 2$.

            \textit{Justification:}
            \vspace{.7cm}
    \end{enumerate}
\end{question}



\newpage
\begin{question}
    [Gaussian--Hermite numerical integration, \mymarks{10}]

The Gauss--Hermite quadrature formula with \(n\) nodes is an approximation of the form
\[
I(u) \coloneq \int_{-\infty}^{\infty} u(x)\, \e^{-x^{2}} \, \d x \approx \sum_{i=1}^n w_i\,u(x_i)=:\widehat I_n(u),
\]
which is exact when \(u\) is a polynomial of degree \(\le 2n-1\).
Note that the nodes are numbered \(1,\dots,n\).
For this question, we take for granted that, for integers \(i\ge 0\),
it holds that
\[
    \int_{-\infty}^{\infty} x^{i} \e^{-x^{2}}\, \d x =
    \begin{cases}
        \displaystyle
        0, & \text{if $i$ is odd},\\[2mm]
        \displaystyle
        (i-1)!! \, \sqrt{\frac{\pi}{2^i}}, & \text{if $i$ is even},
    \end{cases}
\]
where 
\(
    (i-1)!! \coloneq 1 \times 3 \times 5 \times \cdots \times (i-1) \, .
\)
In particular, with all the integrals being over $(-\infty, \infty)$,
the following special cases may be useful in your computations:
\[
    \int \e^{-x^{2}}\, \d x = \sqrt{\pi}, \quad
    \int x^2 \, \e^{-x^{2}}\, \d x = \frac{\sqrt{\pi}}{2}, \quad
    \int x^4 \, \e^{-x^{2}}\, \d x = \frac{3}{4} \sqrt{\pi}, \quad
    \int x^6 \, \e^{-x^{2}}\, \d x = \frac{15}{8} \sqrt{\pi}\, .
\]



\begin{enumerate}
\item (\mymarks{5})
Find the nodes and weights of the Gauss--Hermite rule with \(n=3\) nodes.
By symmetry,
we expect nodes of the form $(-z,0,z)$ and weights $(w_1,w_2,w_1)$,
which reduces the number of unknowns to three.

\textit{Your answer:}
\vspace{7cm}

\newpage
\item (\mymarks{5})  
    Let $\{H_0, H_1, \dotsc\}$ denote orthogonal polynomials for the inner product 
    \[ 
        \ip{f, g} := \int_{-\infty}^{\infty} f(x) g(x) \, \e^{-x^2} \, \d x 
    \] 
    which, in addition,
    satisfy the following two conditions: 
    \begin{itemize} 
        \item 
            For all $i \in \nat$, the polynomial $H_i$ is of degree $i$. 
        \item 
            The leading coefficient of $H_i$, which multiplies $x^i$, is equal to 1. 
    \end{itemize} 
    Calculate $H_0$, $H_1$, $H_2$ and $H_3$.
    What is the relationship between \(H_3\) and the quadrature rule found in the first item?

    \textit{Your answer:}

\newpage
\vspace{10cm}
\item (Bonus, \mymarks{2})  
    Calculate $H_4$ and, using this result,
    deduce the nodes and weights of the Gauss--Hermite quadrature with~$4$ points.

    \textit{Your answer:}
\end{enumerate}
\end{question}
\newpage

\newpage
\begin{compexercise}
    [Numerical integration, \mymarks{10}]
    The midpoint quadrature rule reads
    \[
        \int_{-1}^{1} u(x) \, \d x
        \approx 2 u(0) \, .
    \]
    \begin{itemize}
        \item
            (\mymarks{3})
            Write a function \julia{midpoint(u, a, b)} that returns, using this quadrature rule,
            an approximation of the integral 
            \begin{align}
                \label{eq:integral_midpoint}
                \int_{a}^{b} u(x) \, \d x \, .
            \end{align}

            \begin{minted}{julia}
    function midpoint(u, a, b)
        # Write your code here




    end
            \end{minted}

        \item
            (\mymarks{4})
            Write a function \julia{composite_midpoint(u, a, b, N)}
            that returns an approximation of the integral~\eqref{eq:integral_midpoint},
            this time using a composite version of the midpoint rule.
            More precisely, the approximation should be obtained
            by partitioning the integration interval $[a, b]$ into $N$ cells,
            and applying the midpoint rule within each cell.

            \begin{minted}{julia}
    function composite_midpoint(u, a, b)
        # Write your code here











    end
            \end{minted}

        \item
            (\mymarks{3})
            Take $u(x) = \cos(x)$, $a = -1$ and $b = 1$.
            In this setting, plot the evolution of the error for $N$ varying from 1 to 1000.

            \begin{minted}{julia}
    using Plots
    # Write your code here
            \end{minted}
    \end{itemize}
\end{compexercise}

\end{document}
