\chapter{Notations}%
\label{cha:Notations}

Unless otherwise specified,
we use the following notation throughout these notes.

\begin{itemize}
    \item
        Lower case bold is used to denote vectors, e.g.\ $\vect x \in \complex^n$,
        and upper case sans serif is used to denote matrices, e.g.\ $\mat A \in \complex^{m \times n}$.
        The entries of a vector $\vect x \in \complex^n$ are denoted by~$(x_i)$,
        and those of a matrix $\mat A \in \complex^{m \times n}$ are denoted by~$(a_{ij})$ or~$(a_{i,j})$.

    \item 
        The notations $\ip{\placeholder,\placeholder}$ and $\norm{\placeholder}$ without a subscript always refer to the Euclidean inner product~\eqref{eq:induced_norm} and induced norm.

    \item
        The sequence $x_1, x_2, \dotsc$ is denoted by $(x_n)_{n\in \nat}$, 
        or sometimes just $(x_n)$.

    \item
        The notation 
        \( 
            B_{R}(\vect y)
        \)
        refers to the open ball of radius $R$ centered at $\vect y$.
\end{itemize}
