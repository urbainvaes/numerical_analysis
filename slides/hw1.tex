\documentclass[9pt]{beamer}
\usepackage{xstring}
\usepackage[cache=false,outputdir=build]{minted}
\usepackage[weather]{ifsym}

% Compilation options
\synctex=1

% Packages
\usepackage[utf8]{inputenx}
\usepackage[T1]{fontenc}
% \usepackage[french]{babel}
% \uselanguage{French}
% \languagepath{French}

\usepackage{bbm}
\usepackage{csquotes}
\usepackage{microtype}
\usepackage{listings}
\usepackage{appendix}
\usepackage{mathrsfs, mathenv}
\usepackage{mathtools}
\usepackage{amsmath, amsthm, amssymb, amsfonts, amscd}
\usepackage{graphicx}
\usepackage{epstopdf}
\usepackage{xcolor}
\usepackage{caption}
\usepackage{subcaption}
\usepackage{appendixnumberbeamer}
% \usepackage{varwidth}
\usepackage{xparse}
\usepackage[makeroom]{cancel}
\renewcommand{\CancelColor}{\color{red}}
\usepackage[%
    url=true, backend=biber,%
    style=trad-plain,
    % style=nature,
    isbn=false,
    url=false,
    doi=false]{biblatex}

% Fix setspace
\usepackage{setspace}
\setlength{\parskip}{6pt}

% Layout
\usefonttheme[onlymath]{serif}
\usecolortheme{whale}
\usecolortheme{rose}

\usepackage{tikz}
\usepackage{pgfplotstable}
\usetikzlibrary{patterns}
\usetikzlibrary{calc}
\usetikzlibrary{angles}
\usetikzlibrary{quotes}
\usetikzlibrary{decorations.pathmorphing,patterns}
\usetikzlibrary{decorations.pathreplacing,calc,tikzmark}

\mode<presentation>
\useoutertheme{infolines}
\setbeamercovered{transparent}
\setbeamertemplate{headline}{}
\setbeamertemplate{footline}[frame number]
\setbeamertemplate{navigation symbols}{}
\setbeamertemplate{itemize items}[square]

\setbeamercolor{footercolor}{fg=gray,bg=white}
\setbeamertemplate{footline}{
    \hbox{%
    \begin{beamercolorbox}[wd=\paperwidth,ht=3ex,dp=1.5ex,leftskip=2ex,rightskip=2ex]{footercolor}%
        \usebeamerfont{title in head/foot}%
        \textit \insertsection \hfill
        \insertframenumber{} / \inserttotalframenumber
    \end{beamercolorbox}}%
}

\newtheorem{question}{Question}
\newtheorem{assumption}{Assumption}[section]
% \newtheorem{theorem}{Assumption}[section]
\newtheorem{proposition}{Proposition}[section]
\newtheorem{result}{Result}[section]
\newtheorem{remark}{Remark}[section]
% \numberwithin{equation}{section}

% \definecolor{darkred}{rgb}{0.5,0,0}
\definecolor{green}{rgb}{0,0.5,0}
% \definecolor{darkblue}{rgb}{0,0,.5}
% \definecolor{darkblue}{HTML}{2ca02c}
\definecolor{darkgreen}{HTML}{2CA02C}
\definecolor{darkred}{HTML}{FF7F0E}
\definecolor{darkblue}{HTML}{1F77B4}
% u'#ff7f0e', u'#2ca02c', u'#d62728'
\definecolor{darkgrey}{RGB}{120,120,120}
\definecolor{titlecolor}{RGB}{0,0,0}
\definecolor{citecolor}{RGB}{10,137,116}
\definecolor{linkcolor}{RGB}{142,61,5}
\definecolor{urlcolor}{RGB}{71, 83, 1}
\pgfplotsset{compat=1.17}

\definecolor{myblue}{RGB}{51,51,178}
\definecolor{mygreen}{RGB}{51,178,51}
\definecolor{mygreen}{RGB}{184,53,175}
\definecolor{myred}{RGB}{178,51,51}
\definecolor{myred}{RGB}{52,164,186}
% \definecolor{myblue}{HTML}{0000FF}
% \setbeamercolor*{structure}{bg=myblue!20,fg=myblue}
\setbeamercolor{itemize subitem}{fg=red}
\setbeamercolor{itemize subsubitem}{fg=gray}
\setbeamertemplate{itemize/enumerate subbody begin}{\normalsize}
\setbeamertemplate{itemize/enumerate subsubbody begin}{\normalsize}
\captionsetup{font=scriptsize,labelfont=scriptsize}

\DeclarePairedDelimiter\abs{\lvert}{\rvert}
\DeclarePairedDelimiter\norm{\lVert}{\rVert}
\DeclarePairedDelimiter\vecnorm{\lVert}{\rVert}
\DeclarePairedDelimiter\ip{\langle}{\rangle}
\DeclarePairedDelimiter{\floor}{\lfloor}{\rfloor}
\DeclarePairedDelimiter{\ceil}{\lceil}{\rceil}

\DeclareMathOperator{\diag}{diag}
\DeclareMathOperator{\cond}{cond}
\DeclareMathOperator{\Span}{Span}
\DeclareMathOperator{\arccosh}{arccosh}
\DeclareMathOperator{\sign}{sign}
\DeclareMathOperator{\col}{col}
\DeclareMathOperator{\spectrum}{spectrum}
\DeclareMathOperator*{\trace}{tr}
\DeclareMathOperator*{\argmin}{arg\,min}
\DeclareMathOperator*{\argmax}{arg\,max}

\renewcommand{\leq}{\leqslant}
\renewcommand{\geq}{\geqslant}

\newcommand{\mymarks}[1]{(\textbf{#1 marks})}
\newcommand{\mymark}{(\textbf{1 mark})~}
\newcommand{\mybonus}[1]{(\textbf{Bonus +#1})~}
\renewcommand{\d}{\mathrm d}
\renewcommand{\i}{\mathrm i}
\renewcommand{\t}{T}
\newcommand{\e}{\mathrm e}
\newcommand{\D}{\mathrm D}
\newcommand{\real}{\mathbf R}
\newcommand{\poly}{\mathbf P}
\newcommand{\bigo}{\mathcal O}
\newcommand{\complex}{\mathbf C}
\newcommand{\nat}{\mathbf N}
\newcommand{\integer}{\mathbf Z}
\newcommand{\floating}{\mathbf F}
\newcommand{\madd}{\mathbin{\widehat +}}
\newcommand{\mdiv}{\mathbin{\widehat /}}
\newcommand{\mtimes}{\mathbin{\widehat *}}
\newcommand{\msub}{\mathbin{\widehat -}}
\newcommand{\vect}[1]{\mathbf{\boldsymbol{#1}}}
\newcommand{\mat}{\mathsf}
\newcommand{\placeholder}{\mathord{\color{black!33}\bullet}}%
\newcommand{\moreinfo}{\texorpdfstring{{\normalfont \color{lightred}$^{\text{\faSearchPlus}}$}}{}}
\newcommand{\laplacian}{\triangle}
\newcommand{\expect}{\mathbf E}
\newcommand{\var}{\mathbf V}
\newcommand{\proba}{\mathbf P}
\newcommand{\julia}[1]{\mintinline{julia}{#1}}


\renewcommand{\emph}[1]{\textcolor{blue}{#1}}%
\newcommand{\emphtitle}[1]{\textcolor{yellow}{#1}}%
\renewcommand{\leq}{\leqslant}
\renewcommand{\geq}{\geqslant}

\graphicspath{{figures/}}
\addbibresource{main.bib}

\author{Urbain Vaes}{}
\date{}

\begin{document}

\begin{frame}[fragile]
  {Babylonian square root}
\begin{minted}{julia}
function babylonian_square_root(n)
    x = rand(1:n);
    while abs(x^2-n)>0.1
        x=0.5*(x+n/x);
    end
    println("sqrt($n)=$x")
end
\end{minted}

\begin{minted}{julia}
function babylonian_square_root(n)
    x0 = 0
    x1 = n / 2
    while x1 != x0
        x0 = x1
        x1 = 0.5 * (x0 + n / x0)
    end
    return x1
end
\end{minted}
\end{frame}

\begin{frame}[fragile]
  {Euclid's algorithm}
\begin{minted}{julia}
function euclid_gcd(a, b)
    if a>=b
        m=a
        n=b
    else
        m=b
        n=a
    end
    if n==0
        return m
    end
    r = m % n
    while r != 0
        m = n
        n = r
        r = m % n
    end
    return n
end
\end{minted}
\end{frame}

\begin{frame}[fragile]
  {Euclid's algorithm}
\begin{minted}[highlightlines={2-4}]{julia}
function euclid_gcd(a, b)
    if b > a
      return euclid_gcd(b, a)
    end
    if b==0
        return a
    end
    r = a % b
    while r != 0
        a = b
        b = r
        r = a % b
    end
    return b
end
\end{minted}
\end{frame}

\begin{frame}[fragile]
  {Euclid's algorithm}
\begin{minted}[highlightlines={2}]{julia}
function euclid_gcd(a, b)
    b == 0 && return a
    r = a % b
    while r != 0
        a = b
        b = r
        r = a % b
    end
    return b
end
\end{minted}
\end{frame}

\begin{frame}[fragile]
  {Euclid's algorithm}
\begin{minted}{julia}
function euclid_gcd(a, b)
    while b != 0
        a = b
        b = r
        r = a % b
    end
    return a
end
\end{minted}

or recursively\dots
\begin{minted}{julia}
function euclid_gcd(a, b)
    b == 0 && return a
    return euclid_gcd(b, a % b)
end
\end{minted}
\end{frame}

\begin{frame}[fragile]
  {Sieve of Eratosthenes}
  \begin{minted}{julia}
  function eratosthenes_sieve(n)
    A=ones(1,n);
    A[1]=0;
    i=2;
    while i<=sqrt(n)
       count = 0;
       if A[i]==1
          while i^2+count*i<=n
             j=i^2+count*i;
             A[j]=0;
             count = count+1;
          end
       end
       i=i+1;
    end
    primes = zeros(Int, 0)
    for i=1:n
       if A[i]==1
           append!(primes, i)
       end
    end
    return primes
  end
  \end{minted}
\end{frame}

\begin{frame}[fragile]
  {Sieve of Eratosthenes}
  \begin{minted}[highlightlines={4,12}]{julia}
  function eratosthenes_sieve(n)
    A=ones(1,n)
    A[1]=0
    for i in 2:floor(Int, sqrt(n))
       if A[i]==1
          for j in i^2:i:n
             A[j]=0
          end
       end
    end
    primes = zeros(Int, 0)
    for i=1:n
       if A[i]==1
           append!(primes, i)
       end
    end
    return primes
  end
  \end{minted}
\end{frame}

\begin{frame}[fragile]
  {Sieve of Eratosthenes}
  \begin{minted}[highlightlines={2,6}]{julia}
  function eratosthenes_sieve(n)
    A=ones(n)
    A[1]=0
    for i in 2:floor(Int, sqrt(n))
       if A[i]==1
          A[i^2:i:n] .= 0
       end
    end
    primes = zeros(Int, 0)
    for i=1:n
       if A[i]==1
           append!(primes, i)
       end
    end
    return primes
  end
  \end{minted}
\end{frame}

\begin{frame}[fragile]
  {Sieve of Eratosthenes}
  \begin{minted}[highlightlines={9}]{julia}
  function eratosthenes_sieve(n)
    A=ones(n)
    A[1]=0
    for i in 2:floor(Int, sqrt(n))
       if A[i]==1
          A[i^2:i:n] .= 0
       end
    end
    primes = [i for i in 1:n if A[i] == 1]
    return primes
  end
  \end{minted}
\end{frame}

\begin{frame}[fragile]
  {Sieve of Eratosthenes}
  \begin{minted}[highlightlines={9}]{julia}
  function eratosthenes_sieve(n)
    A=ones(n)
    A[1]=0
    for i in 2:floor(Int, sqrt(n))
       if A[i]==1
          A[i^2:i:n] .= 0
       end
    end
    return findall(x -> x == 1, A)
  end
  \end{minted}
\end{frame}

\begin{frame}[fragile]
  {Sieve of Eratosthenes}
  \begin{minted}[highlightlines={2}]{julia}
  function eratosthenes_sieve(n)
    A = ones(Bool, n)
    A[1] = false
    for i in 2:floor(Int, sqrt(n))
       if A[i]
          A[i^2:i:n] .= false
       end
    end
    return findall(A)
  end
  \end{minted}
\end{frame}

\end{document}

% vim: ts=2 sw=2 spelllang=fr spell
