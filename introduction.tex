\chapter*{Introduction}%
In a wide variety of scientific disciplines,
notably physics and chemistry,
the phenomena under consideration are well-described by mathematical equations.
More often than not,
these equations are too difficult to solve analytically,
and so one has to recur to \emph{numerical simulation} in order to calculate approximate solutions.
The aim of this course is to carefully study the standard numerical methods for many of the most common tasks in applied mathematics:
the solution of linear and nonlinear equations,
the solution of eigenvalue problems,
interpolation and approximation of functions,
and numerical integration using quadrature.
Throughout the course, the \texttt{Julia} programming language will be employed to exemplify the concepts.
Let us briefly describe the six topics covered in these lecture notes.

\begin{itemize}
    \item
        \textbf{Floating point arithmetic.}
        In~\cref{cha:rounding_errors},
        we discuss how real numbers are represented and stored on a computer.
        We discuss in particular the~\emph{IEEE 754} standard, which is widely accepted today
        and employed in most programming languages, including \texttt{Julia}.

    \item
        \textbf{Numerical integration.}
        In~\cref{cha:quadrature},
        we study numerical methods for computing definite integrals.
        The laws of nature are often expressed as differential or integral equations,
        so it is vital 
        we discuss how real numbers are represented and stored on a computer.
        We discuss in particular the~\emph{IEEE 754} standard, which is widely accepted today
        and employed in most programming languages, including \texttt{Julia}.

    \item
        \textbf{Solution of linear systems.}
        In \cref{cha:solution_of_linear_systems},
        we study the standard numerical methods for solving linear systems.
        Linear systems are ubiquitous is science,
        For example, they arise from the discretization of linear elliptic partial differential equations,
        using for example a finite difference method.
        A subclass of linear systems, in which the matrix on the left-hand side is positive semi-definite,
        arise in the context of optimization:
        if $A \in \real^{n \times n}$ is positive definite and $b \in \real^n$,
        then the unique minimizer of $\frac{1}{2} x^\t A x - b^\t x$ satisfies the linear system $A x = b$.

    \item
        \textbf{Solution of nonlinear equations.}
        Linear systems occur in all branches of science.
        For example, they arise from the discretization of linear elliptic partial differential equations,
        using for example finite difference method.
        A subclass of linear systems, in which the matrix on the left-hand side is positive semi-definite,
        arise in the context of optimization:
        if $A \in \real^{n \times n}$ is positive definite and $b \in \real^n$,
        then the unique minimizer of $\frac{1}{2} x^\t A x - b^\t x$ satisfies the linear system $A x = b$.
\end{itemize}
